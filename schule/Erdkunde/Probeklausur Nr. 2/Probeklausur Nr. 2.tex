\documentclass[12pt, a4paper]{report}
\usepackage[utf8]{inputenc}
\usepackage[T1]{fontenc}
\usepackage{german}
\usepackage[left=2cm,right=2cm,top=2cm,bottom=2cm]{geometry}
\usepackage{charter}
\usepackage{custompkg}

\begin{document}
	\bslinespacing{1.5}
	\title{Probeklausur Nr. 2}
	\author{Ben Siebert}
	\date{\today}
	\maketitle
	
	\paragraph{Aufgabe 1} \mbox{} \\
	Die Niederlande liegen auf der Nordhalbkugel auf dem europäischen Kontinent.
	Sie sind im Nordwesten des Kontinents verordnet.
	Im Norden grenzen die Niederlande an die Nordsee.
	Die Nachbarländer sind Deutschland im Osten und Belgien im Süden.
	Die Niederlande liegen bei etwa 52,63708° N, 5,67359° O.
	Die Hauptstadt der Niederlande ist Amsterdam, welche im Westen des Landes verordnet ist.
	Das Klima der Niederlande ist typisch für Mittel- bzw. Westeuropa.
	Es gibt relativ kalte Wintermonate und relative warme Sommermonate.
	Der Niederschlag ist gleichmäßig über das Jahr verteilt und beträgt insgesamt 815,5 mm.
	Die Durchschnittliche Temperatur liegt bei 10°C. \\
	Der Boden der Niederlande kann in zwei große Regionen unterteilt werden.
	Zum einen Marsch Boden, welcher im Westen bis Nordosten verordnet ist und zum anderen Podsol Böden, welche im Osten des Landes vertreten sind.
	Marsch muss vor der landwirtschaftlichen Nutzung trockengelegt werden, ist dafür aber auch ertragreich.
	Podsol hingegen muss nicht besonders behandelt werden, bevor er landwirtschaftlich genutzt werden kann, ist dafür allerdings ertragsarm (M2).
\end{document}
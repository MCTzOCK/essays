\documentclass[a4paper, 12pt]{report}
\usepackage{custompkg}
\usepackage[utf8]{inputenc}
\usepackage[T1]{fontenc}

\begin{document}
	\bslinespacing{1.5}
	\bsremovechaptertitle
	
	\newcommand{\kgk}[0]{\bsunit{kg}{Kopf}\ }
	
	\title{Markt- und exportorientiertes Agrobusiness – ein zukunftsfähiger Lösungsansatz?}
	\author{Ben Siebert}
	\date{\today}
	\maketitle
	\newpage
	
	\chapter{Aufgabe 1}

	Die Weltbevölkerung wird bis 2050 voraussichtlich um 35 \% wachsen, sodass 2050 etwa 10,8 Mrd. Menschen auf der Welt leben werden.
	Um diese Menge an Menschen versorgen zu können,
	muss die Pflanzenproduktion verdoppelt werden.
	Durch diese Verdoppelung könnten Konflikte entstehen,
	da mehr Fläche benötigt wird, die nicht vorhanden ist (M3).
	\\
	Der Anstieg des täglichen Proteinbedarfs pro Kopf ist besonders in Schwellen- und Entwicklungsländern sehr hoch,
	woraus zu folgern ist, dass ein höheres Bevölkerungswachstum in diesem Ländern vorliegt (M3).
	\\
	Die Oberfläche der Erde umfasst insgesamt 510 Mio. $km^{2}$ und ist zu 70,7 \% von Wasser bedeckt und ist somit unbrauchbar für die Landwirtschaft.
	Von der übrigen Fläche sind nur 46,5 \% unerschlossen, hierzu zählen etwa Wälder, Wüsten oder Hochgebirge.
	Von diesen unerschlossenen Flächen können nur die Wälder Landwirtschaftlich genutzt werden, was die Abholzung dieser mit sich führen würde. \\
	Der zweitgrößte Anteil an der nicht von Wasser bedeckten Oberfläche ist der Landwirtschaft zuzusprechen, welche etwa 50,2 Mio. $km^{2}$. \\
	Alle anderen Flächen machen nur insgesamt 19,4 Mio. $km^2$ aus.
	Zu diesen Flächen zählen ländliche Siedlungen und Betriebe, welche nur durch Land Grabbing für die Landwirtschaft nutzbar wären, städtische Gebiete, Erosionen, Minen und Straßen, welche nicht für die landwirtschaftliche Nutzung geeignet sind. Die einzigen nutzbaren anderen Flächen sind Forstpflanzungsgebiete und Abholzungsgebiete (M2).
	\\
	Der Fleischkonsum der Welt ist in den letzten 30 Jahren seit 1991 von 34 \kgk auf etwa 43.2 \kgk angestiegen. In Deutschland ist der Konsum insgesamt leicht rückgängig.
	1991 lag dieser noch bei 88 \kgk und ist bis 2020 auf etwa 85.9 \kgk gesunken. In den Vereinigten Staaten von Amerika ist der Konsum zwar bis 2011 zunächst von 114 \kgk auf 118 \kgk gestiegen, jedoch bis 2020 wieder auf 115,1 \kgk gefallen.
	Besonders gestiegen ist der Konsum in China.
	Hier lag er 1991 noch bei 27 \kgk, stieg jedoch bis 2011 auf 58 \kgk an.
	Im Jahr 2020 ist dieser weiterangestiegen und zwar auf 61.8 \kgk.

\end{document}
\documentclass[11pt, a4paper]{report}

\usepackage{custompkg}
\usepackage{ngerman}
\usepackage[utf8]{inputenc}
\usepackage[T1]{fontenc}

\begin{document}
	\bslinespacing{1.5}
	\bsremovechaptertitle

	\title{Kants kategorischer Imperativ}
	\author{Ben Siebert}
	\date{\today}
	\maketitle
	\chapter{Der kategorische Imperativ}
	
	\textit{\dq
		Handle nur nach derjenigen Maxime, durch die du zugleich wollen kannst, dass sie ein allgemeines Gesetz werde.\dq}
	
	\paragraph{Definition} \mbox{} \\
	Der kategorische Imperativ ist ein Leitsatz, der dazu dient Handlungen auf ihre moralische Richtigkeit zu prüfen.
	Er wird als \dq kategorisch\dq\ bezeichnet, da er allgemeingültig ist und somit nicht nur auf ein bestimmtes Ziel bezogen ist.
	
	\paragraph{Wann ist eine Handlung moralisch?} \mbox{} \\
	Man muss sich die Frage stellen, ob man möchte, dass alle anderen Menschen auch so handeln.
	Wenn die Antwort positiv ist, ist die Handlung moralisch vertretbar.
	Andernfalls ist sie nicht moralisch. 
	
	\paragraph{Was sind Maxime?} \mbox{} \\
	Maxime sind eine Art Regeln, nach denen eine Person gerne handelt.
	Diese Regeln sind mit persönlichen Leitsätzen vergleichbar.
	
	\paragraph{Goldene Regel} \mbox{} \\
	Die Goldene Regel lautet:
	\textit{\dq Was du nicht willst, das man dir tut, das füg auch keinem anderen zu.\dq}
	Der Unterschied zum kategorischen Imperativ besteht darin, dass die Goldene Regel sich nur auf deinen Willen bezieht und nich allgemeingültig ist.
\end{document}
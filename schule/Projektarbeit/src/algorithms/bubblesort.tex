%! Author = ben
%! Date = 23.10.2023

\documentclass[../entry.tex]{subfiles}
\usepackage{biblatex}
\usepackage{multirow}
\usepackage{colortbl}

\begin{document}
    % Source: https://informatik-bg.de/jg2-bpe-7-2-sortier-und-suchalgorithmen/bubble-sort (Oliver Kilthau, Berthold Metz) letzter Zugriff: 23.10.2023

    Das \dq BubbleSort\dq-Verfahren ist ein einfacher Sortieralgorithmus,
    der nach dem Prinzip des \dq Vergleichens und Vertauschens\dq\ arbeitet.
    Dabei werden die zu sortierenden Elemente paarweise verglichen und bei Bedarf vertauscht.
    Dieser Vorgang wird so lange wiederholt, bis die Elemente in der gewünschten Reihenfolge angeordnet sind.\\
    Das \dq BubbleSort\dq-Verfahren ist auf alle linearen Datenstrukturen anwendbar,
    die eine sequentielle Zugriffsmöglichkeit auf die Elemente bieten.
    %\footnote{\cite{informatik-bg}, \citetitle{informatik-bg}}
    \footnote{\bscite{informatik-bg}}

    \paragraph{Beispiel} \mbox{}\\

    \begin{table}[h]
        \centering
        \begin{tabular}{|c|c|c|c|}
            \hline
            \textbf{1. Durchlauf} & {\color{red}3} & {\color{red}1} & 2 \\
            \hline
            \textbf{2. Durchlauf} & 1 & {\color{red}3} & {\color{red}2} \\
            \hline
            \textbf{3. Durchlauf} & 1 & 2 & 3 \\
            \hline
        \end{tabular}
        \caption{Beispiel für das \dq BubbleSort\dq-Verfahren}
        \label{tab:bubblesort}
    \end{table}
\end{document}

%! Author = ben
%! Date = 24.10.2023

\documentclass[../entry.tex]{subfiles}

\begin{document}
    Das \dq QuickSort\dq-Verfahren ist im Vergleich den anderen Verfahren, ein sehr effizienter,
    aber auch komplexerer Sortieralgorithmus.
    Es basiert auf dem Prinzip \dq Teile und Herrsche\dq.
    Dabei wird zunächst ein Element aus der zu sortierenden Datenmenge ausgewählt,
    welches als \dq Pivotelement\dq\ bezeichnet wird.
    Anschließend werden alle Elemente der Datenmenge mit dem Pivotelement verglichen.
    Die Elemente, die kleiner als das Pivotelement sind, werden in eine Teilliste einsortiert.
    Die Elemente, die größer als das Pivotelement sind, werden in eine andere Teilliste einsortiert.
    Die beiden Teillisten werden anschließend rekursiv sortiert.
    \footnote{\bscite{quick-sort}}

    \paragraph{Beispiel} \mbox{}\\

    \begin{table}[h]
        \centering
        \begin{tabular}{|c|c|c|c|c|c|}
            \hline
            \textbf{1. Durchlauf} & 3 & 1 & {\color{red}2} & 5 & 4 \\
            \hline
            \textbf{2. Durchlauf} & 1 & {\color{red}2} & 3 & 5 & 4 \\
            \hline
            \textbf{3. Durchlauf} & 1 & 2 & 3 & {\color{red}4} & 5 \\
            \hline
            \textbf{4. Durchlauf} & 1 & 2 & 3 & 4 & 5 \\
            \hline
        \end{tabular}
        \caption{Beispiel für das \dq QuickSort\dq-Verfahren}
        \label{tab:quicksort}
    \end{table}
\end{document}

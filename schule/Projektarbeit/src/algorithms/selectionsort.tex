%! Author = ben
%! Date = 23.10.2023

\documentclass[../entry.tex]{subfiles}
\usepackage{biblatex}
\usepackage{multirow}
\usepackage{colortbl}

\begin{document}
    Das \dq SelectionSort\dq-Verfahren ist ebenfalls ein einfacher Sortieralgorithmus,
    der, wie das \dq BubbleSort\dq-Verfahren, nach dem Prinzip des \dq Vergleichens und Vertauschens\dq\ arbeitet.
    Dieses Verfahren ist auf alle linearen Datenstrukturen anwendbar,
    die eine sequentielle Zugriffsmöglichkeit auf die Elemente bieten.
    Der Algorithmus arbeitet wie folgt\footnote{\bscite{selection-sort}}:
    \begin{itemize}
        \item Das erste Element wird mit allen anderen Elementen verglichen.
        \item Das kleinste Element wird an die erste Stelle gesetzt.
        \item Das zweite Element wird mit allen anderen Elementen verglichen.
        \item Das kleinste Element wird an die zweite Stelle gesetzt.
        \item Dieser Vorgang wird so lange wiederholt, bis die Elemente in der gewünschten Reihenfolge angeordnet sind.
    \end{itemize}

    \paragraph{Beispiel} \mbox{}\\
    \begin{table}[h]
        \centering
        \begin{tabular}{|c|c|c|c|c|}
            \hline
            \textbf{1. Durchlauf} & 3 & 4 & {\color{red}1} & 2 \\
            \hline
            \textbf{2. Durchlauf} & 1 & 3 & 4 & {\color{red}2} \\
            \hline
            \textbf{3. Durchlauf} & 1 & 2 & 4 & {\color{red}3} \\
            \hline
            \textbf{4. Durchlauf} & 1 & 2 & 3 & 4 \\
            \hline
        \end{tabular}
        \caption{Beispiel für das \dq SelectionSort\dq-Verfahren}
        \label{tab:selectionsort}
    \end{table}


\end{document}

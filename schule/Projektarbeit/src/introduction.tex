%! Author = ben
%! Date = 23.10.2023

\documentclass[./entry.tex]{subfiles}
\usepackage{biblatex}
\usepackage{multirow}
\usepackage{colortbl}

\begin{document}
    \chapter{Einleitung}

    Sortieralgorithmen sind ein wichtiger Bestandteil der Informatik.
    Sie dienen allerdings nicht nur zum Sortieren von Daten,
    sondern auch als Grundlage für andere Algorithmen,
    denn Sortieralgorithmen sind in der Regel einfach zu implementieren
    und enthalten viele wichtige Konzepte der Informatik.
    Zu diesen essenziellen Konzepten gehören Schleifen,
    Variablen, lineare Datenstrukturen und Rekursion, welche
    alle in der Informatik häufige Anwendung finden.
    So bietet der QuickSort-Algorithmus zum Beispiel einen guten Einsteig in die Rekursion
    und der BubbleSort-Algorithmus in die Komplexitätsanalyse.
    Ebenfalls helfen Sortieralgorithmen dabei zu verstehen,
    wie Aufwandsanalysen und die damit verbunde \dq Big-O Notation\dq\ funktionieren. \\


    In dieser Arbeit werden zunächst die Funktionsweisen der Sortieralgorithmen
    BubbleSort, SelectionSort, InsertionSort und QuickSort erläutert.
    Zur besseren Veranschaulichung ist für ausgewählte Algorithmen der entsprechende Java-Quellcode im Anhang angegeben.
    Anschließend werden die Algorithmen im Bezug auf ihren Aufwand analysiert.
    Im Zuge dessen werden die Algorithmen mit einander verglichen.
    Abschließend wird anhand von Beispielen erläutert, wie die Funktionsweise
    von Sortieralgorithmen Schülern anschaulich im Unterricht vermittelt werden kann. \\

    Im Anhang sind die wichtigsten fachspezifischen Begriffe und Konzepte
    angegeben, die in dieser Arbeit verwendet werden. \\

    \paragraph{Definition Sortieralgorithmus}
    Ein Sortieralgorithmus ist ein Algorithmus, der eine Menge von Elementen in eine bestimmte Reihenfolge bringt.
    Die Reihenfolge wird durch eine Relation zwischen den Elementen festgelegt.
    Die Relation kann zum Beispiel eine Ordnungsrelation (kleiner als, größer als) sein.
    Sortieralgorithmen können auf alle linearen Datenstrukturen angewendet werden,
    die eine sequentielle Zugriffsmöglichkeit auf die Elemente bieten.
    Die im Anhang aufgeführten Java-Quellcodes sind immer auf Arrays angewendet,
    da Arrays die einfachsten und bekanntesten linearen Datenstrukturen sind.

\end{document}
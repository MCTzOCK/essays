\documentclass[a4paper, 12pt]{report}

\usepackage{german}
\usepackage{bookman}
\usepackage[T1]{fontenc}
\usepackage[utf8]{inputenc}
\usepackage{./custompkg}
\usepackage{xcolor}
\usepackage{graphicx}

\begin{document}
	\thispagestyle{empty}
	
	\bslinespacing{1.5}
	{
		\centering
		\Huge
		\color{blue}
		Warum brauch ich das?
	}
	
	\raggedright
	\paragraph{\color{blue}Beispiel} \mbox{} \\
	
	Stell dir vor, dass deine Klasse vor 2 Wochen eine Arbeit in Mathe geschrieben hat
	und heute hat der Lehrer diese zurückgegeben hat.
	Du willst eine Übersicht über den Notenschnitt haben, aber der Lehrer möchte dir diesen nicht mitteilen.
	Deswegen bittest du alle sich in einer Reihe aufzustellen.
	Jetzt könntest du einfach nach und nach zählen, aber wirst wahrscheinlich irgendwann durcheinander kommen und dich verzählen.
	Deswegen willst du die Schüler nach ihren individuellen Noten sortieren, damit du diese Noten nach und nach betrachten kannst.
	Jetzt gibt es viele verschiedene Möglichkeiten, wie deine Mitschüler sich in der richtigen Reihenfolge aufstellen können.
	Für Menschen ist das Sortieren einer solchen Datenmenge keine Herausforderung.
	Wenn es darum geht, dass Computer sortieren müssen, wird dies schnell zu einer komplexen Aufgabe.
	Deswegen haben sich ein paar sehr schlaue Menschen Gedanken darüber gemacht, wie Computer möglichst schnell etwas sortieren können.
	\\
	Sortieralgorithmen werden also benötigt, weil der Computer nur mit Nullen und Einsen \dq denken\dq\ kann und nicht wie der Mensch mittels Intuition handelt.

\end{document}
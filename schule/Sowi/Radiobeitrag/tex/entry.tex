\documentclass[11pt,a4paper]{article}
\usepackage{ngerman}
\usepackage[utf8]{inputenc}
\usepackage[T1]{fontenc}
\usepackage{longtable,multicol}

\begin{document}
    \textsc{Radiobeitrag}

    \begin{longtable}[c]{p{3cm}p{9cm}}\\
        \multicolumn{2}{c}{\textsl{Radiobeitrag über Bürgergeld}} \\\\
        Moderator: & Liebe Zuhöhrer, heute diskutieren wir über das Bürgergeld!
        Das Bürgergeld ist eine neue Form der Sozialhilfe, welche Hartz IV ersetzen soll.
        Es unterstützt Menschen, die nicht genug Geld haben, um ihren Lebensunterhalt zu bestreiten.
        Zugeschaltet ist uns Herr Birke.
        Herr Birke, was denken Sie über das Bürgergeld? \\\\
        Herr Birke: & Ich denke, dass das Bürgergeld eine gute Idee ist, weil
        es die Bürokratie im Vergleich zu Hartz IV vereinfacht. \\\\
        Moderator: & Was sind denn die Hauptunterschied zwischen Hartz IV und dem
        Bürgergeld Ihrer Ansicht nach? \\\\
        Herr Birke: & Was mir direkt ins Auge gestochen ist, dass die Leistung des Bürgergeldes
        direkt von der Inflation abhängt, wodurch die Legitimität im Bezug zur Gerechtigkeit erhöht wird.
        Momentan liegt der Regelsatz bei 502 Euro, was für viele Menschen zu wenig ist. \\\\
        Moderator: & Wie sieht es denn mit Sanktionen aus?
        Diese waren ja bei Hartz IV relativ stark. \\\\
        Herr Birke: & Das Bürgergeld hat genauso, wie Hartz IV eine Mitwirkungspflicht, also zum Beispiel die Pflicht, sich um eine Arbeit zu bemühen.
        Allerdings sind die Sanktionen nicht so stark wie bei Hartz IV, sodass maximal 30 \% des Regelsatzes gestrichen werden können.
        Dies ist im Gegensatz zu Hartz IV, bei welchem die Leistungen komplett gestrichen werden konnten, deutlich humaner. \\\\
        Moderator: & Wann hat man denn einen Anspruch auf das Bürgergeld?
        Hat sich das im Vergleich zu Hartz IV geändert? \\\\
        Herr Birke: & Der Anspruch auf das Bürgergeld ist im Vergleich zu Hartz IV deutlich erweitert worden.
        Als bedürftig gilt jemand, der ein Vermögen von weniger als 40.000 Euro hat, was deutlich mehr ist, als bei Hartz IV, wo es nur 10.050 Euro waren.
        Wenn man in einer Bedarfsgemeinschaft lebt, gelten die 40.000 Euro für die erste Person und für jede weitere Person liegt die Grenze bei 15.000 Euro.
        Allerdings sinkt die Vermögensgrenze auf 15.000 Euro, wenn eine Person länger als 12 Monate Bürgergeld bezieht. \\\\
        Moderator: & Also ist das Bürgergeld deutlich großzügiger als Hartz IV. Wie sieht es denn mit der Wohnung aus? \\\\
        Herr Birke: & Im ersten Jahr gibt es kein Höchstgrenze für Miet- und Heizkosten.
        Auch die Größe der Wohnung ist für das erste Jahr nicht begrenzt und die Kosten werden ohne Obergrenze übernommen und orientieren sich nicht, wie bei Hartz IV am bestimmten Vergleichsraum.
        Wenn eine Person allerdings länger als zwei Jahre Bürgergeld bezieht, gelten die gleichen Beschränkungen wie bei Hartz IV. \\\\
        Moderator: & Gibt es Ihrer Ansicht nach noch Verbesserungspotential? \\\\
        Herr Birke: & Ja, ich denke, dass die Vermögensgrenze von 40.000 Euro zu hoch ist.
        Die meisten Gründe, warum Menschen in die Armut abrutschen, sind unvorhersehbar, wie zum Beispiel eine Krankheit oder eine Scheidung, welche durch die Kinder in die Armut führt, aber auch Migration spielt eine große Rolle.
        Kinder tragen zur Armut der Eltern bei, da sie viel Zeit benötigen und die Eltern dadurch weniger Zeit für die Arbeit haben.
        Meiner Meinung nach sollte es mehr Förderung für Migranten, wie zum Beispiel Sprachkurse, aber auch eine separate Sozialhilfe für kranke geben.
        Ebenfalls sollte es günstige Angebote für Kinderbetreuung geben, damit die Eltern mehr Zeit für die Arbeit haben. \\\\
        Moderator: & Vielen Dank für das Interview! Danke für die Informationen! \\\\
    \end{longtable}

\end{document}

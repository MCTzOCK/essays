\documentclass[12pt,a4paper]{report}

\usepackage{ngerman}
\usepackage[T1]{fontenc}
\usepackage[utf8]{inputenc}
\usepackage{custompkg}
\usepackage{bookman}
\usepackage[left=2cm,right=2cm,top=2cm,bottom=2cm]{geometry}

\begin{document}
	\bslinespacing{1.75}
	\centering
	\LARGE
	\textbf{Probeklausur}
	\normalsize
	\leftskip
	
	\paragraph{1. Aufgabe - Analyse des Gedichtes} \mbox{} \\
	Das Gedicht \dq Der Jäger Abschied\dq, welches von Joseph von Eichendorff im Jahr 1810 veröffentlicht wurde und somit der Epoche der Romantik zuzuordnen ist, thematisiert die Schönheit und Eigenschaften der Natur. \\
	Ich lege meiner Analyse die Hypothese zu Grunde, dass das Gedicht eine Reise durch die Natur beschreibt und währenddessen auf die entsprechenden Besonderheiten eingeht. \\
	Das Gedicht besteht aus vier Strophen á sechs Versen.
	Das Reimschema das Gedichtes ist ein umschlossener Reim, welcher in den letzten beiden Versen jeder Strophe unterbrochen wird.
	Auffallend ist, dass die letzten beiden Verse jeder Strophe der gleichen Struktur folgen.
	Der vorletzte Vers jeder Strophe ist immer \dq Lebe wohl \dq\ (vgl. V. 5), während der letzte Vers jeder Strophe immer mit \dq du schöner Wald\dq\ (vgl. V. 6) endet.
	Das Metrum des Gedichtes ist ein Trochäus.
	Der Sprecher des Gedichtes ist ein lyrisches Ich, was durch die Verwendung der Personalpronomen der ersten Person (vgl. V. 19)  deutlich wird. \\
	Die erste Strophe des Gedichtes thematisiert den Aufbau des Waldes und seine Schönheit.
	In der zweiten Strophe beschreibt das lyrische Ich seinen Weg durch den Wald, auf dem es verschiedene Tiere sieht.
	In der dritten Strophe wird anschließend ebenfalls die Schönheit des Waldes verdeutlicht.
	Die vierte und somit letzte Strophe thematisiert schlussendlich den Abschied von dem Wald. \\
	Durch die Anapher \dq Lebe wohl\dq\ (V. 5,11,17) wird die Emotionalität des Abschieds des lyrischen Ichs vom Wald betont.
	Die Metapher \dq du schöner Wald\dq\ (V. 6,12,18) veranschaulicht eine vergangene Zeit, die sich dem Ende zugewandt hat.
	Die Personifikation \dq Tiefe die Welt verworren schallt\dq\ (V. 7) beschreibt die Welt als verworren, als ob sie Geräusche von sich gibt und stellt die Welt somit als lebendig bzw. aktiv dar.
	Allerdings wird auch die Kontrastreichheit der Natur durch die Antithese \dq Oben einsam Rehe grasen, Und wir ziehen fort und blasen\dq\ (V. 9) beschrieben, wodurch die Kontraste zwischen Mensch und Natur verdeutlicht werden.
	Das Gedichts besteht im Prinzip ausschließlich aus Enjambements, wodurch sich der Fluss der Gedanken über das Zeilenende fortsetzt.
	Dies schafft ein fließenden Übergang zwischen den einzelnen Gedanken. \\
	Insgesamt zieht sich das Reisemotiv durch das ganze Gedicht.
	Es geht um einen Abschied von dem Wald, aber auch um die Kontraste, die in der Natur existieren.
	Jedoch steht alles unter dem Gedanken des schnellen und dynamischen Reisens, was vor allem durch die vielen Enjambements deutlich wird.
	Hiermit ist meine Deutungshypothese bestätigt.
	
	\paragraph{Aufgabe 2 - Vergleich der Gedichte} \mbox{} \\
	Im Folgenden vergleiche ich die beiden Gedichte \dq Der Jäger Abschied\dq\ von Eichendorff und \dq ausflug\dq\ von Kirsten im Bezug auf das \dq Reisemotiv\dq.
	Das Gedicht von Kirsten thematisiert eine sehr schnell vorbeiziehende Welt, was durch die \dq pfeilschnell überrollt[en]\dq\ (V. 2) Straßen aus Asphalt deutlich wird.
	Ebenfalls wird die Vergänglichkeit der Natur während der Reise betont. \\
	In beiden Gedichten wird das Reisemotiv als sehr hektisch beschrieben.
	Das lyrische Ich nimmt dieses als sehr schnell auf.
	Ebenfalls auffallend ist, dass das trotz der, in beiden Gedichten sehr hektisch wirkenden Reise, viele Aspekte der Natur detailiert beschrieben werden und immer wieder die gleichen Aspekte aufgegriffen werden.	
\end{document}
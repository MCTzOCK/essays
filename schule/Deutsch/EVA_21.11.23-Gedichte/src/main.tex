%! Author = ben
%! Date = 28.11.2023

% Preamble
\documentclass[12pt, a4paper]{article}

% Packages
\usepackage[left=2cm, right=2cm, top=2cm, bottom=2cm]{geometry}
\usepackage[T1]{fontenc}
\usepackage[utf8]{inputenc}
\usepackage{ngerman}
\usepackage{bookman}

% Document
\begin{document}
    \renewcommand{\baselinestretch}{1.5}\normalsize

    \title{EVA 21.11.23 Gedichte}
    \author{Ben Siebert}
    \date{\today}
    \maketitle

    \thispagestyle{empty}
    \clearpage

    \newpage
    \setcounter{page}{1}

    \section{Aufgabe 1}
    \paragraph{Aufgabenstellung} \mbox{} \\
    Lies die beiden Gedichte und schreibe zu beiden einen Einleitungssatz, in welchem du möglichst konkret das jeweilige Thema benennst.

    \paragraph{Heinrich Heine, Die Harzreise} \mbox{} \\
    Das Gedicht \dq Die Harzreise\dq\ von Heinrich Heine, welches 1824 veröffentlicht wurde, thematisiert eine Reise durch den Harz, bei welcher vor allem die Natur und die Landschaft im Vordergrund stehen.

    \paragraph{Heinrich Heine, Lyrisches Intermezzo} \mbox{} \\
    Das Gedicht \dq Lyrisches Intermezzo\dq\ von Heinrich Heine, welches 1823 veröffentlicht wurde, thematisiert die Liebe und die Natur, welche in einem Kontrast zueinander stehen.

    \section{Aufgabe 2}
    \paragraph{Aufgabenstellung} \mbox{} \\
    Lies die Info zum uneigentlichen Sprechen und arbeite anschließend stichpunktartig heraus, worauf sich Kritik und Spott in den beiden Gedichten beziehen und welche Intentionen mit dem uneigentlichen Sprechen jeweils verfolgt werden.


\end{document}

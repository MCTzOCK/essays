\documentclass[11pt,a4paper]{report}
\usepackage{bookman}
\usepackage[utf8]{inputenc}
\usepackage[T1]{fontenc}
\usepackage{german}
\usepackage[left=2cm,right=2cm,top=2cm,bottom=2cm]{geometry}
\usepackage{custompkg}

\begin{document}
	\bslinespacing{1.5}
	\clearpage
	\thispagestyle{empty}
	\paragraph{Sprachliche Analyse \dq das Schiff\dq} \mbox{} \\
	In der ersten Strophe wird eine sehr bedrohliche Situation beschrieben, welche vor allem durch die Metapher des \dq rote[n] Mond[es]\dq\ (V. 3) deutlich wird.
	Verstärkt wird diese Wirkung durch das Adjektiv \dq schaukelnd\dq\ (V. 2) und das Substantiv \dq Haie\dq\ (V. 3).
	Hierdurch wird die ganze Lage des lyrischen Ichs als auswegslos und gefährlich beschrieben.
	Durch die Verwendung von Wörtern, die vor allem mit dem Verfall in Verbindung gebracht werden, wie \dq fault\dq\ (V. 4), \dq schlissen\dq\ (V. 4) und \dq modern\dq\ (V. 5) wird der Ernst der Situation nochmals untermalt.
	Ebenfalls kommt es dem lyrischen Ich so vor, als ob der Horizont \dq entfernter [...] und bleicher \dq\ (V. 6) ist, was die Hoffnungslosigkeit des lyrischen Ichs darstellt.
	In der zweiten Strophe verdeutlicht das lyrische ich seine depressive Haltung zum Leben durch den Ausspruch \dq Fühl ich tief, daß ich vergehen soll\dq\ (V. 9).
	Dies wird durch seine wehrlose Haltung veranschaulicht (vgl. V. 11f).
	Diese Haltung wird durch die, in Strophe eins beschriebene, düstere Stimmung hervorgerufen.
	Die Metapher \dq Ließ ich mich den Wassern ohne Groll.\dq\ (V. 12)	 verdeutlicht, dass das lyrische Ich zwar keineswegs zufrieden ist, aber sich auch nicht gegen die Gegebenheiten auflehnt.
	
\end{document}














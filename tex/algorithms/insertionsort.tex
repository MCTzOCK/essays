%! Author = ben
%! Date = 24.10.2023

\documentclass[./entry.tex]{subfiles}
\usepackage{biblatex}
\usepackage{multirow}
\usepackage{colortbl}

\begin{document}
    Das \dq InsertionSort\dq-Verfahren funktioniert nach dem Prinzip des \dq Vergleichens und Einfügens\dq.
    Wie das \dq SelectionSort\dq-Verfahren und das \dq BubbleSort\dq-Verfahren ist es auf alle linearen Datenstrukturen anwendbar,
    die eine sequentielle Zugriffsmöglichkeit auf die Elemente bieten.
    Um eine solche Datenstruktur zu sortieren, werden folgende Schritte durchgeführt\footnote{\bscite{insertion-sort}}:
    \begin{itemize}
        \item Das erste Element wird als sortiert angesehen.
        \item Das zweite Element wird mit dem ersten Element verglichen.
        \item Das zweite Element wird an die richtige Stelle eingefügt.
        \item Das dritte Element wird mit dem zweiten Element verglichen.
        \item Das dritte Element wird an die richtige Stelle eingefügt.
        \item Dieser Vorgang wird so lange wiederholt, bis die Elemente in der gewünschten Reihenfolge angeordnet sind.
    \end{itemize}

    \paragraph{Beispiel} \mbox{}\\

    \begin{table}[h]
        \centering
        \begin{tabular}{|c|c|c|c|c|}
            \hline
            \textbf{1. Durchlauf} & 3 & {\color{red}1} & 2 & 4 \\
            \hline
            \textbf{2. Durchlauf} & 1 & 3 & {\color{red}2} & 4 \\
            \hline
            \textbf{3. Durchlauf} & 1 & 2 & 3 & 4 \\
            \hline
        \end{tabular}
        \caption{Beispiel für das \dq InsertionSort\dq-Verfahren}
        \label{tab:insertionsort}
    \end{table}

\end{document}
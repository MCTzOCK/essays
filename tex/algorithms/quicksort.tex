%! Author = ben
%! Date = 24.10.2023

\documentclass[./entry.tex]{subfiles}
\usepackage{biblatex}
\usepackage{multirow}
\usepackage{colortbl}

\begin{document}
    Das \dq QuickSort\dq-Verfahren ist im Vergleich zu anderen Verfahren,
    wie dem \dq BubbleSort\dq-Verfahren, ein sehr effizienter, aber auch komplexerer
    Sortieralgorithmus.
    Es basiert auf dem Prinzip \dq Teile und Herrsche\dq.
    Dabei wird die zu sortierende Datenmenge in zwei Teilmengen aufgeteilt.
    Die Elemente der ersten Teilmengen sind kleiner als die Elemente der zweiten Teilmengen.
    Anschließend werden die Teilmengen rekursiv sortiert.
    \footnote{\bscite{quicksort}}

    \paragraph{Beispiel} \mbox{}\\

    \begin{table}[h]
        \centering
        \begin{tabular}{|c|c|c|c|c|c|}
            \hline
            \textbf{1. Durchlauf} & 5 & 1 & 3 & 2 & 4\\
            \hline
            \textbf{2. Durchlauf} & 1 & 3 & 2 & 4 & 5\\
            \hline
            \textbf{3. Durchlauf} & 1 & 2 & 3 & 4 & 5\\
            \hline
        \end{tabular}
        \caption{Beispiel für das \dq QuickSort\dq-Verfahren}
        \label{tab:quicksort}
    \end{table}
\end{document}
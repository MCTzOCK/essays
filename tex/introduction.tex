%! Author = ben
%! Date = 23.10.2023

\documentclass[./entry.tex]{subfiles}
\usepackage{biblatex}
\usepackage{multirow}
\usepackage{colortbl}

\begin{document}
    \chapter{Einleitung}

    Sortieralgorithmen sind ein wichtiger Bestandteil der Informatik.
    Sie dienen allerdings nicht nur zum Sortieren von Daten,
    sondern auch als Grundlage für andere Algorithmen,
    denn Sortieralgorithmen sind in der Regel einfach zu implementieren
    und enthalten viele wichtige Konzepte der Informatik.
    Zu diesen essenziellen Konzepten gehören Schleifen,
    Variablen, Lineare Datenstrukturen und Rekursion, welche
    alle in der Informatik häufige Anwendung finden.
    So bietet der Quicksort-Algorithmus zum Beispiel ein guter Einstieg in die Rekursion
    und der BubbleSort-Algorithmus ein guter Einstieg in die Komplexitätsanalyse.
    Ebenfalls helfen Sortieralgorithmen dabei, zu verstehen,
    wie Aufwandsanalysen und die damit verbunde \dq Big-O Notation\dq funktionieren. \\

    In dieser Arbeit werden zunächst die vier Sortieralgorithmen BubbleSort, SelectionSort, InsertionSort und Quicksort vorgestellt.
    Anschließend wird die Laufzeit der Algorithmen analysiert und verglichen.

\end{document}
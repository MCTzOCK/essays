%! Author = ben
%! Date = 23.10.2023

\documentclass[./entry.tex]{subfiles}
\usepackage{biblatex}
\usepackage{multirow}
\usepackage{colortbl}
\usepackage{lipsum}

\begin{document}
    \chapter{Anhang}

    \section{Fachspezifische Erklärungen}

    \paragraph{Lineare\ Datenstrukturen}
    Lineare Datenstrukturen sind Datenstrukturen, die die Elemente in einer bestimmten Reihenfolge speichern. Die
    Reihenfolge wird durch die Reihenfolge der Elemente bestimmt. Die Elemente werden nacheinander gespeichert und
    können nur über die Position im Speicher angesprochen werden. Die bekannteste linearen Datenstrukture ist
    das Array.\footnote{\bscite{informatik-bg}}
    \paragraph{Array}
    Ein Array ist eine Datenstruktur, die eine feste Anzahl von Elementen enthält, die alle den gleichen Datentyp
    haben. Die Elemente werden in einem Array in aufeinanderfolgenden Speicherplätzen gespeichert und können über einen
    Index angesprochen werden. Der Index ist eine ganze Zahl, die angibt, an welcher Stelle
    sich das Element im Array befindet. Der Index des ersten Elements ist 0, der des zweiten Elements 1 usw. Der Index
    des letzten Elements ist die Länge des Arrays minus 1. Die Länge eines Arrays ist die Anzahl der Elemente, die es
    enthält. Die Länge eines Arrays ist immer konstant.

\end{document}
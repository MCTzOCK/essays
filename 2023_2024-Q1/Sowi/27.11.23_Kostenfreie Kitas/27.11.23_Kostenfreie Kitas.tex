\documentclass[a4paper]{report}

\usepackage[T1]{fontenc}
\usepackage[utf8]{inputenc}
\usepackage{custompkg}

\begin{document}
	\bslinespacing{1.25}
	\bsremovechaptertitle
	\bsremoveparttitle
	
	\title{Sollen Kitas kostenlos sein?}
	\author{Ben Siebert}
	\date{\today}
	\maketitle
	
	\paragraph{Argumente für kostenfreie Kitas} \mbox{}
	\begin{itemize}
		\item Kitas stellen den Einstieg in ein Leben in der Gesellschaft dar, weil sie grundlegende Kompetenzen, wie Sprechen und soziale Interaktion vermitteln. Ohne diese Kompetenzen ist ein Leben in einer Gesellschaft nicht möglich.
		\item Kitas bilden die Grundlage der Bildung und sind eine Investition des Staates in seine Bürger.
		\item Die Chancengleichheit wird gestärkt, da die Kosten für einen Kita-Platz nicht mehr vom Bundesland und Träger abhängig sind.
		\item Frühkindliche Erziehung ist in wirksames Instrument gegen Bildungsungerechtigkeit.
	\end{itemize}

	\paragraph{Argumente gegen kostenfreie Kitas} \mbox{}
	\begin{itemize}
		\item Die ohnehin schon limitierten Kita-Plätze werden noch limitierter, sodass es für Eltern noch schwerer wird, einen Platz für ihre Kinder zu finden.
		\item Das Geld, welches dem Staat ohnehin schon für mehrere Aspekte fehlt, wird noch limitierter.
		\item Kostenfreie Kitas helfen reichen Menschen, die es garnicht nötig hätten.
		\item Stärkung sozialer Separierung durch Kitas in freien Trägern (bspw. Betriebskitas), da diese von der Beitragsfreiheit ausgenommen sind.
	\end{itemize}

\end{document}
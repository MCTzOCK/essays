\documentclass[a4paper, 12pt]{report}
\usepackage[left=2cm,right=2cm,top=2cm,bottom=2cm]{geometry}
\usepackage[utf8]{inputenc}
\usepackage[T1]{fontenc}
\usepackage{charter}
\usepackage{custompkg}

\begin{document}
	\bslinespacing{1.5}
	\bsremovechaptertitle
	\noindent
	\chapter{S. 400 Nr. 1: Bedingungsloses Grundeinkommen}
	Das bedingungslose Grundeinkommen ist eine Idee, die darauf abzielt, dass jeder Bürger einer Gesellschaft einen gewissen Geldbetrag erhält, unabhängig davon, ob dieser arbeiten geht, oder nicht.
	Jeder würde auf den gleichen Grundsatz aufbauen und wenn jemand mehr Mittel für seinen Lebensstil benötigt müsste dieser einer Arbeit nachgehen.
	Dieses Grundeinkommen orientiert sich am Mindestlohn.
	
	\chapter{S. 400 Nr. 2: Auswertung der Studie}
	Auffallend ist, dass überwiegend junge Menschen, die einen höheren Bildungsabschluss besitzen das bedingungslose Grundeinkommen befürworten.
	Allerdings sind auch politisch links verordnete Bürger mit einem geringen Einkommen Befürworter sind.
	Zwar haben über die Hälfte der Befragten angegeben, dass sie grundsätzlich für das bedingungslose Grundeinkommen sind, allerdings sind nur insgesamt 16 \% der Befragten sehr dafür.
	Insgesamt ist eine Spaltung in dieser Frage deutlich zu erkennen, da ebenfalls 16 \% absolut dagegen sind.
	
	\chapter{S. 400 Nr. 3 (M19b): Analyse der Argumentation}
	Butterwegge beginnt seine Argumentation damit, dass laut ihm die Reichen das bedingungslose Grundeinkommen nicht benötigen und es für die Armen viel zu gering ist.
	Er stellt zudem die Behauptung auf, dass das BGE im Gegensatz zum Wohlfahrtsstaat steht und somit nicht mit dem Prinzip der Sozialversicherungen vereinbar ist.
	Ebenfalls gibt Butterwegge an, dass das BGE für das Leben eines reichen Müßiggängers konzipiert wurde und so den Armen nicht weiterhilft. 
	Er argumentiert, dass das BGE unfassbar teuer für den Staat wäre, wodurch die öffentliche Armut gesteigert wird.
	Butterwegge kritisiert, dass das BGE nicht auf die individuellen Lebensbedingungen angepasst wäre und so beispielsweise Schwerbehinderte nicht den angemessenen Betrag erhalten würden.
	 Zuletzt kritisiert er, dass Menschen, die nicht arbeiten gehen, durch das BGE von dem Teil der Bevölkerung finanziert werden würden, die arbeiten gehen. 
	
\end{document}
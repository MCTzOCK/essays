\documentclass[11pt,a4paper]{article}
\usepackage[left=2cm,right=2cm,top=2cm,bottom=2cm]{geometry}
\usepackage{ngerman}
\usepackage[utf8]{inputenc}
\usepackage[T1]{fontenc}
\usepackage{bookman}

\begin{document}
    \section{Radiobeitrag}

    Das Bürgergeld soll also das neue Hartz IV sein?
    Aber ist es wirklich eine Verbesserung?
    Das Bürgergeld ist eine Art der Sozialhilfe, die ab 2023 in Deutschland eingeführt wurde, um Menschen zu unterstützen, die nicht genug Geld für die Deckung ihrer Grundbedürfnisse besitzen.
    \\\\
    Der Anspruch auf das Bürgergeld ist im Vergleich zu Hartz IV deutlich erweitert worden.
    Als bedürftig gilt jemand, der ein Vermögen von weniger als 40.000 Euro hat.
    Raten sie mal, wie viel es bei Hartz IV war.
    Es waren nur 10.050 Euro.
    Wenn man in einer Bedarfsgemeinschaft lebt, gelten die 40.000 Euro für die erste Person und für jede weitere Person liegt die Grenze bei 15.000 Euro.
    Allerdings sinkt die Vermögensgrenze auf 15.000 Euro, wenn eine Person länger als 12 Monate Bürgergeld bezieht.
    Das ist dennoch um Welten mehr als bei Hartz IV.
    \\\\
    Im ersten Jahr gibt es kein Höchstgrenze für Miet- und Heizkosten.
    Auch die Größe der Wohnung ist für das erste Jahr nicht begrenzt und die Kosten werden ohne Obergrenze übernommen und orientieren sich nicht, wie bei Hartz IV am bestimmten Vergleichsraum.
    Wenn eine Person allerdings länger als zwei Jahre Bürgergeld bezieht, gelten die gleichen Beschränkungen wie bei Hartz IV.
    \\\\
    Durch die Vermögensgrenzen und die Übernahmen der Kosten wurde die Effizienz des Bürgergeldes, hinsichtlich der Wirksamkeit, drastisch erhöht.
    \\\\
    Was mir direkt ins Auge gestochen ist, ist dass die Leistung des Bürgergeldes direkt von der Inflation abhängt, wodurch die Effizienz im Bezug auf Wirksamkeit erhöht wurde.
    Ebenfalls wurde der Regelsatz auf 502 Euro erhöht, wodurch die Legitimität hinsichtlich der Rechtmäßigkeit gesteigert wurde.
    \\\\
    Was Sie erstaunen wird ist, dass das Bürgergeld  die Bürokratie erleichtert, welche bei Hartz IV ein großes Problem war.
    Bei Hartz IV war das Problem, dass die Beantragung mit viel Bürokratie verbunden und viele Bedürftige nicht in der Lage waren, die Anträge auszufüllen.
    \\\\
    Durch diese Änderung wurde die Effizienz im Bezug auf die Durchsetzbarkeit erhöht.
    \\\\
    Vertreter der Hartz IV Sozialhilfe begründen ihre Position dadurch, dass bei Hartz IV durch die komplette Streichung der Leistungen, die Beziehenden zum Arbeiten regelrecht gezwungen wurden.
    Die komplette Streichung ist viel zu hart, da so die Leute eventuell ihre Existenzgrundlage verlieren.
    Das Bürgergeld hat genauso, wie Hartz IV eine Mitwirkungspflicht, also die Pflicht, sich um Arbeit zu bemühen.
    Allerdings sind die Sanktionen nicht so stark wie bei Hartz IV, sodass maximal 30 \% des Regelsatzes gestrichen werden können.
    Dies ist im Gegensatz zu Hartz IV, bei welchem die Leistungen komplett gestrichen werden konnten, deutlich humaner und fördert die Legitimität in Anbetracht der Rechtmäßigkeit.
    Sie müssen sich nur einmal vorstellen, dass die Menschen, die sowieso schon kein Geld mehr hatten, gar keine Unterstützung mehr erhalten haben.
    \\\\
    Hartz IV folgt somit dem Prinzip der Leistungsgerechtigkeit, während das Bürgergeld dem Prinzip der Bedarfsgerechtigkeit folgt.
    \\\\
    Trotz der positiven Aspekte des Bürgergeldes sind mir auch einige negative Aspekte aufgefallen.
    Zum Beispiel sind die häufigsten Gründe, warum Menschen in Armut abrutschen, unvorhersehbar, wie eine Krankheit oder eine Scheidung, welche durch die Kinder in die Armut führt, aber auch Migration spielt eine große Rolle.
    Kinder tragen zur Armut der Eltern bei, da sie viel Zeit benötigen und die Eltern dadurch weniger Zeit für die Arbeit haben.
    \\\\
    Dies ist ein Problem für die Effizienz der Nützlichkeit.
    \\\\
    Meiner Meinung nach ist das Bürgergeld eine deutliche Verbesserung zu Hartz IV.
    Allerdings sollte es mehr Förderung für Migranten, wie zum Beispiel Sprachkurse, aber auch eine separate Sozialhilfe für Kranke geben.
    Ebenfalls sollte es günstige Angebote für Kinderbetreuung geben, damit die Eltern mehr Zeit für die Arbeit haben.

\end{document}

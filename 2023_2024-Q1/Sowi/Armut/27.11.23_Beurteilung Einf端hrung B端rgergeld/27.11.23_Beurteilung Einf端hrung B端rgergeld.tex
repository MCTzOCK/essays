\documentclass[a4paper]{report}

\usepackage{german}
\usepackage[T1]{fontenc}
\usepackage[utf8]{inputenc}
\usepackage{custompkg}
\usepackage{eurosym}

\begin{document}
	\bslinespacing{1.5}
	
	\title{Beurteilung der Einführung \\ des Bürgergeldes}
	\author{Ben Siebert}
	\date{\today}
	\maketitle
	\newpage
	
	Die Einführung des Bürgergeldes hat eine Namensänderung von \dq Hartz IV\dq\ auf \dq Bürgergeld\dq\ hervorgerufen.
	Dieser Name reduziert das Stigma und eliminiert so das schlechte Image, das Hartz IV zuvor hatte. Dies ist sowohl legitim, als auch effizient.
	Des Weiteren wurde die Effizienz der Sozialhilfe gesteigert,
	indem eine Weiterbildungsprämie von 150 \euro\ für Jugendliche eingeführt wurde.
	Ebenfalls effizienzsteigert für Jugendliche ist, dass diese sich nun etwas hinzuverdienen dürfen,
	ohne, dass es auf den Satz angerechnet wird. \\
	
	Dennoch bringt die Einführung des Bürgergeldes auch schlechtes mit sich.
	So ist beispielsweise der Grenzsteuersatz sehr hoch, sodass fast der gesamte Zuverdienst auf das Bürgergeld angerechnet wird, was auf sich negativ die Effizienz auswirkt.
	Außerdem ist der zuständige Mitarbeiter im Jobcenter
	vollkommen alleine verantwortlich für die
	Einschätzung der Zumutbarkeit einer Tätigkeit, was wiederum einen negativen Einfluss auf die Legitimität des Bürgergeldes nimmt.
	Das Bürgergeld ist aber vor allem auch nicht effizient, da viele kranke oder behinderte Menschen es beziehen, weil sie nicht arbeiten gehen können.
	\\
	\\
	Abschließend komme ich zu dem Urteil, dass das Bürgergeld nicht effizient ist, da
	es Menschen beziehen, die nicht arbeiten gehen können, wobei der Sinn hinter dem Bürgergeld eigentlich ist, die Beziehenden wieder zum Arbeiten zu bringen.
	Außerdem ist es nicht legitim, da die Einschätzung der Zumutbarkeit einer Tätigkeit einzig und allein im Ermässen des Sachbearbeiters liegt.
	
\end{document}
\documentclass[12pt,a4paper]{report}
\usepackage[T1]{fontenc}
\usepackage[utf8]{inputenc}
\usepackage{charter}
\usepackage{ngerman}
\usepackage[left=2cm,right=2cm,top=2cm,bottom=2cm]{geometry}


\begin{document}
	\section{Definitionen Globalisierung}
	\begin{itemize}
		\item Internationalisierung
		\item Abhängigkeit / Verbindung
		\item Prozess / Entwicklung
	\end{itemize}
	\section{Ursachen}
	\begin{itemize}
		\item Globale Kommunikation (geringe Preise für globale Informationen)
		\item niedrige Treibstoffpreise (generel) und Transportkosten (Luftfahrt und Seefahrt)
		\item Öffnung der Märkte (Beseitigung von Handelshemmnissen)
	\end{itemize}
	\section{Dimensionen}
	\begin{itemize}
		\item Wirtschaft
		\begin{itemize}
			\item globale Warenexporte verdreißichfacht
			\item Wachstum des Weldhandels
			\item Direktinvestionen deutlich gestiegen
			\item 65.000 Multinationale Unternehmen
		\end{itemize}
		\item Kultur
		\begin{itemize}
			\item Westliche Kultur setzt sich weltweit durch (Vielfalt wird zerstört)
			\item Rückbesinnung auf lokale und regionale Geschäfte
			\item Gesellschaftlicher Austausch
			\item Globale Öffentlichkeit
		\end{itemize}
		\item Politik
		\begin{itemize}
			\item Klimawandel, Terrorismus
			\item NGOs (Nichtregierungsorganistationen)
		\end{itemize}
		\item Umwelt
		\begin{itemize}
			\item Emissionen - Umweltbelastung
		\end{itemize}
	\end{itemize}
	\section{Folgen}
	\begin{itemize}
		\item Chancen für Schwellenländer
		\item Terrorismus, Klimawandel
		\item Befreiung von Bürgern aus der Armut
		\item Eher negativ für afrikanische Länder, da nicht ausreichen vorbereitet (Zerstörung örtlicher Produktion, keine Direktinvestitionen)
		\item Chancen und Gefahren für Industrieländer: Neue Märkte $\to$ Neue Konkurrenten
	\end{itemize}
\end{document} 
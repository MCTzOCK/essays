\documentclass[11pt,a4paper]{report}
\usepackage[T1]{fontenc}
\usepackage[utf8]{inputenc}
\usepackage{bookman}
\usepackage{ngerman}
\usepackage[left=2cm,right=2cm,top=2cm,bottom=2cm]{geometry}

\begin{document}
	\Large Klassengesellschaft
	\normalsize
	\vspace{1cm}
	\paragraph{Geschichte:}
	\begin{itemize}
		\item Das Konzept wurde in der Mitte des 19. Jahrhunderts von Karl Marx begründet
		\item Eine Klassengesellschaft zeichnet sich durch die Existenz von zwei Bevölkerungsgruppen aus: Bourgeoisie und Proletariat
		\item Bourgeoisie sind die reichen Bürger
		\item Zum Proletariat zählen die Lohnarbeiter, welcher ihre Arbeitskraft verkaufen müssen
		\item Besitztümer waren der größte Determinator für soziale Ungleichheit
	\end{itemize}
	\paragraph{Kriterien einer Klassengesellschaft:} \mbox{} \\
	\begin{itemize}
		\item Die soziale Lage wird mit anderen geteilt
		\item Die soziale Lage ist dauerhaft
		\item Die soziale Lage wird an die Kinder weitergegeben
	\end{itemize}
	\paragraph{Ist Deutschland eine Klassengesellschaft?} \mbox{} \\
	\begin{itemize}
		\item Es gibt einen engen Zusammenhang zwischen der sozialen Herkunft und den Bildungschancen
		\item Wenn Eltern keine/wenig Bildung haben, werden die Kinder ebenfalls keine/wenig Bildung erhalten
		\item Arbeitslosigkeit konzentriert sich auf bestimmte Bildungsgruppen
	\end{itemize}
	Laut den Klassenkriterien ist Deutschland also eine Klassengesellschaft.
	Nach Karl Marx allerdings nicht, da es keine klar getrennten zwei bestimmten Großklassen gibt (Bourgeoisie und Proletariat)
\end{document}
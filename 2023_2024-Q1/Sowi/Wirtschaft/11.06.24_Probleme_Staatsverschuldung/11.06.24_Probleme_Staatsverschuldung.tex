\documentclass[12pt,a4paper]{report}
\usepackage[T1]{fontenc}
\usepackage[utf8]{inputenc}
\usepackage{charter}
\usepackage{ngerman}
\usepackage[left=2cm,right=2cm,top=2cm,bottom=2cm]{geometry}

\renewcommand\thesection{\arabic{section}.} 

\begin{document}
	\section{S. 97 Nr. 2a) Probleme der Staatsverschuldung}
	\begin{itemize}
		\item intertemporale Lastenverschiebung in die Zukunft (spätere Generationen werden an Investitionen beteiligt)
		\item Staatsverschuldung führt meist nicht direkt zu kostendeckenden Erträge (aber stärken meist das Produktionspotential und das BIP, sodass sie indirekt zu staatlichen Mehreinnahmen führen.)
		\item Staatsverschuldungen sind nur begrenzt kalkulierbar
		\item Es ist nicht sicher, dass spätere Generationen eine Investition als Bereicherung ansehen
		\item staatliche zinsrobuste Kreditaufnahmen können private Investoren verdrängen, sodass ein Wachstumsverlust auftreten kann
		\item Die vorgesehene Symmetrie der Staatsverschuldung, also die Rückführung in der Boomphase ist auf Grund Hemmfaktoren in politischen Entscheidungsprozessen meist irreal
		\item Der haushaltspolitische Spielraum wird durch die Zinslast massiv eingeschränkt
	\end{itemize}
	\section{S. 97 Nr. 3}
	\begin{itemize}
		\item Begrenzung der Staatsverschuldung soll das Hauptthema sein
		\item Man ist sich uneinig, wie das Defizit gesenkt werden kann (Steuererhöhung (für Wohlhabende), weniger Staatsausgaben, Steuersenkung als Anreiz für Ausgaben)
		\item 2009 Einführung der Schuldenbremse
		\item Haushalt soll ausgeglichen sein (Innerhalb von Bund und Ländern)
		\item 2011 Beginn des Schuldenabbaus
		\item 2015 1. Mal Einnahmeüberschuss
		\item Ausnahmeregelung für die Schuldenbremse sind
		\begin{itemize}
			\item Kreditaufnahme bei konjunktureller Entwicklung
			\item Krisen
		\end{itemize}
	\end{itemize}
\end{document}
\documentclass[12pt,a4paper]{report}
\usepackage[T1]{fontenc}
\usepackage[utf8]{inputenc}
\usepackage{charter}
\usepackage{ngerman}
\usepackage[left=2cm,right=2cm,top=2cm,bottom=2cm]{geometry}

\renewcommand\thesection{\arabic{section}.}

\begin{document}
	\section{S. 28 Nr. 1}
	\paragraph{Einschätzung der wirtschaftlichen Lage Deutschlands:}
	\begin{itemize}
		\item Abschwung der Industriekonjunktur dauert an
		\item Binnennachfrage ist weiterhin intakt
		\item exportorientierte Industrie steht wegen Handelskonflikten, Brexit und Unsicherheiten im außenwirtschaftlichen Umfeld unter Druck
		\item Binnenkonjunktur ist weiterhin intakt
	\end{itemize}
	\paragraph{Faktoren, mit denen diese Einschätzung begründet wird:}
	\begin{itemize}
		\item Handelskonflikte
		\item Brexit
		\item Unsicherheiten
	\end{itemize}
	\paragraph{Einschätzung über mögliche Entwicklungen in der Zukunft:}
	\begin{itemize}
		\item etwa 45.5 Millionen Erwerbstätige bis Ende 2020
		\item Außenhandel wird stärker expandieren
		\item Rückgang der Investitionen im Privatsektor
		\item Expansion bleibt durch das knappe Arbeitskräfteangebot begrenzt
	\end{itemize}
	\paragraph{Verfahren zur Erstellung von Prognosen zur wirtschaftlichen Entwicklung:}
	\begin{itemize}
		\item Analyse historischer Daten und Trends: Dies beinhaltet die Untersuchung vergangener Wirtschaftsdaten, um Muster und Trends zu identifizieren, die für zukünftige Vorhersagen nützlich sein können.
		\item Anwendung statistischer Modelle und 				Algorithmen: Statistische Methoden und 				maschinelles Lernen werden eingesetzt, um aus 				den historischen Daten Prognosen abzuleiten.
		\item Berücksichtigung von wirtschaftlichen 				Indikatoren: Wichtige Indikatoren wie das 				Bruttoinlandsprodukt (BIP), Inflation und 				Arbeitslosenquote fließen in die Analyse ein.
		\item Durchführung von Umfragen und Befragungen: Die 				Meinungen und Einschätzungen von Experten, 				Unternehmen und Verbrauchern werden gesammelt, 				um ein umfassendes Bild der Marktlage zu 				erhalten.
	\end{itemize}
	\paragraph{Datenbasis, auf deren Grundlage die befragten Personen und das Bundeswirtschaftsministerium zu der jeweiligen Einschätzung kommt:}
	\begin{itemize}
		\item Befragungen von Unternehmen und Verbänden: 				Diese geben Aufschluss über die aktuelle 				Geschäftslage und Erwartungen.
		\item Öffentliche Finanzdaten und 				Wirtschaftsindikatoren: Dazu gehören 				Informationen über das BIP, Verbraucherpreise 				und Arbeitsmarktdaten.
		\item Prognosen von Forschungsinstituten: 				Einrichtungen wie das ifo Institut 				veröffentlichen regelmäßig ihre Einschätzungen 				zur wirtschaftlichen Entwicklung
	\end{itemize}
	\section{S. 31 Nr. 1 Gruppe 2 Punkt 2}
	\paragraph{Akzelerationsprinzip} Das Akzelerationsprinzip besagt, dass eine zusätzliche Investition oder Ausgaben zu einem erhöhten Anstieg der Gesamtwirtschaftsleistung führen können. 
	\paragraph{Multiplikationseffekt}
Der Multiplikationseffekt beschreibt, wie sich eine anfängliche Investition oder Ausgabe in der Wirtschaft ausbreitet und zu weiteren Ausgaben führt, die wiederum das Einkommen und die Produktion erhöhen. Kurz gesagt, eine Ausgabe führt zu mehr Ausgaben, was zu einem größeren Gesamteffekt führt.
\end{document}
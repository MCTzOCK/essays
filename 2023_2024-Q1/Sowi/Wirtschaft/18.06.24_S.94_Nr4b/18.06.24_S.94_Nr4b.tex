\documentclass[12pt,a4paper]{report}
\usepackage[T1]{fontenc}
\usepackage[utf8]{inputenc}
\usepackage{charter}
\usepackage{ngerman}
\usepackage[left=2cm,right=2cm,top=2cm,bottom=2cm]{geometry} 

\renewcommand\thesection{\arabic{section}.}

\begin{document}
	\section{S. 94 Nr. 7b}
	\paragraph{Pro-Argumente}:
	\begin{itemize}
		\item Der Staat hat aktuell hohe Einnahmen durch gestiegene Steueraufkommen und gesunkene Zinsausgaben.
		\item Dies führt dazu, dass es finanziell möglich ist, keine neuen Schulden aufzunehmen.
		\item Durch die Koalitionsverträge von Union und SPD sind die Ausgaben zurückgegangen, was die Haushaltslage entspannt hat.
		\item Keine neuen Schulden aufzunehmen, bedeutet, dass der Staat nicht auf Kreditgeber angewiesen ist.
		\item Dies schützt vor Zinsänderungen und möglichen wirtschaftlichen Abhängigkeiten.
		\item Die Schuldenbremse trägt zur langfristigen finanziellen Stabilität bei und schützt zukünftige Generationen vor einer hohen Schuldenlast.
	\end{itemize}
	
	\paragraph{Contra-Argumente}:
	\begin{itemize}
		\item Eine zu strikte Schuldenbremse kann zu einem Investitionsstau führen, da notwendige Investitionen in Infrastruktur und andere wichtige Bereiche nicht durchgeführt werden.
		\item Zukünftige Generationen werden mit den Herausforderungen des demografischen Wandels konfrontiert, wie z. B. einer höheren Anzahl an Rentnern und einer geringeren Zahl an Erwerbstätigen.
		\item Dies könnte es schwieriger machen, ohne neue Schulden auszukommen.
		\item Öffentliche Ausgaben sind notwendig, um die Wirtschaft zu fördern und wichtige soziale und infrastrukturelle Projekte zu finanzieren.
		\item Ein ausgeglichener Haushalt sollte nicht auf Kosten dringend benötigter Investitionen gehen.
		\item Eine zu rigide Schuldenbremse schränkt die Flexibilität der Finanzpolitik ein und macht es schwieriger, auf wirtschaftliche Krisen oder unvorhergesehene Ereignisse angemessen zu reagieren.
	\end{itemize}
\end{document}
\documentclass[a4paper, 12pt]{report}
\usepackage{ngerman}
\usepackage{charter}
\usepackage[T1]{fontenc}
\usepackage[utf8]{inputenc}
\usepackage{custompkg}
\usepackage[left=2cm,right=2cm,top=2cm,bottom=2cm]{geometry}

\begin{document}
	\bslinespacing{1.5}
	\bsremovechaptertitle
	\chapter{S. 46 Nr. 2a}
	Bund und Länder müssen auf das Gleichgewicht der Wirtschaft achten und eine Stabilität der Preise fördern.
	Ebenfalls wichtig ist das außerwirtschaftliche Gleichgewicht und das Schaffen neuer Arbeitsplätze.
	Hierbei ist vor allem das Wirtschaftswachstum zu beachten.
	\chapter{S. 46 Nr. 3}
	Das \dq magische Viereck\dq\ beinhaltet alle Zentralen Aspekte und Ziele der Wirtschafts- und Finanzpolitik.
	Diese Ziele sind:
	\begin{itemize}
		\item Hoher Beschäftigungsgrad: Arbeitslosenrate möglichst niedrig halten
		\item Stabiles Preisniveau: Inflationsrate nicht über 2\%
		\item Außenwirtschaftliches Gleichgewicht: Etwa gleich viel Import und Export, damit kein Nachteil für andere Länder entsteht.
		\item Angemessenes Wirtschaftswachstum: Beständiges, gleichmäßiges Wachstum, damit Unternehmer und Bürger vorausschauend handeln können
	\end{itemize}
	Aus den \dq magischen Viereck\dq\ hat sich das \dq magische Sechseck\dq\ gebildet.
	Dieses schließt, neben der ursprünglichen vier Zielen auch die folgenden Ziele mit ein:
	\begin{itemize}
		\item Gerechte Einkommens- und Vermögensverteilung: Leistungsgerechtigkeit (mehr Leistung entspricht mehr Geld; weniger Leistung entspricht weniger Geld)
		\item Schutz der natürlichen Umwelt: das Wirtschaften sollte nachhaltig gestaltet sein.
	\end{itemize}
\end{document}
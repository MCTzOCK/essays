\documentclass[12pt,a4paper,landscape]{report}
\usepackage[T1]{fontenc}
\usepackage[utf8]{inputenc}
\usepackage{charter}
\usepackage[ngerman]{babel} \usepackage[left=2cm,right=2cm,top=2cm,bottom=2cm]{geometry} 
\usepackage{tabularx}
\usepackage{pdflscape}

\begin{document}
	\noindent
	\Large
	S. 25 Nr. 2
	\vspace{0.1cm}
	\hrule
	\vspace{.2cm}
	\large
	\noindent
	Der Konjunkturzyklus beginnt im sogenannten Konjunkturtief.
	In diesem lassen sich brachliegende Kapazitäten und Arbeitslosigkeit feststellen.
	Anschließend folgt der Aufschwung in welchem ein Anstieg der Preise, Nachfrage, Gewinne, Investitionen und Löhne festzustellen ist.
	Der höchste Abschnitt im Konjunkturzyklus ist die sogenannte Hochkonjunktur oder auch Boom bezeichnet.
	Nach dem Boom folgt der Abschwung, in dem die alles was im Aufschwung anstieg wieder fällt, allerdings insgesamt höher bleibt als zuvor.
	Nach dem Abschwung wird alles wiederholt.
	\\[1cm]
	\Large S. 25 Nr. 3
	\vspace{0.1cm}
	\hrule
	\vspace{.2cm}
	\large
	\noindent
	\paragraph{Konjunktur} Die Konjunktur beschreibt die aktuelle Gesamtlage einer Wirtschaft zu einem bestimmten Zeitpunkt.
	Die Konjunkturlage und die Konjunkturschwankungen werden durch bestimmte Konjunkturindikatoren angezeigt.
	Hierbei ist der größte Maßstab das BIP.
	\paragraph{Konjunkturzyklus} 	Der Konjunkturzyklus beinhaltet alle Phasen der Konjunktur und zwar Aufschwung, Boom, Rezession / Abschwung und Depression / Tief.
	Dieser Zyklus wird immer wieder in der gleichen Reihenfolge wiederholt.
	Allerdings ist beim Konjunkturverlauf keine Regelmäßigkeit zu erkennen, da dieser von globalen Einflüssen, wie zum Beispiel einer Wirtschaftskrise abhängig ist.
	\newpage
	\noindent
	\Large S. 25 Nr. 4
	\vspace{0.1cm}
	\hrule
	\vspace{1cm}
	\large
	\noindent
	\begin{tabularx}{\linewidth}{|X|X|X|X|X|}
		\hline
		& Aufschwung & Boom & Abschwung & Depression \\
		\hline
		Beschäftigung & nimmt zu & Vollbeschäftigung & nimmt ab (Auftragsmangel; Kurzarbeit; Entlassungen) & Arbeitslosigkeit \\
		\hline
		Nachfrage & nimmt zu & bleibt auf dem gleichen Niveau & nimmt ab & so gut wie nicht vorhanden \\
		\hline
		Produktion / Investition & steigt an & voll ausgelastet & geringe Auslastung & so gut wie nicht vorhanden \\
		\hline
		Preise & steigen an & Erhöhung des Preisniveaus & sinken & extrem gering \\
		\hline
	\end{tabularx}
	
\end{document}
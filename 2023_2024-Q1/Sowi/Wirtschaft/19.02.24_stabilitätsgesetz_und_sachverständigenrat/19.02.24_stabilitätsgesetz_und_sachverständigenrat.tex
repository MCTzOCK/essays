\documentclass[a4paper, 12pt]{report}
\usepackage[utf8]{inputenc}
\usepackage[T1]{fontenc}
\usepackage{ngerman}
\usepackage{custompkg}
\usepackage{charter}
\usepackage[left=2cm,right=2cm,top=2cm,bottom=2cm]{geometry}

\begin{document}
	\bslinespacing{1.5}
	\bsremovechaptertitle
	\chapter{S. 46 Nr. 1: Entstehung des Stabilitätsgesetzes}
	Das Stabilitätsgesetz entstand in einer Zeit, in der die Arbeitslosigkeit in Deutschland einen Höhepunkt erreichte.
	1967 wurde es anschließend mit überwiegender Mehrheit im Bundestag und Bundesrat verabschiedet.
	Das Gesetzt beinhaltet eine Stabilität des Preisniveaus, einen hohen Beschäftigungsstand und das außenwirtschaftliche Gleichgewicht bei stetigem und angemessenen Wirtschaftswachstum.
	Zu den konjunkturpolitischen Instrumenten zählen:
	\begin{itemize}
		\item Konjunkturausgleichsrücklagen
		\item Investmentprämien
		\item variable Einkommenssteuersätze
	\end{itemize}
	Die Notwendigkeit für ein solches Gesetz kam durch den extremen Wirtschaftseinbruch der Bundesrepublik Deutschland in den Vorjahren zustande.
	So stand zum Beispiel die Bundespost kurz vor der Zahlungsunfähigkeit und Siemens konnte auf dem Kapitalmarkt fast kein Geld mehr auftreiben.
	Auch die Papiere an der Börse schlitterten zu dieser Zeit in den Keller.
	\chapter{S. 49 Nr. 1a: Sachverständigenrat}
	Der \dq Sachverständigenrat zur Begutachtung der gesamtwirtschaftlichen Entwicklung\dq\ hat durch einen gesetzlichen Auftrag einige Aufgaben erhalten.
	Dazu zählen:
	\begin{enumerate}
		\item Die Darstellung der wirtschaftlichen Lage und die Prognose für die absehbare Zukunft
		\item Überprüfen, wie die, im Rahmen der marktwirtschaftlichen Ordnung, geregelten Aspekte (Stabilität des Preisniveaus, hoher Beschäftigungsstand und außenwirtschaftliches Gleichgewicht) gewährleistet werden können
		\item Feststellung und Aufzeichnung der Ursachen, die zu aktuellen und potenziellen Spannungen zwischen Angebot und Nachfrage führen.
	\end{enumerate}
	Jedes Jahr muss dieser Rat eine Jahresgutachten veröffentlichen.
	Der Rat besteht aus fünf Mitgliedern, welche alle besondere Erfahrungen und Kenntnisse im Bezug auf Wirtschaft haben.
	Diese Mitglieder werden vom Bundespräsidenten auf Vorschlag der Bundesregierung berufen und können mehrfach berufen werden.
	Der Präsident dieses Rates wird von den Mitgliedern selbst für eine Dauer von 3 Jahren gewählt.
\end{document}
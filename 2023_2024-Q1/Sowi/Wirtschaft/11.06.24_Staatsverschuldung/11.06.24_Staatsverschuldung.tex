\documentclass[12pt,a4paper]{report}
\usepackage[T1]{fontenc}
\usepackage[utf8]{inputenc}
\usepackage{charter}
\usepackage{ngerman}
\usepackage[left=2cm,right=2cm,top=2cm,bottom=2cm]{geometry}
\usepackage{amsmath}

\renewcommand\thesection{\arabic{section}.} 

\begin{document}
	\section{S. 93 Nr. 3}
	\begin{itemize}
		\item Staatsverschuldung entspricht allen von der öffentlichen Hand aufgenommenen Krediten
		\item Schulden in Deutschland kommen vom Bund, den Ländern oder den Sozialversicherungen
		\item Für politische und ökonomische Bewertung sind sowohl Neuverschuldungen, als auch die Gesamtverschuldung von Bedeutung
		\item Von 1950-2014 mehr öffentliche Ausgaben als Einnahmen (fehlender Betrag musste mit Krediten gedeckt werden)
		\item Durch viele Schulden muss ein größerer Anteil der Staatseinnahmen für diese verwendet werden, sodass andere Bereiche, wie Investitionen weniger Geld zur Verfügung haben.
		\item Pro Haushaltsjahr tatsächlich aufgenommene Schulden werden Bruttokreditaufnahmen bezeichnet
		\item Endscheidend ist die Nettokreditaufnahme bzw. die Neuverschuldung, da in jeder Periode auch einige Kredite abbezahlt werden:
	\end{itemize}
	\begin{align*}
		\text{Nettokreditaufnahme} &= \text{Bruttokreditaufnahme} - \text{Tilgung}
	\end{align*}
\end{document}
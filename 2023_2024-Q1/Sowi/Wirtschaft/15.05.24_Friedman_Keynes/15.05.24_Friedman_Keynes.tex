\documentclass[12pt,a4paper]{report}
\usepackage[T1]{fontenc}
\usepackage[utf8]{inputenc}
\usepackage{charter}
\usepackage{ngerman}
\usepackage[left=2cm,right=2cm,top=2cm,bottom=2cm]{geometry}

\renewcommand\thesection{\arabic{section}.} 

\begin{document}
	\section{Keynes}
	\begin{itemize}
	    \item Betonte die Notwendigkeit staatlicher Intervention in die Wirtschaft, insbesondere in Zeiten der Rezession.
	    \item Befürwortete eine expansive Fiskalpolitik, einschließlich staatlicher Ausgaben und Investitionen, um die Nachfrage zu stimulieren.
	    \item Glaubte an die Wirksamkeit der Geldpolitik, war jedoch der Meinung, dass sie in Zeiten wirtschaftlicher Krisen nicht ausreichte, um die Nachfrage anzukurbeln.
   	 \item Zielte darauf ab, Vollbeschäftigung und Wirtschaftswachstum durch staatliche Maßnahmen zu erreichen.
	\end{itemize}

	\section{Friedman}
	\begin{itemize}
	    \item Befürwortete eine begrenzte Rolle der Regierung in der Wirtschaft und betonte die Bedeutung freier Märkte.
	    \item Glaubte an die Theorie des Monetarismus, wonach die Geldmenge die wichtigste Determinante für das Wirtschaftswachstum ist.
	    \item Plädierte für eine stabile Geldpolitik, um Inflation und andere wirtschaftliche Probleme zu kontrollieren.
	    \item Argumentierte gegen expansive Fiskalpolitik und betonte die negativen Auswirkungen von staatlichen Eingriffen auf die Wirtschaft.
	\end{itemize}
	
	
	\section{Antizyklische Fiskalpolitik}
Antizyklische Fiskalpolitik bezieht sich auf staatliche Maßnahmen, die darauf abzielen, wirtschaftliche Schwankungen auszugleichen. Obwohl diese Politik viele Vorteile bietet, birgt sie auch Risiken und Nachteile:

	\begin{itemize}
	    \item \textbf{Risiken:}
	    \begin{itemize}
	        \item \textbf{Zeitverzögerung:} Es kann eine Zeitverzögerung geben, bis die Auswirkungen der Politik spürbar werden, was zu ineffektiven Maßnahmen führen kann.
	        \item \textbf{Überschuldung:} Eine übermäßige Nutzung der antizyklischen Fiskalpolitik kann zu einer Verschuldung des Staates führen, was langfristig nachteilige Auswirkungen haben kann.
	   \end{itemize}    
	   \item \textbf{Nachteile:}
    	\begin{itemize}
        	\item \textbf{Fehlallokation von Ressourcen:} Eine zu starke Intervention des Staates kann zu einer Fehlallokation von Ressourcen führen, da Unternehmen und Verbraucher möglicherweise nicht mehr aufgrund von Marktkräften handeln.
       	\item \textbf{Abhängigkeit:} Eine zu starke Abhängigkeit von antizyklischer Fiskalpolitik kann die Eigenverantwortung von Unternehmen und Verbrauchern untergraben und zu einer dauerhaften Abhängigkeit vom Staat führen.
   	\end{itemize}
	\end{itemize}

\section{Angebotsorientierte Wirtschaftspolitik}
Die angebotsorientierte Wirtschaftspolitik konzentriert sich darauf, die Produktionsfaktoren zu verbessern, um langfristiges Wirtschaftswachstum zu fördern. Trotz ihrer potenziellen Vorteile gibt es auch Risiken und Nachteile:

\begin{itemize}
    \item \textbf{Risiken:}
    \begin{itemize}
        \item \textbf{Langfristige Auswirkungen:} Die positiven Auswirkungen der angebotsorientierten Politik können langfristig spürbar sein, was kurzfristig zu geringeren Verbesserungen führen kann.
        \item \textbf{Ungleichheit:} Wenn die Politik nicht darauf abzielt, die Verteilung der Gewinne aus dem Wirtschaftswachstum zu verbessern, könnte sie zu einer Zunahme der Ungleichheit führen.
    \end{itemize}
    
    \item \textbf{Nachteile:}
    \begin{itemize}
        \item \textbf{Langsame Reaktion:} Angebotsorientierte Politiken können langsamere Reaktionen auf kurzfristige wirtschaftliche Schocks haben, da sie darauf abzielen, langfristige Strukturverbesserungen zu erzielen.
        \item \textbf{Soziale Kosten:} Maßnahmen wie Deregulierung könnten zu sozialen Kosten führen, wie Umweltverschmutzung oder Arbeitsplatzunsicherheit.
    \end{itemize}
\end{itemize}

Insgesamt gibt es bei beiden Ansätzen Vor- und Nachteile, und ihre Effektivität hängt von verschiedenen Faktoren ab, einschließlich der spezifischen wirtschaftlichen Bedingungen und der Umsetzung der Politik.


\end{document}
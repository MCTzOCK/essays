\documentclass[12pt,a4paper]{report}
\usepackage[T1]{fontenc}
\usepackage[utf8]{inputenc}
\usepackage{charter}
\usepackage{ngerman}
\usepackage[left=2cm,right=2cm,top=2cm,bottom=2cm]{geometry}

\renewcommand\thesection{\arabic{section}.} 

\begin{document}
	\section{Aufgabe}
	
	\begin{enumerate}
   	\item \textbf{Aufschwung (Expansion)}
   	\begin{itemize}
      	\item Wirtschaftswachstum: Steigend
      	\item Beschäftigungsniveau: Zunehmend
      	\item Preisniveau: Leicht steigend
       	\item Lohnentwicklung: Steigend
    	\end{itemize}
	   \item \textbf{Hochkonjunktur (Boom)}
    	\begin{itemize}
      	\item Wirtschaftswachstum: Hoch, aber stabil
      	\item Beschäftigungsniveau: Hoch, Vollbeschäftigung
      	\item Preisniveau: Stark steigend, Inflation möglich
      	\item Lohnentwicklung: Stark steigend
    	\end{itemize}
    	\item \textbf{Abschwung (Rezession)}
    	\begin{itemize}
      	\item Wirtschaftswachstum: Abnehmend
      	\item Beschäftigungsniveau: Sinkend
      	\item Preisniveau: Stagnierend oder leicht fallend
      	\item Lohnentwicklung: Stagnierend oder leicht fallend
    	\end{itemize}
    	\item \textbf{Tiefstand (Depression)}
    	\begin{itemize}
      	\item Wirtschaftswachstum: Sehr gering oder negativ
      	\item Beschäftigungsniveau: Niedrig, hohe Arbeitslosigkeit
      	\item Preisniveau: Niedrig oder deflationär
      	\item Lohnentwicklung: Sinkend oder stagnierend
    	\end{itemize}
	\end{enumerate}
	
	\section{Aufgabe}
	\paragraph{a)}
	\paragraph{b)}
	
	\section{Aufgabe}
	
	\begin{itemize}
   	\item Vor allem (neo-)liberale Ökonomen fordern regelmäßig eine Lockerung der gängigen Arbeitsschutzregelungen (z.B. des Kündigungsschutzes) in Deutschland.
   	\item Bewertung dieser Maßnahmen anhand eines sozialen und eines ökonomischen Arguments:
   	\begin{itemize}
      	\item \textbf{Soziales Argument}
      	\begin{itemize}
         	\item \textbf{Pro:} Erhöhung der Flexibilität für Unternehmen: Unternehmen können schneller auf wirtschaftliche Veränderungen reagieren und Beschäftigungsentscheidungen treffen, was insgesamt die Anpassungsfähigkeit und Resilienz des Arbeitsmarktes stärkt.
          	\item \textbf{Kontra:} Erhöhte Arbeitsplatzunsicherheit: Arbeitnehmer könnten sich unsicherer fühlen und ständig um ihren Arbeitsplatz fürchten, was zu psychischem Stress und reduzierter Lebensqualität führen kann.
        \end{itemize}
        \item \textbf{Ökonomisches Argument}
        \begin{itemize}
        		\item \textbf{Pro:} Steigerung der Wettbewerbsfähigkeit: Durch den Abbau von Hürden wie strikten Kündigungsschutzregelungen könnten Unternehmen schneller und effizienter Personalentscheidungen treffen und sich so besser an Marktveränderungen anpassen.
            \item \textbf{Kontra:} Potenzielle Abnahme der Arbeitsproduktivität: Unsicherheit kann die Motivation und Loyalität der Mitarbeiter verringern, was zu geringerer Produktivität und möglicherweise zu höherer Mitarbeiterfluktuation führt.
        \end{itemize}
    	\end{itemize}
    	\item Eigenständiges Urteil:
    	\begin{itemize}
      	\item Während eine Lockerung der Arbeitsschutzregelungen ökonomisch die Flexibilität und Wettbewerbsfähigkeit der Unternehmen erhöhen könnte, muss dies gegen die sozialen Kosten abgewogen werden.
      	\item Insbesondere die potenziell erhöhte Unsicherheit und psychische Belastung der Arbeitnehmer könnten langfristig negative Effekte auf die Gesellschaft haben.
      	\item Eine ausgewogene Reform, die beide Aspekte berücksichtigt und möglicherweise durch begleitende soziale Maßnahmen unterstützt wird, könnte daher der sinnvollste Ansatz sein.
    	\end{itemize}
	\end{itemize}
	
\end{document}
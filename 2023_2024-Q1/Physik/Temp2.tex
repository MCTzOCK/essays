\documentclass{report}

\usepackage{german}
\usepackage[T1]{fontenc}
\usepackage[utf8]{inputenc}
\usepackage{tikz}
\usepackage{pgfplots}
\usepackage{custompkg}

\begin{document}
		\paragraph{Zusammenfassend gilt:}
		
		\begin{enumerate}
			\item $U_{ind} \sim \dot B(t)$ mit $A(t)$ konstant
			\item $U_{ind} \sim \dot A(t)$ mit $B(t)$ konstant
		\end{enumerate}
		$\to$ Produktregel der Mathematik \\\\
		$U_{ind} \sim \dot B(t) \times A(t) + B(t) \times \dot A(t)$ \\\\
		$U_{ind} \sim \dot{\Bigl(B(t) \times A(t)\Bigl)}\ \ \ \Bigl[\sim (B(t) \times A(t))'\Bigl]$ \\\\
		$\phi(t) = A(t) \times B(t)$\ \ \dq Fluss\dq\ \ \ $\Bigl[\phi\Bigl] = 1Wb = 1 Weber = 1m^2 \times T$ \\
		\\ $U_{ind} \sim \dot \phi(t)$
		\\[3cm]
		Also wird in jeder Drahtwindung dieselbe Spannung \\induziert, sodass $U_{ind} \sim $ Anzahl der Windungen $n$ ist.
\end{document}

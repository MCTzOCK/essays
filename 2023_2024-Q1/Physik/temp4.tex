\documentclass[12pt, a4paper]{report}
\usepackage[T1]{fontenc}
\usepackage[utf8]{inputenc}
\usepackage{charter}
\usepackage{inputenc}
\usepackage[left=2cm,right=2cm,top=2cm,bottom=2cm]{geometry}
\usepackage{amsmath}

\begin{document}
	\noindent
	Die Spannung $U_C(t)$ und $U_L(t)$ sind in der Summe stets Null:
	\begin{align*}
		U_C(t) + U_L(t) = 0
	\end{align*}
	Siehe (1.9.1) und (1.9.2):
	\begin{align*}
		\frac{1}{C} \cdot Q(t) - L\cdot \dot I(t) &= 0 \\
		\frac{1}{C} \cdot Q(t) &= -L\cdot \dot I(t) \\
		-\frac{1}{LC} \cdot Q(t) &= \dot I(t)
	\end{align*}
	Mit der Tatsache, dass fließende Ladung den Strom bilden, ergibt sich mit $I(t) = \dot Q(t)$ und daraus $Q(t) = \int I(t) dt$:
	\begin{align*}
		-\frac{1}{LC} \cdot \int I(t) dt &= \dot I(t)
	\end{align*}
	Diese Gleichheit kann nur erfüllt werden, wenn die Stromstärkenfunktion $I(t)$ eine Sinusfunktion ist:
	\begin{align*}
		I(t) &= I_{max} \cdot \sin(\frac{2\pi}{T}\cdot t) \\
		\Rightarrow \int I(t) dt &= -I_{max} \cdot \cos(\frac{2\pi}{T}\cdot t) \\
		\Rightarrow \int I(t) dt &= -\frac{I_{max} \cdot T}{2\pi} \cdot \cos(\frac{2\pi}{T}\cdot t) \\
		\dot I(t) &= \frac{2\pi}{T} \cdot I_{max} \cdot \cos(\frac{2\pi}{T}\cdot t)
	\end{align*}
	Einsetzen:
	\begin{align*}
		-\frac{1}{LC} \cdot \Bigl(-\frac{I_{max}\cdot T}{2\pi} \cdot \cos(\frac{2\pi}{T}\cdot t)\Bigl) &= \frac{2\pi}{T}\cdot I_{max} \cdot \cos(\frac{2\pi}{T}\cdot t) \\
		\Leftrightarrow \frac{1}{LC} \cdot \frac{I_{max}\cdot T}{2\pi} \cdot \cos(\frac{2\pi}{T}\cdot t) &= \frac{2\pi}{T} \cdot I_{max} \cdot \cos(\frac{2\pi}{T}\cdot t) \\
		\Leftrightarrow \frac{1}{LC} \cdot \frac{T}{2\pi} &= \frac{2\pi}{T}\ \Bigl|\cdot T \\
		\Leftrightarrow \frac{1}{LC} \cdot \frac{T^2}{2\pi} &= 2\pi\\
		\Leftrightarrow T^2 &= 4\pi^2 \cdot LC\\
		\Leftrightarrow T &= 2\pi \cdot \sqrt{LC}\\
		&\to \text{Schwingungsformel von Thomson} \\
		\Bigl[T\Bigl] &= \sqrt{1H \cdot 1F} \\
		&= \sqrt{1\frac{Vs}{A} \cdot \frac{As}{V}} \\
		&= \sqrt{s^2} \\
		&\Rightarrow s
	\end{align*}
	In unserem Experiment hatten wir $C=\mu F$ und $L = 0,3 H$: \\
	\begin{align*}
		T &= 2\pi \cdot \sqrt{0,3H \cdot 1 \cdot 10^{-6}F} \\
		&= 2\pi \cdot \sqrt{3 \cdot 10^{-7} \cdot s^2} \\
		&\approx 0,003s
	\end{align*}
\end{document}
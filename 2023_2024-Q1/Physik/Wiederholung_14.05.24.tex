\documentclass[12pt,a4paper]{report}
\usepackage[T1]{fontenc}
\usepackage[utf8]{inputenc}
\usepackage{charter}
\usepackage{ngerman}
\usepackage[left=2cm,right=2cm,top=2cm,bottom=2cm]{geometry}

\renewcommand\thesection{\arabic{section}.}

\begin{document}
	\thispagestyle{empty}
	\section{Aufgabe}
	Das elektrische Feld beschreibt die Kraftwirkung auf eine geladene Teilchenladung in einem Raum.
	Ladungen erzeugen ein elektrisches Feld, das auf andere Ladungen wirkt.
	Die Richtung und Stärke der Kräfte zwischen geladenen Körpern werden durch das elektrische Feld bestimmt.
	Feldlinien zeigen die Richtung und Intensität des elektrischen Feldes an, indem sie vom positiven zum negativen Ladungsträger verlaufen.
	\section{Aufgabe}
	Das magnetische Feld entsteht um einen stromdurchflossenen Leiter und wird durch die Richtung des Stromflusses bestimmt.
	Feldlinien verlaufen in geschlossenen Schleifen um den Leiter herum.
	Die Lorentzkraft wirkt auf geladene Teilchen, die sich in einem magnetischen Feld bewegen, und führt zu einer Ablenkung ihrer Bahn.
	Ladungsträger im Magnetfeld erfahren eine Kraft senkrecht zu ihrer Bewegungsrichtung und zum magnetischen Feld, was zu einer Kreisbewegung oder Ablenkung führt.
	\section{Aufgabe}
	Ein Kondensator besteht aus zwei leitenden Platten, die nah beieinander platziert sind, aber nicht miteinander verbunden sind.
	Die Kapazität eines Kondensators hängt von der Größe der Platten und ihrem Abstand zueinander ab.
	Beim Aufladen eines Kondensators werden positive Ladungen auf einer Platte gesammelt, während gleichzeitig negative Ladungen auf der anderen Platte auftreten.
	Dies erzeugt ein elektrisches Feld zwischen den Platten.
	Beim Entladen des Kondensators fließt der gespeicherte Strom von einer Platte zur anderen, wodurch die gespeicherte Energie freigesetzt wird.
	\section{Aufgabe}
	Eine Spule besteht aus einem Draht, der um einen Kern gewickelt ist.
	Die Induktivität einer Spule hängt von der Anzahl der Windungen, dem Material des Kerns und der geometrischen Form ab.
	Bei einem Einschaltvorgang einer Spule wird durch den Stromfluss ein magnetisches Feld erzeugt, das sich durch die Windungen der Spule ausbreitet.
	Dieses magnetische Feld induziert eine Spannung entgegen der Änderung des Stroms.
	Dadurch baut sich der Strom in der Spule langsam auf.
	Beim Ausschaltvorgang kollabiert das magnetische Feld abrupt, was wiederum eine Spannung induziert, die dem Stromfluss entgegenwirkt.
	\section{Aufgabe}
	Das Induktionsgesetz besagt, dass die Änderung des magnetischen Flusses durch eine Leiterschleife eine elektrische Spannung in dieser Schleife induziert. Quantitativ formuliert lautet das Induktionsgesetz:
	\[ \varepsilon = - \frac{d\Phi}{dt} \]
	Hierbei steht \( \varepsilon \) für die induzierte Spannung, \( \frac{d\Phi}{dt} \) für die zeitliche Änderung des magnetischen Flusses \( \Phi \).
	\section{Aufgabe}
	Selbstinduktion tritt auf, wenn sich der durch eine Spule fließende Strom ändert und dadurch ein magnetisches Feld erzeugt wird.
	Dieses magnetische Feld induziert dann eine Spannung in der Spule selbst, die der Änderung des Stroms entgegenwirkt, gemäß dem Induktionsgesetz.
	Ein einfaches Experiment zum Nachweis der Selbstinduktion beim Ein- und Ausschalten einer Spule könnte wie folgt durchgeführt werden:\\
	\textbf{1. Ein- und Ausschalten}: Eine Spule wird mit einem Schalter und einer Stromquelle verbunden.
	Beim Einschalten der Stromquelle fließt ein Strom durch die Spule, was ein magnetisches Feld erzeugt.
	Beim Ausschalten der Stromquelle kollabiert das magnetische Feld abrupt, was eine Spannung induziert, die dem Stromfluss entgegenwirkt.\\
	\textbf{2. Beobachtung der Spannung}: Anschließend wird ein Voltmeter an die Spule angeschlossen, um die Spannung zu messen.
	Beim Einschalten der Stromquelle kann eine vorübergehende Spannungsspitze beobachtet werden, die durch die Selbstinduktion verursacht wird.
	Beim Ausschalten kann ebenfalls eine Spannung beobachtet werden, die der Änderung des Stromflusses entgegenwirkt.
	\thispagestyle{empty}
	\section{Aufgabe}
	Das Lenz'sche Gesetz besagt, dass die induzierte elektrische Spannung oder die induzierte elektromotorische Kraft in einem geschlossenen Stromkreis immer so gerichtet ist, dass sie einer Änderung des magnetischen Flusses entgegenwirkt, die sie verursacht hat.
	Mit anderen Worten, es besagt, dass eine induzierte Spannung oder eine induzierte Stromrichtung immer dazu führt, dass der Effekt, der sie erzeugt hat, abgeschwächt oder kompensiert wird.
	\section{Aufgabe}
	Die Kenngrößen einer mechanischen Schwingung sind:
	\begin{enumerate}
		\item \textbf{Auslenkung/Amplitude}: Die maximale Abweichung des schwingenden Objekts von seiner Ruhelage.
		\item \textbf{Periodendauer}: Die Zeit, die benötigt wird, um eine vollständige Schwingung durchzuführen.
		\item \textbf{Frequenz}: Die Anzahl der vollständigen Schwingungen pro Zeiteinheit, gemessen in Hertz.
	\end{enumerate}
	Die Kenngrößen einer elektrischen Schwingung sind:
	\begin{enumerate}
		\item \textbf{Auslenkung/Amplitude}: Die maximale Abweichung der elektrischen Spannung oder des Stroms von ihrem Mittelwert.
		\item \textbf{Periodendauer}: Die Zeit, die benötigt wird, um eine vollständige Schwingung eines elektrischen Signals durchzuführen, beispielsweise einer Wechselspannung oder eines Wechselstroms.
		\item \textbf{Frequenz}: Die Anzahl der vollständigen Schwingungen pro Zeiteinheit, gemessen in Hertz, bei elektrischen Schwingungen typischerweise die Frequenz des Wechselstroms oder der Wechselspannung.
	\end{enumerate}
	\thispagestyle{empty}
	\section{Resonanz}
	Resonanz tritt auf, wenn ein System dazu neigt, auf eine bestimmte Frequenz besonders stark zu reagieren.
	Wenn die Frequenz einer äußeren Kraft oder eines äußeren Signals mit der natürlichen Frequenz eines Systems übereinstimmt, verstärkt sich die Schwingung oder Reaktion des Systems.
	\section{Zusammenhang der Resonanz und der Videos}
	Durch die Resonanz wird eine Verstärkung der Schwingung der Feder möglich.
	\section{Resonanz im Alltag}
	Im Alltag werden Resonanzen gerne genommen, um bestimmte Effekte zu erzielen, wie in Musikinstrumenten oder Lautsprechern.
	In solchen Fällen wird Resonanz genutzt, um Schwingungen zu verstärken und eine gewünschte Klangqualität zu erreichen. 
	Auf der anderen Seite entstehen Resonanzen an Stellen, an denen man sie vermeiden möchte, wie zum Beispiel in Maschinen und Motoren.
	Resonanzen können hier zu unerwünschten Vibrationen führen, die die Leistung beeinträchtigen oder sogar Schäden verursachen können.
	Es ist wichtig, Resonanzfrequenzen zu kennen und Maßnahmen zu ergreifen, um unerwünschte Resonanzen zu verhindern.
	\thispagestyle{empty}
\end{document}
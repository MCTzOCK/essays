\documentclass[a4paper, 12pt]{report}
\usepackage{ngerman}
\usepackage[T1]{fontenc}
\usepackage[utf8]{inputenc}
\usepackage{custompkg}
\usepackage{charter}
\usepackage[left=2cm,right=2cm,top=2cm,bottom=2cm]{geometry}

\begin{document}
	\title{Induktionsversuche}
	\date{\today}
	\author{Julina \and Salome \and Alex \and Tanel \and Ben}
	\maketitle
	\bslinespacing{1.5}
	\bsremovechaptertitle
	\noindent
	\tableofcontents
	\chapter{Versuch 1: langes Rohr}
	\paragraph{Beobachtungen:}
	Bei dem Versuch ist aufgefallen, dass die Spannung Wellenförmig verläuft.
	Hierbei ist die Fläche unterhalb der $x$-Achse exakt so groß, wie die Fläche überhalb der $x$-Achse.
	Hierdurch sind die Integrale im Bereich der unterschiedlichen Spulen $0$.
	\paragraph{Änderungen bei Umdrehung des Magnets:}
	Wenn man den Magneten umdreht, so sind die Wellen exakt umgedreht.
	\paragraph{Erkennung der Lenz'schen Regel:}
	\paragraph{Berechnung der magnetischen Stärke:}
	\chapter{Versuch: Schwingspule}
	\paragraph{Beobachtungen:} Die Spannung verläuft wellenförmig sowohl über der $x$-Achse, als auch darunter.
	\paragraph{Änderungen beim Austausch:}
	Der Ausschlag ist bei der dünneren Spule deutlich höher, als bei der breiten Spule.
	\paragraph{Änderung des Startpunktes:}
	Desto höher der Startpunkt der Spule liegt, desto dünner sind die Ausschläge, desto niedriger der Startpunkt der Spule liegt, desto breiter sind die Ausschläge.
\end{document}
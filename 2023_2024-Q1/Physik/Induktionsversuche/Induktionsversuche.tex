\documentclass[a4paper, 12pt]{report}
\usepackage{ngerman}
\usepackage[T1]{fontenc}
\usepackage[utf8]{inputenc}
\usepackage{custompkg}
\usepackage{charter}
\usepackage[left=2cm,right=2cm,top=2cm,bottom=2cm]{geometry}
\usepackage{tabularx}
\usepackage{tikz}
\usepackage{pgfplots}

\begin{document}
	\title{Induktionsversuche}
	\date{\today}
	\author{Julina \and Salome \and Alex \and Tanel \and Ben}
	\maketitle
	\bslinespacing{1.5}
	\bsremovechaptertitle
	\noindent
	\tableofcontents
	\chapter{Versuch: langes Rohr}
	\paragraph{Beobachtungen:}
	Bei dem Versuch ist aufgefallen, dass die Spannung Wellenförmig verläuft.
	Hierbei ist die Fläche unterhalb der $x$-Achse exakt so groß, wie die Fläche überhalb der $x$-Achse.
	Hierdurch sind die Integrale im Bereich der unterschiedlichen Spulen $0$.
	\paragraph{Änderungen bei Umdrehung des Magnets:}
	Wenn man den Magneten umdreht, so sind die Wellen exakt umgedreht.
	\paragraph{Erkennung der Lenz'schen Regel:}
	\paragraph{Berechnung der magnetischen Stärke:}
	\chapter{Versuch: Schwingspule}
	\paragraph{Beobachtungen:} Die Spannung verläuft wellenförmig sowohl über der $x$-Achse, als auch darunter.
	\paragraph{Änderungen beim Austausch:}
	Der Ausschlag ist bei der dünneren Spule deutlich höher, als bei der breiten Spule.
	\paragraph{Änderung des Startpunktes:}
	Desto höher der Startpunkt der Spule liegt, desto dünner sind die Ausschläge, desto niedriger der Startpunkt der Spule liegt, desto breiter sind die Ausschläge.
	\chapter{Versuch: Doppelspule}
	\paragraph{Beobachtung:}
	Die Spannung schwankt zwischen einer negativen und einer positven Volt-Anzahl.
	Desto höher die Frequenz ist, desto schneller kleiner werden die Wellen.
	Die Veränderung der Stromstärke der äußeren Spule hat sich auf die Spannung der inneren Spule ausgewirkt.
	\paragraph{Variation der Probespule:}
	Bei der Spule, die die größere Fläche besitzt, ist die Voltanzahl größer.
	\chapter{Versuch: Waltenhofen-Pendel}
	\paragraph{Beobachtungen:}
	Je höher die Spannung ist, desto schneller wird das Pendel ausgebremst.
	\\[0.6cm]
	\begin{tabularx}{\textwidth}{|X|X|X|}
		\hline
		\textbf{Pendel} & Spannung & \textbf{Zeit bis ausgebremst} \\
		\hline
		Kammförmig & 0 V & $18.71s$ \\
		\hline
		Kammförmig & 5 V & $11.56s$ \\
		\hline
		Kammförmig & 10 V & $7.16s$ \\
		\hline
		Kammförmig & 15 V & $4.93s$ \\
		\hline
		Kreis & 0 V & $15.82s$ \\
		\hline
		Kreis & 5 V & $5.13s$ \\
		\hline
		Kreis & 10 V & $1.68s$ \\
		\hline
		Kreis & 15 V & $1.35s$ \\
		\hline
		Platte & 0 V & $20.42$ \\
		\hline
		Platte & 5 V & $3.63s$ \\
		\hline
		Platte & 10 V & $1.33s$ \\
		\hline
		Platte & 15 V & $0.98s$ \\
		\hline
	\end{tabularx}
	\chapter{Theorie: 2. Aufgabe}
	Windungen: $200$ \\
	Fläche: $0.1m^2$ \\
	Zeit: $2s$ \\
	B: $1.2\ T$ \\
	$
	U_{ind} = -n \times \dot \phi (t) \\
	U_{ind} = -200 \times A(t) \times \dot B(t) \\
	U_{ind} = -200 \times 0.1m^2 \times \dot B(t) \\
	U_{ind} = -20m^2 \times \dot B(t) \\
	U_{ind} = -20m^2 \times 0.6\ \frac{T}{s} \\
	U_{ind} = -12\ \frac{m^2 \times T}{s} \\
	U_{ind} = -12\ \frac{Wb}{s} \\
	U_{ind} = -12\ \frac{V \times s}{s} \\
	U_{ind} = -12\ V \\
	$
	Die Induktionsspannung nach 2 Sekunden beträgt -12 Volt \\[1cm]
	Windungen: $200$ \\
	Fläche: $0.1m^2$ \\
	Zeit: $4s$ \\
	B: $1.2T$ \\
	$
	U_{ind} = -n \times \dot \phi (t) \\
	U_{ind} = -200 \times A(t) \times \dot B(t) \\
	U_{ind} = -200 \times 0.1m^2 \times (-0.3 \frac{T}{s}) \\
	U_{ind} = -20m^2 \times (-0.3 \frac{T}{s}) \\
	U_{ind} = 6 \frac{m^2 \times T}{s} \\
	U_{ind} = 6 \frac{Wb}{s} \\
	U_{ind} = 6 \frac{V \times s}{s} \\
	U_{ind} = 6 V \\
	$
	Die Induktionsspannung beim Abschalten beträgt nach 4 Sekunden 6 V
\end{document}bb
\documentclass[12pt,a3paper,landscape]{report}
\usepackage[T1]{fontenc}
\usepackage[utf8]{inputenc}
\usepackage{charter}
\usepackage{ngerman}
\usepackage[left=1cm,right=1cm,top=1cm,bottom=1cm]{geometry}
\usepackage{tabularx}

\begin{document}
	\thispagestyle{empty}
	\noindent
	\Huge
	\textbf{Barock} \\[0.75cm]
	\Large
	\begin{tabularx}{\textwidth}{|X|X|X|X|}
		\hline
		\textbf{Bevorzugte Themen} & \textbf{Bevorzugte Gattungen} & \textbf{Sprache} &\textbf{Historische Ereignisse} \\
		\hline
		\begin{itemize}
			\item Vergänglichkeit aufgrund der gesellschaftlichen Umstände
			\item Glaube als zentraler Bestandteil des Lebens
			\item Gegensätzlichkeit: Todesangst und Lebenslust
		\end{itemize} &
		\begin{itemize}
			\item Barockschriftsteller gestalteten das Spektrum der bis heute wichtigen Gattungen und Formen
			\item Sonnet (Gedicht aus 14 metrisch gegliederten Verszeilen, die in unterschiedlich lange Strophen geteilt sein können.)
			\item Epigramm
			\item Emblem
			\item Jesuitendrama
			\item Schäferliteratur
			\item Kirchenlieder
		\end{itemize} &
		\begin{itemize}
			\item Weiterentwicklung der neuhochdeutschen Literatursprache
			\item Vorliebe der Dichter für Metaphern, Allegorien und Embleme
			\item barocke Schwulst
		\end{itemize} &
		\begin{itemize}
			\item Der 30-jährige Krieg
			\item Entwicklung zum Absolutismus
			\item Glaubensspaltung in der Bevölkerung
			\item Aufschwung der Mathematik und der Naturwissenschaften
			\item Hunger, Krankheiten und Hexenverfolgungen verursachten Leid
		\end{itemize}
		\\
		\hline
	\end{tabularx}
\end{document}
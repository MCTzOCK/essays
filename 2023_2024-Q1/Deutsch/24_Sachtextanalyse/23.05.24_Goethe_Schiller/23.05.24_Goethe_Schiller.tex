\documentclass[12pt,a4paper]{report}
\usepackage[T1]{fontenc}
\usepackage[utf8]{inputenc}
\usepackage{charter}
\usepackage{ngerman}
\usepackage[left=2cm,right=2cm,top=2cm,bottom=2cm]{geometry}

\renewcommand\thesection{\arabic{section}.} 

\begin{document}
	\setcounter{section}{1}
	\section{Aufgabe}
	
	\paragraph{Schiller:} 
	Lieber Goethe, die Figur der Iphigenie in deinem neuen Entwurf fasziniert mich besonders.
	Sie scheint mir ein hervorragendes Beispiel für eine \dq schöne Seele\dq\ zu sein, wie ich sie in meinem Werk \dq Über Armut und Würde\dq\ definiert habe.
	Iphigenies innerer Konflikt zwischen Pflicht und Gefühl spiegeln diesen Idealzustand wieder.

	\paragraph{Goethe:}
	Da stimme ich dir zu.
	Iphigenie verkörpert in der Tat das Ideal der \dq schönen Seele\dq.
	Ihre Handlungen sind nicht nur moralisch, sondern ihr gesamter Charakter strahlt diese sittlich-moralische Vollkommenheit aus, die du beschrieben hast.
	Sie agiert aus einem inneren Instinkt heraus, der sie dazu bringt, selbst in den schwierigsten Situationen das Richtige zu tun.
	
	\paragraph{Schiller:}
	Genau, ihr Wille entspricht dem Affekt und ihre Entscheidungen sind im Einklang mit ihren tief verwurzelten moralischen Empfinden.
	Dadurch wird sie eine Figur, die ganzheitlich als Ausdruck für einen moralischen Charakter steht.
	
	\paragraph{Goethe:}
	Korrekt, Iphigenies Entscheidung, Thaos die Wahrheit über ihre Herkunft zu sagen und damit ihr eigenes Leben zu riskieren, zeigt ihre innere Reinheit und ihr Vertrauen darauf, dass das Göttliche in ihr Handeln eingreift und sie führt.
	Es ist dieser Glaube an eine höhere moralische Instanz, der sie letztlich zu ihrem heldenhaften Opfer bewegt.
	
	\paragraph{Schiller:}
	Und doch wirkt dieses Opfer nicht als erzwungen oder schmerzhaft, sondern fast mühelos, als wäre es eine natürliche Folge ihres Instikts.
	Diese Leichtigkeit, mit der sie ihre schwersten Pflichten erfüllt, unterstreicht ihre Rolle als \dq schöne Seele\dq.
\end{document}
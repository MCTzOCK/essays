\documentclass[12pt,a4paper]{report}
\usepackage[T1]{fontenc}
\usepackage[utf8]{inputenc}
\usepackage{charter}
\usepackage{ngerman}
\usepackage[left=2cm,right=2cm,top=2cm,bottom=2cm]{geometry}
\usepackage{amsmath}

\renewcommand\thesection{\arabic{section}.} 

\begin{document}
	\section{Ursachen der Heilung des Orest}
	\paragraph{Burger}:
	\begin{itemize}
		\item Heilung durch Mittleid
		\item Heilung durch Menschlichkeit
	\end{itemize}
	\begin{align*}
		\to& \text{Wiedererkennen der Schwestern, Hoffnung, Glaube an Liebe / Menschlichkeit, Versöhnung}
	\end{align*}
	\paragraph{Hodler}:
	\begin{itemize}
		\item Heilung durch die Götter
		\item Heilung durch transzendente Erlösung
	\end{itemize}
	\begin{align*}
		\to& \text{Götter legen Fluch auf ihn und erlösen ihn; Ansprache an Götter vor  Gebet Iphigenies}
	\end{align*}
	\paragraph{Eissle}:
	\begin{itemize}
		\item Selbstheilung
	\end{itemize}
	\begin{align*}
		\to& \text{innere Wandlung, eigene Absolution; Selbstheilung nicht anerkennen} \to \text{Hilfe Iphigenies}
	\end{align*}
	\paragraph{Staiger}:
	\begin{itemize}
		\item Heilschlaf
		\item \dq Gnade des Vergessens\dq
	\end{itemize}
	\begin{align*}
		\to& \text{\ \dq aus Betäubung erwachend\dq}\to \text{Schlaf} \\
		\to&\  \text{wird wacher und klarer und erscheint geheilt.} \\
		\\
		\Rightarrow&\ \text{Alle Deutungsversuche sind am Text belegbar.} \\
		&\ \text{Daher ist eine prozesshafte Heilung auf verschiedenen} \\
		&\ \text{Ebenen denkbar. Deutung zwei (Götter) widerspricht} \\
		&\ \text{der weiteren Deutung / Entwicklung Iphigenie.}
	\end{align*}
\end{document}
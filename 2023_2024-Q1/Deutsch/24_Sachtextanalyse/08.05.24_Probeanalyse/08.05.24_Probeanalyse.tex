\documentclass[12pt,a4paper]{report}
\usepackage[T1]{fontenc}
\usepackage[utf8]{inputenc}
\usepackage{charter}
\usepackage{ngerman}
\usepackage[left=2cm,right=2cm,top=2cm,bottom=2cm]{geometry}
\renewcommand\thesection{\arabic{section}.}

\begin{document}
	\section{Sachtextanalyse \dq Erlösung durch reine Menschlichkeit\dq}
	Der vorliegende literaturwissenschaftliche Sachtext \dq Erlösung durch reine Menschlichkeit\dq, welcher 1959 von Heinz-Otto Burger veröffentlich wurde, bezieht sich \dq Iphigenie auf Tauris\dq\ und thematisiert wie Goethe Iphigenies reine Menschlichkeit darstellt und wie er dadurch vom mythologischen-theologischen zum psychologischen Aspekt übergeht.
	Der Text richtet sich an eine Leserschaft, die literaturwissenschaftlichen6 interessiert ist. \\
	Burger beginnt mit einer kurzen inhaltlichen Beschreibung der Geschehnisse aus dem original Text von Goethe.
	Anschließend geht Burger auf die Existenz der biblischen Thelogie ein.
	Im letzten Abschnitt thematisiert Burger die Veränderung von Goethes Dichtung innerhalb von \dq Iphigenie auf Tauris\dq.
	\\
	Der Text verfügt formal über keine besonderen Auffälligkeiten.
	Er ist mit einer Standardschriftart und -größe geschrieben.
	Ebenfalls sind keinerlei Illustrationen in Form von Bildern oder Diagrammen vorhanden.
	Burger hat seinen Text mit einem gewissen wertenden Ton verfasst, was vor allem durch die Verwendung der Phrasen \dq ohne Zweifel\dq\ (Z. 10), \dq aber auch in Wirklichkeit\dq\ (Z. 10), \dq furchtbare[r] Schuld\dq\ (Z. 11) oder \dq sinnvolles Glied\dq\ (Z. 18) deutlich wird. \\
	Seine Wortwahl ist der Textsorte entsprechend sehr fachbezogen gehalten.
	So verwendet er beispielsweise Ausdrücke, wie \dq Evokation\dq\ (Z. 17) und \dq Nexus\dq\ (Z. 18), welche für seine Fachkenntnisse im Bereich der Literaturwissenschaften sprechen.
	Burger verwendet in seinem Text viele verschiedene Adjektive, wie zum Beispiel \dq furchtbar\dq\ (Z. 11) und \dq tief\dq\ (Z. 13).
	Auffällig oft ist allerdings die Verwendung des Adjektives \dq rein\dq (vgl. Z. 6, 9, 17).
	\\
	Der Autor wiederholt einige Satzkonstrukte mehrfach in seinem Text.
	Ein Beispiel hierfür ist der Satz \dq So hat Goethe ohne Zweifel gemeint\dq\ (Z. 10), wessen Struktur direkt im nächsten Satz mit \dq So ist es aber auch Wirklichkeit im Drama\dq\ (Z. 10) wieder aufgegriffen wird.
\end{document}
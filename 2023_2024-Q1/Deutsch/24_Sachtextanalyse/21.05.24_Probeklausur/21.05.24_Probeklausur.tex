\documentclass[12pt,a4paper]{report}
\usepackage[T1]{fontenc}
\usepackage[utf8]{inputenc}
\usepackage{charter}
\usepackage{ngerman}
\usepackage[left=2cm,right=2cm,top=2cm,bottom=2cm]{geometry}

\renewcommand\thesection{\arabic{section}.} 

\begin{document}
	\large
	\section{Aufgabe 1}
	Der vorliegende Text von Johann Gottfried Herder aus seinen \dq Ideen zur Philosophie der Geschichte der Menschheit\dq\ beschreibt die Position des Autors hinsichtlich des menschlichen Strebens und der Humanität. 	Herder argumentiert, dass der Zweck einer Sache in ihr selbst liegen muss.
	Er verweist auf die Absurdität eines Schicksals, das die Menschheit dazu verdammt, einem unerreichbaren Ziel nachzujagen, was uns zu bloßen Maschinen machen würde, die einem grausamen und unvollkommenen Wesen dienen.
	\\
	Herder lehnt die Vorstellung ab, dass die Menschheit einem externen Punkt der Vollkommenheit nachstrebt, der nie erreicht werden kann. Stattdessen betont er, dass Humanität das höchste Ziel des menschlichen Lebens ist. 	Diese Humanität ist in der Natur des Menschen verankert, und alle menschlichen Anstrengungen sollten darauf abzielen, diese zu fördern.
	Herder beschreibt, wie unsere Sinne, Vernunft, Freiheit, Gesundheit, Sprache, Kunst und Religion uns gegeben sind, um Humanität zu entwickeln.
	Er erläutert, dass alle sozialen Strukturen und Lebensweisen der Menschen darauf ausgerichtet sind, Humanität zu fördern, sei es durch Erziehung, Gesetze, Regierungsformen oder kulturelle Gebräuche.
	\\
	Zusammenfassend ist Herders Kernaussage, dass alle positiven Errungenschaften der Menschheitsgeschichte der Humanität dienen, während alles Negative gegen die Humanität gerichtet ist.
	Der Mensch kann sich keinen anderen Zweck seiner Existenz vorstellen als die Förderung der in ihm angelegten Humanität.

	\section{Aufgabe 2}

	Um zu erörtern, inwieweit Goethes Drama \dq Iphigenie auf Tauris\dq\ Humanitätsverständnis Herders entspricht, kann man mehrere Aspekte des Dramas heranziehen. In \dq Iphigenie auf Tauris\dq\ die Titelfigur Iphigenie für Ideale wie Menschlichkeit, Vernunft und moralische Integrität, die zentrale Elemente von Herders Humanitätskonzept sind.
	\\
	Ein Beispiel ist Iphigenies Weigerung, sich auf Täuschung und Gewalt einzulassen, um ihre Flucht von Tauris zu ermöglichen.
	Stattdessen setzt sie auf Ehrlichkeit und menschlichen Dialog, was schließlich zu einem friedlichen Ausgang führt.
	Diese Haltung spiegelt Herders Idee wider, dass der Mensch seine Kräfte nutzen und einen freien, schönen Genuss des Lebens anstreben sollte, ohne anderen zu schaden.
	\\
	Ein weiteres Beispiel ist Iphigenies Einfluss auf König Thoas, den sie durch ihre humanen Werte und ihre moralische Überzeugungskraft dazu bringt, seine anfängliche Härte aufzugeben und menschlicher zu handeln.
	Dies zeigt, wie Humanität, wie Herder sie versteht, auch gesellschaftliche und politische Strukturen positiv beeinflussen kann.
	\\
	Insgesamt kann Goethes "Iphigenie auf Tauris" als ein Werk gesehen werden, das die Prinzipien von Herders Humanitätsverständnis verkörpert.
	Es illustriert, wie die Förderung von Humanität durch individuelle Integrität und moralisches Handeln zu einer besseren und gerechteren Welt führen kann.

\end{document}
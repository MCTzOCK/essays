\documentclass[12pt,a4paper]{report}
\usepackage[T1]{fontenc}
\usepackage[utf8]{inputenc}
\usepackage{charter}
\usepackage{ngerman}
\usepackage[left=2cm,right=2cm,top=2cm,bottom=2cm]{geometry}
\usepackage{tabularx}

\renewcommand\thesection{\arabic{section}.} 

\newcommand{\mod}[0]{\textbf{Moderator}:&}
\newcommand{\ben}[0]{\textbf{Ben}:&}
\newcommand{\alex}[0]{\textbf{Alex}:&}
\newcommand{\gregor}[0]{\textbf{Gregor}:&}
\newcommand{\greta}[0]{\textbf{Greta}:&}
\newcommand{\inga}[0]{\textbf{Inga}:&}
\newcolumntype{s}{>{\hsize=.25\hsize}X}

\begin{document}
	\section{Podcast}
	\begin{tabularx}{\textwidth}{sX}
		\mod Herzlich Willkommen zu unserem Podcast über Sprache. Lasst uns zunächst mit der Entwicklung der Sprache starten! \\\\
		\gregor Die Sprache entwickelt sich natürlich ständig. Zum Beispiel mutiert das Deutsche zunehmen zum Denglischen. Es wird von Anglizismen überschwemmt und die Grammatik wird von Lehnwendungen unterwandert. Allerdings werden viele Wortimporte rasch wieder aussortiert. Die verbliebenen Fremdwörter werden oft bis zur Unkenntlichkeit assimiliert. So zum Beispiel Keks oder Gulli, welchen man ihre Herkunft ebensowenig ansieht, wie Küche oder Esel! Schon vor 1600-2000 Jahren wurden Worte wie diese aus dem Latein importiert. Andersrum verschwinden aber auch Worte, wie \dq Wählscheibe\dq\ aus dem Alltagswortschatz, da sie einfach nicht mehr gebraucht werden. Für solche Worte gibt es die rote Liste der bedrohten Wörter. Nicht nur andere Sprachen, sondern auch neue Erfindungen und Phänomene bringen neue Begriffe in die Sprache, wie zum Beispiel Heizpils oder Fanmeile. Die Sprache wird durch den Wunsch in die Veränderung getrieben, verstanden zu werden und dabei auch originell und zeitgemäß zu werden.\\\\
		\mod Vielen Dank für diese aufschlussreichen Informationen! Für das folgende Theme
	\end{tabularx}
\end{document}
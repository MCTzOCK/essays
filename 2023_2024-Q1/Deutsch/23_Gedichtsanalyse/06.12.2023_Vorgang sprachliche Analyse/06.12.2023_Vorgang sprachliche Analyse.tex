\documentclass[12pt,a4paper]{report}

\usepackage{ngerman}
\usepackage{bookman}
\usepackage[utf8]{inputenc}
\usepackage[T1]{fontenc}
\usepackage[left=2cm,right=2cm,top=2cm,bottom=2cm]{geometry}
\usepackage{tabularx}
\usepackage{custompkg}

\begin{document}
	\bslinespacing{1.25}
	\clearpage
	\thispagestyle{empty}

	\Large \textbf{Wie geht man bei der sprachlichen Analyse vor?}
	
	\vspace{1.5cm}
	\large
	\paragraph{Allgemein}
	\begin{enumerate}
		\item Benenne die sprachlichen Mittel (z.B. Personifikation)
		\item Beschreibe die Ausgestaltung der sprachlichen Mittel (z.B. im Zitationskontext)
		\item Deute die sprachlichen Mittel (welche Wirkung wird mit dem sprachlichen Mittel erzielt? Welche Stimmung wird erzeugt?)
	\end{enumerate}
	
	\paragraph{Vorgehen}
	\begin{enumerate}
		\item chronologisch (von Vers 1 bis zum Ende)
		\item thematisch (Inhaltsorientiert, anhand sprachlicher Mittel sortiert)
	\end{enumerate}
	\vspace{1cm}
	\textbf{Wichtig}: Es sollte immer eine Verknüpfung von Sprache und Inhalt erfolgen.
	
	\newpage
	\clearpage
	\thispagestyle{empty}
	\Large\textbf{Sprachliche Mittel}
	\vspace{1cm}
	\large	

	\begin{tabularx}{0.9\textwidth}{|X|X|}
		\hline
		\textbf{Sprachliches Mittel} & \textbf{Funktion} \\		\hline
		Personifikation & etwas wird vermenschlicht \\
		\hline
		Vergleich & Vergleicht eine Sache mit einer anderen \\
		\hline
		Anapher/Epipher & Wiederholung von Worten (Vers- oder Satzanfang/ende) \\
		\hline
		Alliteration & Lautwiederholung, die direkt aufeinander folgt \\
		\hline
		Metapher & Etwas wird mit etwas anderem beschrieben\\
		\hline
		rhet. Frage & Frage, die keine Antwort bedarf \\
		\hline
		Neologismus & Wortneuschöpfung \\
		\hline
		Hyperbel & Übertreibung \\
		\hline
		Klimax & Steigerung (inhaltlich) \\
		\hline
		Parallelismus & Gleicher Aufbau von Sätzen oder Versen (gleiche Struktur) \\
		\hline
		Antithese & Gegenüberstellung \\
		\hline
		Symbol & bekannte bildliche Beschreibung (z.B. Rose für Liebe)\\
		\hline
		Paradoxon & Etwas wird paradox dargestellt (Adjektiv passt nicht zum Nomen) \\
		\hline
		Pleonasmus & Verdeutlichung (durch Vorstellung eines Adjektivs) \\
		\hline
		Enjambement & Zeilensprung \\
		\hline
		Ironie & ironische Darstellung einer Sache\\
		\hline
	\end{tabularx}

\end{document}
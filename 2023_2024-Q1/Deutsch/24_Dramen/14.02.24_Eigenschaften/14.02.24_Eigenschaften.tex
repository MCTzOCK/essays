\documentclass[12pt, a4paper]{report}
\usepackage[left=2cm,top=2cm,right=2cm,bottom=2cm,landscape]{geometry}
\usepackage[utf8]{inputenc}
\usepackage[T1]{fontenc}
\usepackage{charter}
\usepackage{ngerman}
\usepackage{tabularx}
\usepackage{custompkg}

\begin{document}
	\newcommand{\elabo}[0]{\texttt{elaboriert}}
	\newcommand{\restr}[0]{\texttt{restringiert}}
	
	\bslinespacing{1.25}
	\noindent
	\Large \textbf{Eigenschaften der Charaktere von \dq Woyzeck\dq} \\[1cm]
	\normalsize
	\begin{tabularx}{\textwidth}{|X|X|X|X|X|X|}
		\hline
		 & \textbf{Marie} & \textbf{Hauptmann} & \textbf{Doktor} & \textbf{Tambourmajor} & \textbf{Andres} \\
		\hline
		\textbf{Alter} & 20-25 Jahre & 50 Jahre & 55 Jahre & 30 Jahre & 25 Jahre \\
		\hline
		\textbf{Beruf} & Hausfrau & Hauptmann & Doktor & Tambourmajor & Soldat \\
		\hline
		\textbf{Gesellschaftliche Stellung} & gleichgestellt mit Woyzeck, zweitunterste Schicht; unter Hauptmann, Tambourmajor und Doktor & oberste Schicht & oberste Schicht & zweitoberste Schicht, unter Hauptmann und Doktor & gleichgestellt mit Woyzeck, zweitunterste Schicht, unter Hauptmann, Tambourmajor und Doktor \\
		\hline
		\textbf{Auftreten / Körperhaltung} &Selbstbewusst& Dominant&interessiert&selbstverliebt; egozentrisch & unscheinbar \\
		\hline
		\textbf{Redeweise} &einfach& intellektuell; gehoben&intellektuell; gehoben&einfach; maskulin; Schimpfwörter& einfach; stottert \\
		\hline
		\textbf{Beziehung zu Woyzeck} &hintergeht ihn, seht sich allerdings nach Sicherheit & überlegen & experimentiert mit Woyzeck &ist überlegen; möchte Marie \dq ausspannen\dq & befreundet mit Woyzeck \\
		\hline
		
	\end{tabularx}
\end{document}
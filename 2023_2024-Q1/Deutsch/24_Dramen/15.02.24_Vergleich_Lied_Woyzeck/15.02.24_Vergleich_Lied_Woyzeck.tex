\documentclass[a4paper, 12pt]{report}
\usepackage[utf8]{inputenc}
\usepackage[T1]{fontenc}
\usepackage{ngerman}
\usepackage{charter}
\usepackage[left=2cm,right=2cm,top=2cm,bottom=2cm]{geometry}

\begin{document}
	\thispagestyle{empty}
	\noindent
	\Large Aufgabe 1: Bild von Marie und ihrem \dq Liebsten\dq
	\\[0.5cm]
	\large
	Das Lied vermittelt das Bild der Hinrichtung von Marie, deren Grund nicht bekannt ist. (vgl. z.B. V. 11-12) Marie wurde von dem \dq Liebsten\dq, welcher als böse dargestellt wird (vgl. V. 17-20), zerschlagen. (vgl. V. 11)
	Trotzdem wird in Frage gestellt, ob Marie die Ursache für das Dilemma ist. (vgl. V. 20) Marie wird als sanft und unschuldig.
	\\[1cm]
	\Large Aufgabe 2: Vergleich des Liedes und der Szene \dq Abend\dq
	\\[0.5cm]
	\large
	Die beiden Texte haben gemeinsam, dass der \dq Liebste\dq\ (Woyzeck) als Monster dargestellt wird, welches schreckliche Dinge vollbringt.
	Ein Unterschied zwischen dem Lied und der Szene ist, dass am Ende des Liedes beschrieben wird, wie Marie friedlich in ihrem Grab ruht, während die Szene nach dem Mord durch Woyzeck endet.
	Der größte Unterschied ist allerdings, dass der Erzähler des Liedes mit der bereits gestorbenen Marie spricht, aber keine Antwort erhält.
	In der Szene hingegen gibt es einen Dialog zwischen Woyzeck und der noch lebenden Marie, die allerdings am Ende der Szene von Woyzeck hingerichtet wird.
	Beide Texte thematisieren zu Anfang einen Spaziergang, wobei das Lied das Ziel genauer beschreibt.
	Dieser Spaziergang endet im Lied in einem schwarzen Hain.
	Die Szene gibt nur an, dass der Spaziergang aus der Stadt herausführt, allerdings nicht wo er schließlich endet.
	Ein weiterer Unterschied ist, dass im Lied Marie erschlagen und in der Szene erstochen wird.
\end{document}
\documentclass[12pt,a4paper]{report}
\usepackage[T1]{fontenc}
\usepackage[utf8]{inputenc}
\usepackage{charter}
\usepackage{ngerman} \usepackage[left=2cm,right=2cm,top=2cm,bottom=2cm]{geometry}

\newcommand{\psx}[0]{$\to$\ }

\begin{document}
	\noindent
	\Large
	\textbf{Analyse Wodzeck}
	\hrule
	\vspace{0.2cm}
	\large
	\noindent
	\textbf{1. Einleitungssatz:} \\
	Thema: Psychische und physische Verfassung Wodzecks wird in einem Gespräch mit dem Doktor thematisiert.
	\\
	\textbf{2. Setting:} \\
	Entspricht inhaltlich in groben Zügen der 18. Szene des Dramas Woyzeck, ist jedoch ein Auszug aus dem Film Wodzeck.
	\\
	\textbf{3. Inhaltsangabe:}
	\begin{itemize}
		\item Zu Beginn Austausch über gesundheitliche Beschwerden
		\item Einfache körperliche Untersuchung und Erschließen weiterer Problemfelder über Fragen
		\item Weitere Behandlung wird dargestellt, Lösung der gesundheitlichen Probleme über Medikamente und gegebenenfalls weitere Untersuchungen
	\end{itemize}
	\textbf{4. Analyse:}
	\begin{enumerate}
		\item \textbf{Struktur:} \\
		\psx klare Struktur \\
		\psx insgesamt ausgeglichene Redeanteile, jedoch stark wechselnd. \\
		\psx es fehlt jede Regieanweisung
		\item \textbf{Charakterisierung:} \\
		\psx Wodzeck verwirrt, verzweifelt, hilflos, psychisch vermutlich krank \\
		\psx Doktor eher oberflächlich, wenig interessiert, arbeitet die \dq Punkte\dq\ einfach ab, distanziert.
		\item \textbf{Sprache:} \\
		\psx Parataxen \\
		\psx Alltagssprache, formelles Arzt-Patienten-Gespräch \\
		\psx Pausen \\
		\psx Fragen
	\end{enumerate}
	\textbf{5. Schluss:} \\
	Zielsetzung: \dq Wahrnehmung Wodzecks wird infrage gestellt, psychische Gesundheit thematisiert\dq
\end{document}
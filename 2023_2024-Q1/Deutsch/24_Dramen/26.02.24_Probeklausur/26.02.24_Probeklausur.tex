\documentclass[a4paper,12pt]{report}
\usepackage[T1]{fontenc}
\usepackage[utf8]{inputenc}
\usepackage{ngerman}
\usepackage{charter}
\usepackage[left=2cm,right=2cm,top=2cm,bottom=2cm]{geometry}

\begin{document}
	\noindent
	\Large
	\textbf{Probeklausur}
	\\
	Aufgabe 1: Analyse
	\hrule
	\vspace{0.5cm}
	\noindent
	\large
	Die Szene \dq Wodzeck\dq, welche in der Fassung von Oliver Herbrich im Jahre 1984 veröffentlicht wurde und aus dem Drama \dq Woyzeck\dq\ von Georg Büchner stammt, welches 1837 veröffentlicht wurde, thematisiert Woyzecks Besuch beim Doktor.
	In der Originalfassung von Woyzeck trifft Woyzeck Marie zuvor auf der Straße und stammelt unverständlich vor sich herum.
	Nach der vorliegenden Szene findet Woyzeck heraus, dass er von Marie betrogen wird und fühlt sich gedemütigt.
	Die Szene zeigt Woyzecks Gespräch mit dem Doktor.
	Woyzeck klagt über seine Beschwerden bezüglich den Stimmen die er hört, woraufhin der Doktor ihm ein neues Rezept für seine Tabletten verschreibt.
	Ich lege meiner Analyse die Hypothese zu Grunde, dass der Doktor Woyzeck nur für seine Experimente an ihm ausnutzt und ihm nichts an Woyzecks eigentlichem Wohlbefinden liegt.
	\\
	Die Szene zeigt die ungleiche Beziehung zwischen dem Arbeiter Wodzeck und dem Werksarzt, der ihn als Versuchskaninchen für seine Medikamente benutzt. Der argumentative Aufbau der Szene besteht aus vier Sinnabschnitten, die jeweils eine Frage des Arztes und eine Antwort von Wodzeck enthalten.
	Die Fragen des Arztes sind jedoch nicht an Wodzecks individuelle Situation angepasst, sondern dienen nur dazu, seine Symptome zu überprüfen und ihm weitere Anweisungen zu geben.
	Die Antworten von Wodzeck sind meist kurz und unterwürfig, außer wenn er versucht, dem Arzt von seinen Wahnvorstellungen zu erzählen.
	Der Arzt zeigt jedoch kein Interesse an Wodzecks psychischem Zustand und unterbricht ihn immer wieder.
	Die Figurenkonzeptionen und -charakterisierungen sind durch die Gesprächsführung und die sprachliche Gestaltung deutlich gemacht.
	Der Arzt ist eine autoritäre und kalte Figur, die Wodzeck nicht als Menschen, sondern als Objekt seiner Forschung behandelt.
	Er verwendet eine distanzierte und fachliche Sprache, die seine Überlegenheit und Macht ausdrückt. Er nennt Wodzeck nicht beim Vornamen, sondern nur bei seinem Nachnamen oder mit dem unpersönlichen “uns”.
	Er stellt keine persönlichen Fragen, sondern nur solche, die seine Diagnose und Therapie betreffen.
	Er gibt Wodzeck keine Wahl, sondern nur Befehle. Er zeigt keine Empathie oder Verständnis für Wodzecks Leiden, sondern reduziert ihn auf seine körperlichen Funktionen.
	Er beendet das Gespräch abrupt und schiebt Wodzeck zur Tür hinaus, ohne ihm eine Perspektive oder Hoffnung zu geben.
	Wodzeck ist eine tragische und passive Figur, die unter der Ausbeutung und Entfremdung der kapitalistischen Gesellschaft leidet.
	Er verwendet eine einfache und umgangssprachliche Sprache, die seine soziale und bildungsmäßige Benachteiligung zeigt.
	Er nennt den Arzt immer mit “Herr Doktor”, was seine Unterordnung und Respekt ausdrückt.
	Er antwortet meist mit “Jawohl, Herr Doktor”, was seine Gehorsamkeit und Anpassung zeigt.
	Er versucht, dem Arzt von seinen inneren Konflikten und Ängsten zu erzählen, aber er findet keine passenden Worte, um sie auszudrücken.
	Er verwendet vage und unbestimmte Formulierungen wie “so’n komisches Gefühl” oder “wie wenn etwas ist und doch nicht ist”.
	Er spricht von “Stimmen”, die er hört, aber nicht versteht.
	Er fragt den Arzt, ob er sie auch gehört hat, aber er bekommt keine Antwort.
	Er ist verzweifelt und hilflos, aber er kann sich nicht gegen den Arzt wehren oder ihm widersprechen.
	Die sprachliche Gestaltung der Szene ist durch verschiedene Mittel gekennzeichnet, die die Spannung und den Kontrast zwischen den beiden Figuren erhöhen.
	Die Wortwahl ist oft ironisch oder paradox, wie zum Beispiel, wenn der Arzt Wodzeck fragt, wie es seiner Familie geht, obwohl er weiß, dass er keine hat.
	Die Syntax ist oft elliptisch oder unvollständig, wie zum Beispiel, wenn Wodzeck sagt: “Die Stimmen, Herr Doktor, da steckt’s.”
	Die stilistischen Figuren sind oft rhetorisch oder symbolisch, wie zum Beispiel, wenn der Arzt Wodzeck einen Zettel reicht, der seine Abhängigkeit und Fremdbestimmung bedeutet.
	Die Szene ist auch durch einen Wechsel von kurzen und langen Sätzen, von Fragen und Antworten, von direkter und indirekter Rede, von sachlicher und emotionaler Sprache strukturiert.
	\\
	Zusammenfassend bestätigt sich meine zu Anfang aufgestellte Deutungshypothese, da der Doktor Woyzeck tatsächlich nur für seine Experimente ausnutzt und sich ebenfalls nicht um Woyzecks Wohlbefinden kümmert, was durch die vielen Unterbrechungen und oberflächlichen Fragen deutlich wird.
	\\[2cm]
	\Large
	Aufgabe 2: Vergleich
	\hrule
	\vspace{0.5cm}
	\noindent
	\large
	Die Szene von Herbrich ist länger und ausführlicher als die Szene von Büchner, die nur aus wenigen Dialogzeilen besteht. Herbrich fügt mehr Details und Handlungen hinzu, die die Situation und die Charaktere veranschaulichen. Zum Beispiel beschreibt er, wie Wodzeck dem Arzt beide Hände hinsteckt, wie der Arzt nur kurz aufschaut, wie Wodzeck von seinen Wahnvorstellungen erzählt, wie der Arzt ihn körperlich untersucht, wie er ihm neue Tabletten verschreibt und wie er ihn zur Tür hinausdrängt. Diese Elemente verstärken die Spannung und den Kontrast zwischen den beiden Figuren.
Die Szene von Herbrich ist in einem realistischen Stil geschrieben, der die medizinischen und psychologischen Aspekte der Szene betont. Die Szene von Büchner ist eher grotesk und absurd, was die Verzerrung und Entmenschlichung der Szene hervorhebt. Zum Beispiel verwendet Büchner eine komische und übertriebene Sprache, die die Lächerlichkeit und Sinnlosigkeit der Situation zeigt. Er lässt den Arzt Woyzeck fragen, ob er ein Mann seines Wortes sei, ob er seine Natur beherrschen könne, ob er noch immer Erbsen esse, ob er noch immer nichts trinke, ob er noch immer nicht rauche. Er lässt Woyzeck antworten, dass er alles tue, was der Arzt ihm sage, dass er sich wie eine Maschine fühle, dass er nichts mehr schmecke, dass er nichts mehr rieche, dass er nichts mehr höre. Diese Elemente zeigen die Absurdität und Grausamkeit der Situation.
Die Szene von Herbrich verlegt den Schauplatz von einem Arztzimmer in ein Behandlungszimmer, das an eine Fabrik erinnert. Damit wird die industrielle und kapitalistische Atmosphäre der Szene verstärkt, die Wodzeck als ein Produkt und nicht als eine Person behandelt. Zum Beispiel beschreibt er, wie der Arzt Wodzeck einen Zettel reicht, der seine Abhängigkeit und Fremdbestimmung bedeutet, wie er ihn an eine Tomografie verweist, die seine organische Ursache klären soll, wie er ihn nicht krankschreiben kann, weil er nichts Handfestes vorliegen hat. Diese Elemente zeigen die Ausbeutung und Entfremdung von Wodzeck.
Die Szene von Herbrich verändert den Namen von Woyzeck in Wodzeck, was eine Anspielung auf den Namen des Autors Georg Büchner sein könnte. Damit wird die Verbindung zwischen dem Schicksal von Wodzeck und dem Schicksal von Büchner hergestellt, der ebenfalls an einer Krankheit litt und früh starb. Dieses Element zeigt die Tragik und Bedeutung von Wodzeck.
\end{document}
\documentclass[12pt, a4paper]{report}

\usepackage[top=2cm,right=2cm,bottom=2cm,left=2cm,landscape]{geometry}
\usepackage{charter}
\usepackage[utf8]{inputenc}
\usepackage[T1]{fontenc}
\usepackage{tabularx}
\usepackage{custompkg}

\begin{document}
	\thispagestyle{empty}
	\bslinespacing{1.25}
	\noindent
	\Huge
	\centering
	\textbf{Dramenaufbau Woyzeck}
	\normalsize
	\\[1cm]
	\begin{tabularx}{\textwidth}{|X|X|X|X|X|X|}
	 \hline
	 \textbf{Akt} & \textbf{I} & \textbf{II} & \textbf{III} & \textbf{IV} & \textbf{V} \\
	 \hline
	 & \textbf{Einleitung} & \textbf{Steigerung} & \textbf{Höhepunkt} & \textbf{Fall/Umkehr} & \textbf{Katastrophe/Lösung} \\
	 \hline
	 \textbf{Szene/Inhalt/} & \textbf{Szene 1-5} & \textbf{Szene 6-10} & \textbf{Szene 11-20} & \textbf{Szene 21-24} & \textbf{Szene 25-27} \\
	 \textbf{Funktion} & Es werden die Personen vorgestellt und man erhält einen Einblick in die Lebenslage der Personen. & Die Situation verschärft sich und Marie betrügt Woyzeck. & Woyzeck bemärkt, dass Marie ihn betrügt und entschließt sich deswegen sie zu töten. Darum lockt er sie in einen Wald und bringt sie auf brutalste Weise mit einem Messer um. & Woyzeck bemärkt, dass er noch Spuren des Mordes entfernen muss und begibt sich deswegen zurück zum Tatort, um die Tatwaffe zu entsorgen.& Das ganze Dorf hat von dem Mord erfahren und sein Sohn hat sich von Woyzeck abgewendet.\\
	 \hline
	\end{tabularx}
\end{document}
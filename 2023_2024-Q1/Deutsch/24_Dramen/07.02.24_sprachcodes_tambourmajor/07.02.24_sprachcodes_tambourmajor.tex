\documentclass[12pt, a4paper]{report}
\usepackage[left=2cm,top=2cm,right=2cm,bottom=2cm,landscape]{geometry}
\usepackage[utf8]{inputenc}
\usepackage[T1]{fontenc}
\usepackage{charter}
\usepackage{ngerman}
\usepackage{tabularx}
\usepackage{custompkg}

\begin{document}
	\newcommand{\elabo}[0]{\texttt{elaboriert}}
	\newcommand{\restr}[0]{\texttt{restringiert}}
	
	\bslinespacing{1.25}
	\noindent
	\Large \textbf{Sprachcode Tambour-major} \\[1cm]
	\normalsize
	\begin{tabularx}{\textwidth}{|X|X|X|X|}
		\hline
		\textbf{Szene} & \textbf{Zitat} & \textbf{Code} & \textbf{Begründung} \\
		\hline
		3 (S.9 Z.19) & \textit{Teufel, zum Fortpflanzen von Kürassierregimenter und zur Zucht von Tambourmajors!}& \elabo & Längerer Satz und gehobene Wortwahl. \\
		\hline
		3 (S.9 Z.24f.) & \textit{Als ob man in ein Ziehbrunn oder zu eim Schornstein hinabguckt} & \elabo & Langer Satz \\
		\hline
		3 (S.9 Z.25)& \textit{Fort hinte drein} & \restr & Kurze, undifferenzierte Aussage. \\
		\hline
		6 (S.14 Z.14)& \textit{Wild Tier.} & \restr & Kurze, undifferenzierte Aussage. \\
		\hline
		6 (S.14 Z.16)& \textit{Sieht dir der Teufel aus den Augen} & \restr & Informelle Wortwahl \\
		\hline
		6 (S.14 Z.6f)& \textit{Wenn ich am Sonntag erst den großen Federbusch hab und die weiße Handschuh, Donnerwetter, Marie, der Prinz sagt immer: Mensch, Er ist ein Kerl.} &\elabo & Verschachtelter Satz\\
		\hline
		14 (S.22 Z.18)& \textit{Ich bin ein Mann!} & \restr & Kurze Aussage \\
		\hline
		14 (S.22 Z.21)& \textit{Ich will ihm die Nas ins Arschloch prügeln.} & \restr & Kurzer Satz; nicht gehobene Sprache. \\
		\hline
		
		\hline
	\end{tabularx}
\end{document}
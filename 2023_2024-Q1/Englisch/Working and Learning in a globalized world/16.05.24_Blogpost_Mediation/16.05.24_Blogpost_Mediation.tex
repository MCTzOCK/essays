\documentclass[12pt,a4paper]{report}
\usepackage[T1]{fontenc}
\usepackage[utf8]{inputenc}
\usepackage{charter}
\usepackage{ngerman}
\usepackage[left=2cm,right=2cm,top=2cm,bottom=2cm]{geometry}

\renewcommand\thesection{\arabic{section}.} 

\begin{document}
	\section{Options in Germany after leaving school}
	In Germany there are many options that students can choose from after finishing school.
	But first of all let me explain the basics of the german education system.
	Every students starts in primary school with an average age of 6 years old.
	After the primary school there are four options: the Hauptschule, the Realschule, the Gesamtschule and the Gymnasium.
	Going to the Hauptschule will eventually end in a Hauptschulabschluss.
	With this degree you can not study at a university but you are able to start an apprenticeship.
	The Realschule is almost the same but the degree is called the Realschulabschluss.
	To study at a university you will need the allgemeine Hochschulreife which is obtainable at the Gesamtschule and the Gymnasium.
	From primary school to a fulfilled degree it takes 12-13 years at minimum, so you will probably be around 18-19 years old after school.
	\\
	I will now explain a few of the different options after school.
	First of all you have the apprenticeship which as mentioned earlier can be chosen with any kind of degree.
	In an apprenticeship you will have to choose a specific job that you later want to do.
	After that you have to find a company that offers apprenticeships in your field of interest and can then apply at that company.
	An apprenticeship will usually last around 2-3 years strongly depending on the profession you want to learn.
	\\
	Another quite popular option is a classic study at a university.
	As mentioned before you will need the allgemeine Hochschulreife to apply for a study.
	You have to decide which field you want to study.
	After the study you are able to work in the field you studied but are not limited to those jobs.
	The study doesn't force you into a specific job like the apprenticeship.
	\\
	Neither a study nor an apprenticeship will cost you much money.
	In fact a study in its entire is mostly free of charge in germany.
	If you have some kind of costs because of your lifestyle it will be fairly easy for you to get a job.
	If you want to study there is also the Duales Studium which increases your study time by a bit, but also lets you work in a company while studying so you can earn money.
	\\
	I hope I could give you a few insights into the german education system and the options after school.
\end{document}
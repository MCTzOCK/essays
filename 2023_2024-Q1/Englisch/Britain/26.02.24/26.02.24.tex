\documentclass[a4paper,12pt]{report}
\usepackage[T1]{fontenc}
\usepackage[utf8]{inputenc}
\usepackage{charter}
\usepackage{ngerman}
\usepackage[left=2cm,right=2cm,top=2cm,bottom=2cm]{geometry}

\newcommand{\ap}[0]{he/she\ }
\newcommand{\apx}[0]{his/her\ }

\begin{document}
	\noindent
	\Large Number 1: Complete the sentences
	\\[0.5cm]
	\large
	The editorial poses the question whether the monarchy has a future.\\\\
	In the article it is stressed that the monarch, Queen Elizabeth, is seen as honest and is respected by the public. \\\\
	All in all, the Queen's reign is regarded as successful because she always found the right words and made almost no mistakes. \\\\
	Despite these positive comments, the editorial claims that the monarchy is outdated.\\\\
	\Large Number 2: Analysis
	\\[0.5cm]
	\large
	The author of the text speeks about the Queen very positively without stating \apx own opinion.
	This is because \ap only mentions facts such as her amazing public ratings (ll. 26-27).
	But even when \ap talks about the negative aspects, such as mistakes the Queen made \ap states that there were only a few of them and doesn't mention any of them which makes the Queen look really good. (ll. 24).
	Whilst talking about her mistakes he uses the phrase \dq incredible few\dq\ which makes it sound like there were almost no mistakes at all.
	In the second half of \apx text there is a break.
	After which \ap gets very critical about the monacharial system.
	But even though \ap is that criticial \ap never says something against the Queen but states that the monarchy will fail after the Queens death.
\end{document}
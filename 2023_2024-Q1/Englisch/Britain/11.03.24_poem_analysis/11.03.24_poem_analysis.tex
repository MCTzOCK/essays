\documentclass[12pt,a4paper]{report}
\usepackage[T1]{fontenc}
\usepackage[utf8]{inputenc}
\usepackage{charter}
\usepackage{ngerman}
\usepackage[left=2cm,right=2cm,top=2cm,bottom=2cm]{geometry}
\usepackage{tabularx}

\begin{document}
	\thispagestyle{empty}
	\noindent
	\Large Working with the poem
	\large
	\paragraph{a) What does the poem say about the British?} \mbox{} \\
	The British people consist of many different nations.
	
	\paragraph{b) Identify themes focused on} \mbox{} \\
	equality, languages, Britain's history, diversity, the British Empire and the Commonwealth
	\paragraph{c) Comment on the style of the poem} \mbox{} \\
	It's written like a recipe which makes it quite unusual for a poem, because there is no rhyme
	\paragraph{d) Explain the tone of the poem} \mbox{} \\
	The poem uses a very instructive and formal tone.
	\paragraph{e) Why might the first verse be seen as an introduction to the British} \mbox{} \\
	Because the first verse includes many different ethnicities who are living in Britain or have lived their in the past.
	\paragraph{f) What functions does language play and what does Zephaniah suggest} \mbox{} \\
	He suggests to glue all nations together by speaking the English language; input from other languages
	\paragraph{g) Interpret l. 13 and l. 19} \mbox{} \\
	Those lines are very instructive with instructions from a typical recipe.
	l. 13 \dq turn up the heat\dq\ could be a Metaphor to describe the hate between the different people.
	l. 19 \dq Leave the ingredients to simmer\dq\ is also most likely a Metaphor to describe that conflicts between people should firstly be tried to be solved on their own.
	\paragraph{h) What is the implication behind l. 22 and the following three lines} \mbox{} \\
	Time has to pass until the people get unity, understanding, respect for the future and justice.
	\paragraph{i) Explain the last four lines} \mbox{} \\
	The mention that it is important that everyone is treated equally and that justice is spread around all people.
	\paragraph{2nd Task: Table} \mbox{} \\\\
	\begin{tabularx}{\linewidth}{|X|X|}
		\hline
		West Indies & Caribbean, Jameica \\
		\hline
		South Asia & India, Pakistan, Bangladesh, Sri Lanka, Nepal, ... \\
		\hline
		Africa & Nigeria, Ghana, Kenya, Somalia, South Africa, etc. \\
		\hline
		Asia & China, Philippines, Vietnam, Thailand, Malaysia, etc. \\
		\hline
		Others &  \\
		\hline
	\end{tabularx}
\end{document}
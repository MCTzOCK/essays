\documentclass[12pt,a4paper]{report}
\usepackage[T1]{fontenc}
\usepackage[utf8]{inputenc}
\usepackage{charter}
\usepackage{ngerman}
\usepackage[left=2cm,right=2cm,top=2cm,bottom=2cm]{geometry}
\usepackage{custompkg}
\usepackage{hyperref}
\usepackage{xcolor}
\usepackage{tabularx}

\begin{document}
	\bsremovechaptertitle
	\title{Presse}
	\date{\today}
	\maketitle
	\noindent
	\chapter{Internetadresse der Schule}
	Die Webseite des Gymnasiums Holthausen ist unter \color{blue}https://gyho.de \color{black} erreichbar 
	\chapter{Zitate}
	\paragraph{Was machst das Exponat so besonders?} \mbox{}
	\begin{enumerate}
		\item \dq Die MINT-Nacht gibt es nur an unserer Schule.\dq\ (\texttt{Salome Böke, Q1})
		\item \dq Es ist eine tolle Aktion, um Grundschülern die Welt der Naturwissenschaften auf spaßige Art und Weise näher zu bringen.\dq\ (\texttt{Lana Schulte, Q1})
		\item \dq Das einzigartige Konzept, welches bei jungen Schülerinnen und Schüler ein Interesse für MINT-Fächer weckt.\dq\ (\texttt{Ben Siebert, Q1})
	\end{enumerate}
	\begin{tabularx}{\textwidth}{|X|X|}
		\hline
		Frage & Antworten \\
		\hline
		Wie fandest du die MINT-Nacht insgesamt? & Sehr gut;  Sehr schön gestaltet und mit spannenden Experimenten; Gut \\
		\hline
		Was hat dir besonders gefallen? & Die Show (Mitmachexperimente); Die Experimente; Mir haben die Experimente woran man selbst teilnehmen konnte sehr gut gefallen und das Fotoshooting mit Minti war toll! \\
		\hline
		Würdest du die MINT-Nacht weiterempfehlen? & Ja; Ja;  Ich würde sie auf jeden Fall weiterempfehlen weil es sehr viele spaß gemacht hat und die Experimente sehr schön und verständlich gestaltet waren. \\
		\hline
		Was war das beste Erlebnis der MINT-Nacht? & Mein Highlight war die Mint-Show; Als ich auf die Bühne gerufen wurde; Die Experimente \\
		\hline
		Hat die MINT-Nacht deine Schullaufbahn beeinflusst? & ein Bisschen; Das Interesse für MINT wurde geweckt; Ja \\
		\hline
		Verbesserungsvorschläge und Anmerkungen & Keine; Nein; 
Toll war auch das wir von Schülern durch die Mint-Nacht begleitet wurden. \\
		\hline
	\end{tabularx}
	\chapter{Bilder}
	\begin{enumerate}
		\item Von Minti \& Pipetta
		\item Vom Projektkurs
		\item Von den Teilnehmern (Kinder)
		\item Während eines Experiments (Raum)
	\end{enumerate}
	
\end{document}
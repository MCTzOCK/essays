%! Author = ben
%! Date = 23.10.2023

\documentclass[./entry.tex]{subfiles}

\begin{document}
    \chapter{Anhang}

    \section{Fachspezifische Begriffe und Konzepte}

    \paragraph{Lineare\ Datenstrukturen}
    Lineare Datenstrukturen sind Datenstrukturen, die die Elemente in einer bestimmten Reihenfolge speichern. Die
    Reihenfolge wird durch die Reihenfolge der Elemente bestimmt. Die Elemente werden nacheinander gespeichert und
    können nur über die Position im Speicher angesprochen werden. Die bekannteste lineare Datenstruktur ist
    das Array. Weitere lineare Datenstrukturen sind Listen, Schlangen und Stapel.
    Einer der größten Unterschiede der verschiedenen linearen Datenstrukturen ist die Zugriffsmöglichkeit auf die
    Elemente. Bei Arrays kann auf die Elemente über den Index zugegriffen werden. Bei Listen kann auf die Elemente
    über einen Zeiger zugegriffen werden. Bei Schlangen kann nur auf das erste Element und bei
    Stapeln nur auf das oberste Element zugegriffen werden.\footnote{\bscite{linear-datastructures}}

    \paragraph{Big-O Notation}
    Die Big-O-Notation ist eine mathematische Notation, die verwendet wird, um das asymptotische Verhalten von
    Funktionen zu beschreiben, das heißt das Grenzwertverhalten der Funktion zu klassifizieren\footnote{\bscite{asymptomic-analysis}}.
    Die Big-O-Notation wird verwendet, um die Laufzeit eines Algorithmus in Abhängigkeit von der Anzahl der zu
    sortierenden Elemente zu beschreiben. Dabei wird die Anzahl der zu sortierenden Elemente mit $n$ bezeichnet.
    Die Laufzeit wird in Abhängigkeit von $n$ angegeben. Die Laufzeit wird mit $O(n)$ angegeben.
    \newpage
    Dabei ist $O(n)$ die obere Schranke der Laufzeit.\footnote{\bscite{big-o-notation}}

    \paragraph{Rekursion}
    Rekursion ist ein Prinzip, das in der Informatik verwendet wird, um ein Problem in kleinere Teilprobleme zu
    zerlegen. Dabei wird das Problem so lange in kleinere Teilprobleme zerlegt, bis diese einfach zu lösen sind.
    Anschließend werden die Teilprobleme gelöst und die Lösungen werden zu einer Lösung des ursprünglichen Problems
    zusammengefügt.\footnote{\bscite{recursion}}


    \section{Java-Quellcode der Sortieralgorithmen}

    \paragraph{BubbleSort} \mbox{}\\
    \begin{lstlisting}
        public static int[] bubbleSort(int[] unsorted) {
            int[] sorted = unsorted.clone();
            int temp = 0;
            for (int i = 0; i < sorted.length; i++) {
                for (int j = 1; j < (sorted.length - i); j++) {
                    if (sorted[j - 1] > sorted[j]) {
                        temp = sorted[j - 1];
                        sorted[j - 1] = sorted[j];
                        sorted[j] = temp;
                    }
                }
            }
            return sorted;
        }
    \end{lstlisting}

    \paragraph{SelectionSort} \mbox{}\\
    \begin{lstlisting}
        public static int[] selectionSort(int[] unsorted) {
            int[] sorted = unsorted.clone();
            int temp = 0;
            for (int i = 0; i < sorted.length - 1; i++) {
                int min = i;
                for (int j = i + 1; j < sorted.length; j++) {
                    if (sorted[j] < sorted[min]) {
                        min = j;
                    }
                }
                temp = sorted[min];
                sorted[min] = sorted[i];
                sorted[i] = temp;
            }
            return sorted;
        }
    \end{lstlisting}

    \paragraph{InsertionSort} \mbox{}\\
    \begin{lstlisting}
        public static int[] insertionSort(int[] unsorted) {
            int[] sorted = unsorted.clone();
            int temp = 0;
            for (int i = 1; i < sorted.length; i++) {
                for (int j = i; j > 0; j--) {
                    if (sorted[j] < sorted[j - 1]) {
                        temp = sorted[j];
                        sorted[j] = sorted[j - 1];
                        sorted[j - 1] = temp;
                    }
                }
            }
            return sorted;
        }
    \end{lstlisting}

\end{document}
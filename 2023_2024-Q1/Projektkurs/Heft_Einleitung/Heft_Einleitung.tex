\documentclass[12pt,a4paper]{report}
\usepackage[T1]{fontenc}
\usepackage[utf8]{inputenc}
\usepackage{charter}
\usepackage{ngerman}
\usepackage[left=2cm,right=2cm,top=2cm,bottom=2cm]{geometry}

\newcommand{\gen}[0]{:innen }
\newcommand{\gens}[0]{:innen}
\newcommand{\mn}[0]{MINT-Nacht }
\newcommand{\mns}[0]{MINT-Nacht}

\begin{document}
	\thispagestyle{empty}
	\noindent
	\large
	\paragraph{Einleitungstext MINT-Nacht-Heft} \mbox{} \\[0.5cm]
	Liebe Leser\gens, \\
	wir sind sehr erfreut über Ihr Interesse an der \mns.
	Die \mn\ ist ein Projekt, welches am Gymnasium Holthausen in Hattingen entwickelt und zuerst im Jahr 2019 durchgeführt wurde.
	Das Konzept ist darauf ausgelegt, Interesse für MINT bei Viertklässler\gen zu wecken bzw. zu fördern.
	Gegliedert ist die Veranstaltung in ein Bühnenprogramm und praktische Experimente, die durch die Grundschüler\gen selbst und unter Aufsicht durchgeführt werden.
	Alle Experimente wurden durch den Projektkurs der Q1 geplant und mit den Viertklässler\gen erarbeitet.
	Um die Sicherheit zu gewähren, befindet sich in jedem Raum mindestens eine Lehrkraft.
	\\
	Die Show auf der Bühne wird von den beiden Maskottchen der \mns, Minti und Pipetta, moderiert.
	Der Erlenmeyerkolben Minti und die Pipette Pipetta werden von Schülern gespielt, die sich im Vorfeld ein Ablaufskript für die Show überlegen.
	\\
	\[Bild\ von \ Minti\ \&\ Pipetta\]
	\\
	In jeder \mn gibt es über 15 verschiedene Raum- und Bühnenexperimente, die von Jahr zu Jahr variieren.
	Dabei teilt sich alles auf die unterschiedlichsten Themenbereiche, Physik, Mathematik, Informatik, Chemie, Biologie oder auch Technik auf.
	Die \mn ist für viele Viertklässler\gen der erste Berührungspunkt mit Naturwissenschaften, da diese in der Grundschule meist nicht thematisiert werden.
	\paragraph{Schlusswort MINT-Nacht-Heft} \mbox{} \\
	Oben sehen Sie ein Bild des Projektkurses Naturwissenschaften der Q1.
	Dessen Schüler\gen haben die MINT-Nacht geplant und dieses Heft entworfen.
	Zur Planung gehörte das Ausprobieren spannender Experimente, das Protokollieren der Versuchsaufbauten und die Konzeption eines Bühnenprogrammes.
	Am Gymnasium Holthausen wird viel experimentiert, nicht nur im Unterricht, sondern auch in den verschiedensten Arbeitsgemeinschaften.
	Geleitet wird der Projektkurs von Frau Dr. Schmidtseifer-Sürig und Frau Ricke, diese haben sich die MINT-Nacht ausgedacht, um Viertklässler\gen für MINT zu begeistern.
	Daher gibt es die MINT-Nacht exklusiv am Gymnasium Holthausen.
\end{document}
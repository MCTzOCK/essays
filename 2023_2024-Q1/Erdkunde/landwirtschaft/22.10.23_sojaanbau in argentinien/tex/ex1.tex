%! Author = ben
%! Date = 22.10.2023

\documentclass[./entry.tex]{subfiles}

\begin{document}

    Die Anbaufläche von Soja in Argentinien ist seit 1988 stetig
    gestiegen. So gab es 1988 eine gesamt Anbaufläche von 4 Millionen
    Hektar und nur 12 Jahre später, 2000, eine Anbaufläche von 9
    Millionen Hektar. 2012 wurden bereits 19 Millionen Hektar für den
    Anbau von Soja genutzt.
    Die Sojabohnen-Ernte ist ebenfalls stark gestiegen. 1988 wurden
    nur 10 Millionen Tonnen Sojabohnen geerntet, 2012 waren es bereits
    52 Millionen Tonnen. (M1)

    \begin{table}[h]
        \centering
        \begin{tabular}{|l|l|l|l|l|}
            \hline
            \textbf{Jahr} & \textbf{Anbaufläche} & \textbf{Ernte} & \textbf{Ertrag pro Hektar} \\ \hline
            1988          & 4 Mio. ha            & 10 Mio. t      & 2,5 t/ha                   \\ \hline
            2000          & 9 Mio. ha            & 20 Mio. t      & 3,3 t/ha                   \\ \hline
            2012          & 19 Mio. ha           & 52 Mio. t      & 2,7 t/ha                   \\ \hline
        \end{tabular}
        \caption{Anbaufläche im Vergleich zur Ernte}
        \label{tab:table1}
    \end{table}

    2012 gingen allein 22 \% der Ausfuhren Argentiniens auf die
    Sojaproduktion zurück. Ebenfalls war Argentinien mit 24 \%
    Anteil am Weltmarkt einer der größten Exporteure von Soja. (M2)
    Man kann also erkennen, dass Soja eine sehr wichtige Rolle in
    Argentinien spielt. Es ist ein wichtiger Wirtschaftsfaktor und
    die Anbaufläche und Ernte steigen stetig.

\end{document}
%! Author = ben
%! Date = 22.10.2023

\documentclass[./entry.tex]{subfiles}

\begin{document}
    Die Fläche, die in Argentinien landwirtschaftlich genutzt wird, steigt stetig.
    Während 1995 46,7 \% der Fläche landwirtschaftlich genutzt wurden, waren es 2009 bereits 51,3 \%.
    Der Anteil der Erwebstätigen im Landwirtschaftssektor ist jedoch rückläufig.
    1995 waren es ca. 1,5 Millionen Menschen. Diese Zahl sank bis 2010 auf etwa 1,4 Millionen. (M5)

    Seit 1997 wurden allerdings immer mehr Pestizide verkauft und.
    1997 waren es lediglich 125 Millionen Kilogramm.
    Die Menge stieg bis 2005 bereits auf über 270 Millionen Kilogramm
    und erreichte 2013 die 300 Millionen Kilogramm Marke.
    Diese Pestizide sind extrem schädlich für die Umwelt und die Menschen,
    die damit in Kontakt kommen. (M6)

    Es ist also zu erkennen, dass die Landwirtschaft in Argentinien immer weiter wächst,
    jedoch immer weniger Menschen in diesem Sektor arbeiten und immer mehr Pestizide eingesetzt werden.

    Sojaanbau ist eine industrielle Landwirtschaft.
    Das kann man vor allem an folgenden Aspekten erkennen (M3):
    \begin{itemize}
        \item Soja wird meist in Monokulturen angebaut, was zu einer geringen Biodiversität führt.
        \item Im Sojaanbau werden viele Pestizide eingesetzt, die die Umwelt und die Menschen schädigen, aber die Ernteerträge steigern.
        \item Es wird viel Soja angebaut, um es als Futtermittel für die Massentierhaltung zu nutzen.
    \end{itemize}

\end{document}
\documentclass[12pt, a4paper]{report}

\usepackage{ngerman}
\usepackage[utf8]{inputenc}
\usepackage[T1]{fontenc}
\usepackage{custompkg}
\usepackage{eurosym}
\usepackage[left=2cm,right=2cm,top=2cm,bottom=2cm]{geometry}

\begin{document}
	\bslinespacing{1.5}
	\bsremovechaptertitle
	\chapter*
{Entwicklung des ökologischen Landbaus in Deutschland}
	Die Fläche, die in Deutschland für den ökologischen Landbau verwendet wird, ist seit 1996 extrem stark gestiegen.
	1996 waren es ca. 350.000ha, allerdings verdoppelte sich diese bereits innerhalb der nächsten 6 Jahre bis 2002.
	Nach 2002 ist der Anstieg der Fläche abgeflacht und stieg bis 2012 auf etwas über 1.000.000ha an.
	Der Anteil des ökologischen Anbaus an der gesamten landwirtschaftlichen Fläche, ist jedoch auch 2012 nur relativ gering.
	Von 1996 bis 2012 hat sich dieser zwar verdreifacht, jedoch liegt er 2012 nur bei 6,2\%.
	Die Anzahl der ökologischen Betriebe hat eine ähnliche Entwicklung mitgemacht.
	Von 1996 bis etwa 2000 hat sich die Anzahl der ökologischen Betriebe von 5.000 auf über 12.000 mehr als verdoppelt.
	Der Anstieg der Betriebe war also noch schneller, als der der Fläche, woraus sich schließen lässt, dass die durchschnittliche Fläche jedes individuellen Betriebes gesunken ist.
	Auch der Anstieg der Betriebe ist zwischen 2002 und 2012 etwas abgeflacht, ist allerdings in diesem Zeitraum trotzdem von etwa. 16.000 auf über 23.000 um etwa 7.000 angestiegen (M5).
	\\
	Der ökologische Fußabdruck in Deutschland liegt durchschnittlich bei 4,2 gha pro Person und ist so deutlich höher als der Weltdurchschnitt, welcher bei 2,7 gha pro Person liegt.
	Den größten Anteil an diesem ökologischen Fußabdruck hat die Ernährung mit durchschnittlich 35\%.
	Dieser ist deutlich höher als der Anteil des Wohnens, welcher mit 25\% den zweitgrößten Anteil ausmacht (M6).
	\\
	Doch selbst auf die Bundesländer Deutschlands gesehen, ist der ökologische Landbau sehr ungleich verteilt.
	Das Saarland verwendet mit 16\% prozentual auf seine Größe gerechnet den größten Anteil der landwirtschaftlich genutzten Fläche für den ökologischen Landbau.
	Allerdings macht der Anteil dieser Fläche im Saarland nur gerade einmal 0.9\% der gesamten, für ökologischen Landbau verwendeten, Fläche in Deutschland aus.
	Auf ganz Deutschland bezogen ist Bayern mit 22,9\% das Bundesland, welches die meiste Fläche für den ökologischen Landbau benutzt.
	Insgesamt wird in den meisten Bundesländern zwischen 5 und 12\% der landwirtschaftlich genutzten Fläche für den ökologischen Landbau verwendet.
	Nur 11\% der landwirtschaftlichen Betriebe in Deutschland sind ökologisch, wobei der größte Anteil dieser in Barden-Württemberg mit 21,7\% verordnet ist.
	Auffallend ist, dass vor allem die Bundesländer im Süden und im Nordosten den größten Anteil an der insgesamten landwirtschaftlichen Fläche haben.
	Niedersachsen ist das Bundesland, welches am wenigsten von seiner Fläche für den ökologischen Landbau verwendet (M7,M8).
	\\
	76\% der Deutschen achten gezielt oder regelmäßig darauf Bioprodukte zu konsumieren. 
	Der Biomarkt ist insgesamt ein sehr dynamischer Markt und verzeichnet pro Jahr etwa 8\% Umsatzsteigerung. 
	Ebenfalls ist die Wertschöpfung in der Biobranche sehr hoch.
	Der ökologische Landbau gilt als relevante Schlüsseltechnologie für eine nachhaltigere Zukunft.
	Bis 2050 sollen 20\% der landwirtschaftlichen Gesamtfläche für den ökologischen Landbau eingesetzt werden.
	Heute sind es allerdings nur ca. 6\%, wodurch die Erreichung dieses Ziels noch in weiter Ferne liegt.
	Damit dieses Ziel erreicht werden kann, muss die Nachfrage nach biologischen Produkten immer weiter dynamisch ansteigen und dike Politik muss als Impulsgeber für die etwa 35.000 Landwirte dienen, die ihre Produktion umstellen müssen (M9).
	\\
	Im Vergleich zu anderen Ländern in der Europäischen Union ist Deutschland mit etwa 1.088 Mio. ha das Land, welches insgesamt die zweitgrößte Fläche für den ökologischen Landbau verwendet.
	Das Land mit der größten Fläche für den ökologischen Landbau ist Spanien mit fast 2 Mio. ha.
	Prozentual betrachtet liegt Deutschland allerdings nur auf Platz 7, hinter Österreich, Schweden, Estland, Finnland, Slowenien und Spanien.
	Das Land, welches prozentual am meisten Fläche verwendet ist Österreich, welches 20,3\% seiner gesamten Fläche für den ökologischen Landbau aufbringt.
	Deutschland hingegen benutzt nur 6,5\% (M10).
	\\
	Im Gegensatz zu der konventionellen Landwirtschaft verwendet die ökologische Landwirtschaft deutlich weniger Euro pro Hektar für Düngemittel und Pflanzenschutzmittel.
	Jedoch sind die Personalkosten deutlich höher, als die der konventionellen Landwirtschaft.
	Im Durchschnitt werden in der konventionellen Landwirtschaft etwa 72 \euro /ha für Personal ausgegeben, während es bei der ökologischen Landwirtschaft 151 \euro /ha sind.
	Die Erträge sind in der ökologischen allerdings auch geringer, als in der konventionellen Landwirtschaft.
	Pro Kuh werden durchschnittlich 7096 L in der konventionellen Landwirtschaft erwirtschaftet, während es in der ökologischen Landwirtschaft nur 5585 L sind.
	Auch bei Weizen sind die Erträge deutlich geringer,
	sodass nur 3400kg/ha im Vergleich zu 7800kg/ha bei der konventionellen Landwirtschaft erwirtschaftet werden.
	Ein weiteres Problem des ökologischen Landbaus ist der deutlich höherer Preis der Endprodukte, sodass 1kg Kartoffeln mit 34,25\euro\ um 24,51\euro\ teurer sind, als Kartoffeln aus der konventionellen Landwirtschaft (M11).
\end{document}









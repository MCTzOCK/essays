\documentclass[12pt, a4paper]{report}

\usepackage{ngerman}
\usepackage[utf8]{inputenc}
\usepackage[T1]{fontenc}
\usepackage{custompkg}
\usepackage[left=2cm,right=2cm,top=2cm,bottom=2cm]{geometry}

\begin{document}
	\bslinespacing{1.5}
	\bsremovechaptertitle
	\chapter{Wirtschaftsweise ökologischer Landbau}
	Im ökologischen Landbau wird die Landwirtschaft als Wirtschaften im Einklang mit der Natur angesehen.
	Ein landwirtschaftlicher Betrieb wird als ein Organismus angesehen, welcher die Bestandteile Tier, Boden, Mensch und Pflanze hat.
	Es ist wichtig für so einen ökologischen Betrieb, dass es einen möglichst kleinen Nährstoffkreislauf gibt.
	Außerdem wird großer Wert auf die artgerechte Haltung von Tieren und die Erhaltung der Bodenfruchtbarkeit gesetzt. \\
	Es wird absolut kein Pflanzenschutzmittel mit chemisch-synthetischen Mitteln verwendet.
	Ebenfalls gibt es keine Anwendung von leicht löslicher mineralischer Dünger.
	Die ausgeprägte Humuswirtschaft pflegt die Fruchtbarkeit des Bodens.
	Im Bezug auf die Tiere wird weitgehend auf Antibiotika verzichtet, sofern diese nicht das einzige Mittel sind.
	(M1) \\
\end{document}
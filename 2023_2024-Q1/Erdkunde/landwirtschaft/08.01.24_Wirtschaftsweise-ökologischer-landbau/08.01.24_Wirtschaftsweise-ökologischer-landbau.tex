\documentclass[12pt, a4paper]{report}

\usepackage{ngerman}
\usepackage[utf8]{inputenc}
\usepackage[T1]{fontenc}
\usepackage{custompkg}
\usepackage[left=2cm,right=2cm,top=2cm,bottom=2cm]{geometry}

\begin{document}
	\bslinespacing{1.5}
	\bsremovechaptertitle
	\chapter{Wirtschaftsweise ökologischer Landbau}
	Im ökologischen Landbau wird die Landwirtschaft als Wirtschaften im Einklang mit der Natur angesehen.
	Ein landwirtschaftlicher Betrieb wird als ein Organismus angesehen, welcher die Bestandteile Tier, Boden, Mensch und Pflanze hat.
	Es ist wichtig für so einen ökologischen Betrieb, dass es einen möglichst kleinen Nährstoffkreislauf gibt.
	Außerdem wird großer Wert auf die artgerechte Haltung von Tieren und die Erhaltung der Bodenfruchtbarkeit gesetzt. \\
	Es wird absolut kein Pflanzenschutzmittel mit chemisch-synthetischen Mitteln verwendet.
	Ebenfalls gibt es keine Anwendung von leicht löslicher mineralischer Dünger.
	Die ausgeprägte Humuswirtschaft pflegt die Fruchtbarkeit des Bodens.
	Im Bezug auf die Tiere wird weitgehend auf Antibiotika verzichtet, sofern diese nicht das einzige Mittel sind (M1).
	\\
	Durch ökologischen Landbau wird die Humusbildung und das Bodenleben gefördert.
	Ebenfalls der Gewässerschutz ist beim ökologischen Landbau sehr wichtig.
	Es gelangen weniger Nährstoffe, wie Nitrat, in das Grundwasser, wodurch diese weniger belastet wird.
	Das ist vor allem auf den Ausschluss von chemisch-synthetischen Pflanzenschutzmitteln zurückzuführen.
	Dieser Verzicht auf solche Schutzmittel hat auch zu Folge, dass die Vielfalt des Tier- und Pflanzenlebens gefördert wird.
	Den gehaltenen Tieren erhalten genügend Auslauf und deren Haltungsbedingungen werden regelmäßig überprüft (M2).
	\\
	Das Konzept des ökologischen Landbaus (geschlossener Kreislauf) unterscheidet sich deutlich vom konventionellen Konzept, dem offenen System.
	Beim geschlossenen Kreislauf werden sämtliche Futtermittel selbst erzeugt, während beim offenen System auch industriell gefertigte Futtermittel zum Einsatz kommen.
	Auch bei den Pflanzen gibt es große Unterschiede, so werden beim geschlossenen Kreislauf keinerlei chemisch-synthetische Pflanzenschutzmittel verwendet, welche beim offenen System durchaus zum Einsatz kommen.
	Die Dünung dieser Pflanzen basiert beim geschlossenen Kreislauf allein auf selbst erzeugten, organischen Mitteln, während das offene System auf synthetische Düngemittel und eine Überproduktion setzt.
	Beim geschlossenen Kreislauf wird großer Wert auf eine artgerechte Haltung mit genügend Auslauf gesetzt, während beim offenen System oft Massentierhaltung zum Einsatz kommt.
	Ebenfalls werden im geschlossenen Kreislauf keinerlei Antibiotika angewendet, wenn diese nicht absolut erforderlich sind, was ein großer Unterschied zum offenen System ist, wo regelmäßig Antibiotika eingesetzt werden.
	Das offene System sieht vor, dass Mastvieh nachgekauft wird, während der geschlossene Kreislauf auf eigene Nachzucht setzt (M3).
\end{document}
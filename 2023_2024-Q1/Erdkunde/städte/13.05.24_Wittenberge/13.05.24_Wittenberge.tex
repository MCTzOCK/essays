\documentclass[12pt,a4paper]{report}
\usepackage[T1]{fontenc}
\usepackage[utf8]{inputenc}
\usepackage{charter}
\usepackage{ngerman}
\usepackage[left=2cm,right=2cm,top=2cm,bottom=2cm]{geometry}
\usepackage{gensymb}

\renewcommand\thesection{\arabic{section}.}

\begin{document}
	\section{Lokalisierung Wittenberge}	Wittenberge ist eine Stadt im Bundesland Brandenburg in Deutschland, welche an der Grenze zu Sachsen-Anhalt liegt.
	Deutschland ist ein Land, welches sich auf der Nordhalbkugel auf dem Kontinent Europa befindet.
	Deutschland ist im Zentrum des Kontinents verordnet.
	Die Nachbarländer von Deutschland sind Frankreich, Luxemburg, die Niederlande und Belgien im Westen, Dänemark im Norden, Österreich und die Schweiz im Süden und Polen und Tschechien im Osten.
	\\
	Brandenburg liegt im Nordosten Deutschlands und hat als Landeshauptstadt Potsdam.
	Wittenberge liegt im Nordwesten von Brandenburg bei etwa 53\degree N 12,3\degree O.
	In der Nähe von Wittenberge befindet sich im Süden die Elbe, welche in ihrem späteren Verlauf in die Nordsee mündet.
	Durch Wittenberge fließt der Fluss Stepenitz.
	Im Südosten von Wittenberge liegt die Bundeshauptstadt Berlin und im Nordwesten liegt Hamburg.
	Größere Städte gibt es in der näheren Umgebung keine.
	\section{Probleme der Bevölkerungssinkung}
	\begin{itemize}
		\item Herausforderung für Einzelhandel / Gastro etc. $\to$ Keine Kunden
		\item Leerstand
		\item Arbeitskräfte fehlen bzw. werden weniger
		\begin{itemize}
			\item[$\to$] wenig attraktiv für Firmen
		\end{itemize}
		\item fehlende Attraktivität für Menschen
		\item Neubau löst Probleme nicht
		\item Infrastruktur könnte leiden z.B. Schulen, Krankenhäuser, etc.
	\end{itemize}
	\section{AB Nr. 1/2}
	Die Bevölkerung Wittenberges stieg von 1870 bis etwa 1920 sehr linear von etwa 7.000 auf etwa 25.000 Bewohner an.
	Nach einer kürzen Abflachung des Wachstums stieg die Bevölkerung in der NS-Zeit extrem an.
	zwischen etwa 1930 und 1942 kamen etwa 8.000 Menschen dazu, sodass ca. 32.000 Menschen in Wittenberge wohnten.
	Während der DDR-Zeit blieb die Bevölkerung zwischen 1950 und 1985 auf dem gleichen Niveau, bis sie dann schließlich zwischen etwa 1986 und 201 wieder auf etwa 17.000 Menschen fiel.
	Zwischen 2010 und 2020 blieb das Niveau in etwa gleich, jedoch ist eine klare Tendenz nach unten zu erkennen.
	2020 lag die Bevölkerungsanzahl zuletzt bei etwa 16.000 Menschen.
	\\
	Die Altersstruktur in Wittenberge ist was die Einwohner unter 15 Jahren und über 64 Jahren angeht, seit 2005 weitgehend gleich geblieben.
	2010 hab es in etwa 1.800 Menschen unter 15 Jahren und 6.000 Menschen über 64 Jahren.
	Was sich jedoch extrem verändert hat ist die Anzahl der Menschen zwischen 15 und 64 Jahren.
	Lag diese 2005 noch bei etwa. 13.000 Menschen, so sunk sie bis 2020 auf gerade einmal 9.000 Menschen.
	Für 2030 ist prognostiziert, dass diese Anzahl weiter auf etwa 7.000 Menschen abfällt.
	Auch die Anzahl der Jugendlichen soll auf etwa 1.7000 abfallen.
	Die Anzahl der Senioren wird laut Prognose weitgehend gleich bleiben.
	\\
	Die Altersstruktur Wittenberges basiert darauf, dass in den untersten, sowie in den obersten Altersklassen etwas das gleiche Niveau an Menschen bleibt, jedoch die mittleren Jahrgänge stark stagnieren.
	Dies könnte zum Beispiel den Grund haben, dass die im Berufsleben stehenden Menschen ein anderes Jobangebot an einem attraktiveren Standort erhalten haben, oder das junge Erwachsene die Stadt verlassen.
	Die gleichbleibende Anzahl der Senioren spricht dafür, dass einige Menschen zu Beginn der Rente in das Dorf ziehen und auf dem gleich Niveau Menschen sterben.
	\section{AB Nr. 3}
\end{document}
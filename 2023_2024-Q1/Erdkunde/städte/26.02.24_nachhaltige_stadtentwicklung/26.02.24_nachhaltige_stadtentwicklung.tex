\documentclass[12pt,a4paper]{report}
\usepackage[T1]{fontenc}
\usepackage[utf8]{inputenc}
\usepackage{charter}
\usepackage{ngerman}
\usepackage{xcolor}
\usepackage[left=2cm,right=2cm,top=2cm,bottom=2cm]{geometry}

\newcommand{\psx}[0]{\color{green!100}{$(+)$}\color{black}\ }
\newcommand{\ngx}[0]{\color{red}$(-)$\color{black}\ }
\newcommand{\ssx}[0]{\color{blue}$(/)$\color{black}\ }

\begin{document}
	\thispagestyle{empty}	
	\noindent
	\Large
	Nachhaltige Stadtentwicklung - die wichtigen Merkmale
	\hrule
	\large
	\vspace{0.25cm}
	\begin{enumerate}
		\item \color{red}relativ hohe bauliche Dichte \color{black} $\to$ \textbf{ABER} keine Großwohnsiedlung / innerstädtische Nachverdichtung \\
		$\to$ innerstädtische Nachverdichtung
		\item \color{red}Kurze Wege z.B. zum Supermarkt, Schule, etc. \color{black} \\
		$\to$ Funktionsmischung (Wohnen, Arbeiten, Freizeit, etc.)
		\item \color{red}Reduzierung des motorisierten Individualverkehrs (MIV)\color{black} \\
		$\to$ Ausbau des \color{red}ÖPNV\color{black} / \textbf{ODER} Stärkung anderer Verkehrsmittel
		\item \color{red}gemeinschaftlich\color{black}\  nutzbare Fläche \\
		$\to$ z.B. Parks, Innenhöfe
		\item \color{red}soziale Mischung\color{black} \\
		$\to$ unterschiedliche Wohnungstypen / Haustypen \\
		$\to$ geringere Mieten für weniger wohlhabende Bevölkerung durch staatliche Förderung
	\end{enumerate}
	\Large
	Anwendung auf Hannover Kronsberg
	\hrule
	\vspace{0.25cm}
	\large
	\begin{enumerate}
		\item \psx Dichte Bebauung ist gegeben \\
		\psx bis zu 4-geschossige Bauweise \\
		\ngx Reihenhäuser im Osten \\
		\psx Häuser sind eng aneinander gebaut \\
		\psx 12-15 tsd. Menschen leben dort in 6000 Wohnungen \\
		\ssx ehemals landwirtschaftliche Fläche
		\item \psx Kindergärten, Grundschulen + Gesamtschule vorhanden \\
		\psx gute Anbindung an die Innenstadt per ÖPNV (20 Minuten Fahrzeit, alle 10 Minuten) \\
		\psx Jede Wohnung ist maximal 500m von einer Haltestelle entfernt. \\
		\psx Arbeitsplätze im Viertel vorhanden \\
		\psx Versorgung im Viertel möglich
		\item \psx Infrastruktur ist auf Fußgänger und Fahrradfahrer ausgerichtet \\
		\psx Carsharing \\
		\psx gute Anbindung an die Innenstadt per ÖPNV (20 Minuten Fahrzeit, alle 10 Minuten) \\
		\psx Jede Wohnung ist maximal 500m von einer Haltestelle entfernt. \\
	\end{enumerate}
	\textbf{ökologische Aspekte}: \\
	\psx Niedrigenergiehäuser / Passivhäuser \\
	\psx Abfallkonzept vorhanden \\
	\psx Blockheizkraftwerke \\
	\psx Regenwasserkonzept \\
	\psx Windräder vorhanden 
	\\
	\textbf{soziale Mischung}: \\
	\psx 4-stöckige Häuser entlang der S-Bahn $\to$ Wohnung \\
	\psx Förderung vorhanden \\
	$\to$ \textbf{ABER}: Daten zu Bev. (Alter, Einkommen, Bildung) fehlen. \\
	\psx Bezirkssportanlage sorgt für Begegnung
	\\[0.5cm]
	\Large
	Ausfomulierung
	\hrule
	\vspace{0.25cm}
	\large
	\noindent
	Das Model der nachhaltigen Stadt umfasst vielen verschiedene Aspekte.
	Wichtig zu erwähnen ist hierbei vor allem die hohe bauliche Dichte, die sich allerdings nicht durch Großwohnsiedlungen auszeichnet.
	Ein weiterer Zentraler Aspekt der nachhaltigen Stadt sind die kurzen Wege zu Einrichtungen des alltäglichen Lebens, wie zum Beispiel der Supermarkt oder die Schule.
	Durch diese kurzen Wege wird ebenfalls versucht den motorisierten Individualverkehr zu reduzieren und den ÖPNV auszubauen.
	Alternativ zum Ausbau des ÖPNV können auch andere Verkehrsmittel, wie beispielsweise das Fahrrad gestärkt werden.
	Nachhaltige Städte zeichnen sich durch ihre vielen Parks und Innenhöfe aus, die gemeinschaftlich Nutzbar sind.
	Die Mischung sozialer Schichten wird durch den Bau von unterschiedlichen Wohnungstypen bzw. Haustypen, aber auch durch geringere Mieten für weniger wohlhabende Bürger erreicht.
	Hierbei spielt die staatliche Förderung dieser Menschen eine große Rolle. \\
	Hannover Kronsberg ist dicht mit bis zu 4-geschossigen Häusern bebaut.
	Allerdings gibt es im Osten des Viertels ebenfalls Reihenhäuser, was gegen das Modell der nachhaltigen Stadt spricht.
	Die Bevölkerungsdichte ist mit 12.500 Menschen pro Quadratkilometer extrem hoch, wobei insgesamt 15.000 Menschen in insgesamt 6000 Wohnungen das Viertel bewohnen.
	Auch Einrichtungen des alltäglichen Lebens sind vorhanden.
	Es gibt mehrere Kindergärten, Grundschulen und sogar eine Gesamtschule.
	Die Anbindung an die Innenstadt ist durch den ÖPNV gewährleistet
	Hierbei hat man insgesamt 20 Minuten Fahrzeit und kann alle 10 Minuten abfahren.
	Um die kurzen Wege innerhalb des Viertels zu gewährleisten, ist jede Wohnung maximal 500m von einer Haltestelle entfernt.
	Auch Arbeitsplätze sind im Viertel vorhanden, wobei die Bewohner natürlich nicht zwingend in ihrem eigenen Viertel arbeiten müssen.
	Ebenfalls ist in Kronsberg die Versorgung durch Supermärkte möglich.
	Die Reduzierung des motorisierten Individualverkehrs wird durch eine auf Fußgänger und Fahrradfahrer ausgelegte Infrastruktur erreicht.
	Des Weiteren gibt es mehrere Carsharing Angebote und eine gute Anbindung a die Innenstadt mit ÖPNV.
	Die Innenstadt ist alle 10 Minuten innerhalb von 20 Minuten erreichbar und jede Wohnung ist maximal 500 Meter von der nächsten Haltestelle entfernt.
	Parks und Innenhöfe sind durch die Art der Bebauung ebenfalls ausreichend vorhanden.
	Durch das Sportzentrum ist auch die Freizeitbeschäftigung gewährleistet.
	Da es in Kronsberg viele verschiedene Wohnungs- und Haustypen, wie zum Beispiel mehrstöckige Wohnhäuser und Einfamilienhäuser, gibt ist für eine soziale Mischung gesorgt. \\
	Insgesamt entspricht Hannover Kronsberg fast in allen Punkten dem Modell der nachhaltigen Stadtentwicklung, lediglich die Reihenhäuser sind nicht modellgerecht.
\end{document}
\documentclass[12pt,a4paper]{report}
\usepackage[T1]{fontenc}
\usepackage[utf8]{inputenc}
\usepackage{charter}
\usepackage{ngerman}
\usepackage[left=2cm,right=2cm,top=2cm,bottom=2cm]{geometry}
\usepackage{tabularx}

\renewcommand\thesection{\arabic{section}.} 

\begin{document}
	\section{S. 220 Nr. 3}
	\begin{tabularx}{\textwidth}{|X|X|}
		\hline
		\textbf{Vorteile} & \textbf{Nachteile} \\
		\hline
		\begin{itemize}
			\item Schaffung eines starken wirtschaftlichen Zentrums
			\begin{itemize}
				\item schafft Arbeitsplätze
				\item Lagevorteile z.B. Verkehrsdrehscheibe
				\item hohe Nachfrage z.B. unter den Unternehmen
				\item Kommunikationsvorteile
				\item[$\to$] teilweise stärkere Kooperation möglich
			\end{itemize}
			\item Qualität und Quantität der Einrichtungen ist besser
			\begin{itemize}
				\item gut für die Region
				\item gut für das Land selber
			\end{itemize}
			\item wirtschaftliche, politische, kulturelle und infrastrukturelle Überlegenheit
		\end{itemize}
		&
		\begin{itemize}
			\item starkes Gefälle zwischen armen und reichen Regionen
			\begin{itemize}
				\item[$\to$] Polarisation
				\item[$\to$] Migration wird verstärkt
			\end{itemize}
			\item fehlende Entwicklung in den Gebieten
			\begin{itemize}
				\item[$\to$] verstärkt sich selber
			\end{itemize}
			\item Abhängigkeit eines ganzen Landes von einer Region
			\item Wohnraum in Primatstädten wird knapp und teuer $\to$ Verdrängung
			\begin{itemize}
				\item Gefahr der Verschlechterung der Wohnqualität
			\end{itemize}
		\end{itemize}
		\\
		\hline
	\end{tabularx}
\end{document}
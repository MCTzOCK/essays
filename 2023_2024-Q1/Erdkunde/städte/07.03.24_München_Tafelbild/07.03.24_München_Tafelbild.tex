\documentclass[12pt,a3paper]{report}
\usepackage[T1]{fontenc}
\usepackage[utf8]{inputenc}
\usepackage{charter}
\usepackage{ngerman}
\usepackage[left=2cm,right=2cm,top=2cm,bottom=2cm]{geometry}
\usepackage{xcolor}
\usepackage{eurosym}

\begin{document}
	\noindent
	\huge Wie haben sich Münchens Immobilienpreise entwickelt und was beeinflusst die Preise?
	\Large
	\noindent
	\begin{itemize}
		\item 2017-2018 Anstieg bei Ein- und Zweifamilienhäusern um 5,9 \%  auf durchschnittlich 6780 \euro/$m^2$
		\item 2017-2018 Anstieg der Mietpreise um 3,2\% auf durchschnittlich 18,60 \euro/$m^2$
		\begin{itemize}
			\item Dortmund: 4,6 \% auf 6,90 \euro
			\item Köln: 4,3 \% auf 11,70 \euro
		\end{itemize}
		\item[$\Rightarrow$] prozentual geringerer Anstieg in München.
		\textbf{ABER} Preise sind schon extrem hoch und absolut höherer Anstieg. \\\\
		\color{red}\textbf{WARUM?}\color{black}
		\begin{itemize}
			\item bis 2040 kommen 340.000 Menschen neu nach München 
			\begin{itemize}
				\item[$\Rightarrow$] 1,85 Mio. Einwohner (+20\% Bev.-Wachstum)
				\item[$\Rightarrow$] theoretisch kommt ganz Augsburgs \textbf{ODER} Bochum \textbf{ODER} Bielefeld bis 2040 hinzu
			\end{itemize}
			\item München ist extrem attraktiv für viele Menschen (Standortfaktoren)
			\begin{itemize}
				\item Wirtschaft $\to$ 7 DAX-Konzerne (2018)
				\item Wirtschaft $\to$ Deutschland- und Europazentrale von internationalen Firmen
				\item Wirtschaft $\to$ Hidden Champions
				\item Wirtschaft $\to$ starke Startup-Szene
				\item hohe Lebensqualität
				\begin{itemize}
					\item Englischer Garten
					\item Olympiagelände
					\item Iserterassen
					\item Alpen, Seen, etc. im Umfeld + Österreich
					\item[$\Rightarrow$] Hoher Freizeitwert
				\end{itemize}
			\end{itemize}
			\item extremst geringe Leerstandsquote
			\begin{itemize}
				\item 2018: 0.2 \%
				\item 2023: 0.34 \%
				\item vgl. Leerstand Bochum: 2-3\%
				\item München 2023: insgesamt 828.000 Wohnungen
				\item 2815 Wohnungen insgesamt etwa frei
				\item[$\Rightarrow$] 121 Menschen müssten sich \textbf{eine} Wohnung teilen
				\item[$\Rightarrow$] München \textbf{muss} Wohnungen schaffen
				\begin{itemize}
					\item München muss sich stärker verdichten
				\end{itemize}
			\end{itemize} 
			\item[$\Rightarrow$] \textbf{ABER} München ist bereits sehr dicht bebaut
			\begin{itemize}
				\item[$\Rightarrow$] höhere Häuser
				\item[$\Rightarrow$] Lebensqualität leider aber ggf. durch Verdichtung
				\item[$\Rightarrow$] München muss theoretisch in die Fläche expandieren 
				\begin{itemize}
					\item München muss Flächen fürs Wohnen, für die Wirtschaft schaffen
					\item[$\Rightarrow$] neue Flächen müssen aber attraktiv bleiben
					\item[$\Rightarrow$] nachhaltige Leitlinie
				\end{itemize}
			\end{itemize}
		\end{itemize}
	\end{itemize}
\end{document}
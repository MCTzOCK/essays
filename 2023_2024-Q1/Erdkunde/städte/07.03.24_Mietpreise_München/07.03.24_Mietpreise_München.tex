\documentclass[12pt,a4paper]{report}
\usepackage[T1]{fontenc}
\usepackage[utf8]{inputenc}
\usepackage{charter}
\usepackage{ngerman}
\usepackage[left=2cm,right=2cm,top=2cm,bottom=2cm]{geometry}
\usepackage{eurosym}

\begin{document}
	\noindent
	\Large
	\textbf{Mietpreise in München - EVA}
	\\[1cm]
	\large
	In München ist Wohnungsmangel vorherrschend.
	Bis 2040 wurde ein Bevölkerungswachstum von 20 \% im Vergleich zum Jahre 2017 prognostiziert.
	Laut Peter Finkbeiner gibt es bei Angebot und Nachfrage der Wohnungen in München eine Differenz von 42 \%.
	Es gibt in Zukunft keine wirklich realistische Möglichkeit für die Schaffung neuer Wohnungen, da bereits fast jede nutzbare Fläche verwendet wird.
	Nur 0,2 \% der Wohnungen stehen leer.
	Im Schnitt müssen Münchener rund 31 \% ihres Haushaltseinkommens nur für Miete ausgeben.
	Auf die gesamte BRD gesehen sind das durchschnittlich 11 \% mehr.
	Das Stadtbild wird sich in Zukunft mit großer Wahrscheinlichkeit dahin gehend verändern, als dass enger zusammengebaut werden muss.
	\\
	In München werden pro Quadratmeter durchschnittlich 11,69 \euro.
	Damit ist München im Vergleich zu anderen Städten, wie Hattingen, Dortmund oder Essen, deutlich teurer.
	In Hattingen beispielsweise kostet der Quadratmeter im Durchschnitt nur etwa 6,04 \euro, womit Hattingen nur noch etwa halb so teuer ist.
	In größeren Städten, wie Essen und Bochum ist der Quadratmeterpreis zwar höher, als in Hattingen, jedoch mit 8,63 \euro\ bzw. 7,40 \euro\ deutlich günstiger.
	Besonders teuer sind Wohnungen im Münchener Zentrum, wo ein Quadratmeter bis zu 25 \euro\ kosten kann.
	\paragraph{Segregation} bezeichnet die räumliche, soziale oder rechtliche Trennung von Menschen oder Gruppen aufgrund von Merkmalen wie Rasse, Ethnie, Religion, Geschlecht oder Klasse.
	Segregation kann durch Gesetze, Gewalt, Diskriminierung oder freiwillige Entscheidungen entstehen oder aufrechterhalten werden. Segregation kann negative Folgen für die benachteiligten Gruppen haben, wie z.B. geringere Bildungs- und Gesundheitschancen, weniger politische Teilhabe oder höhere Kriminalitätsraten.
	Segregation kann in verschiedenen Bereichen auftreten, wie z.B. Wohnen, Bildung, Arbeit, Verkehr, Freizeit oder Religion.
\end{document}
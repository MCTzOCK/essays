\documentclass[12pt,a4paper]{report}
\usepackage[T1]{fontenc}
\usepackage[utf8]{inputenc}
\usepackage{charter}
\usepackage{ngerman}
\usepackage{custompkg}
\usepackage[left=2cm,right=2cm,top=2cm,bottom=2cm]{geometry}
\usepackage{tikz}
\usepackage{tcolorbox}
\tcbuselibrary{skins,breakable}
\usetikzlibrary{shadings,shadows}

\newenvironment{gblock}[1]{
    \tcolorbox[beamer,
        noparskip,breakable,
        colback=green!50!,
        colbacklower=green!75!green,
        title=#1]}
{\endtcolorbox}

\begin{document}
	\bsremovechaptertitle
	\chapter{Chartar von Athen}
	\begin{gblock}{Leitbild nach Le Corbusiers (funktionale Stadt)}
	Die Großstadt soll in einzelne Funktionszonen aufgegliedert werden.
	Das Zentrum soll dem öffentlichen Leben vorbehalten sein (Handel, Verwaltung und Kultur). Die Einwohner werden konzentriert, um mehr Platz für Grünfläche zu schaffen.
	\end{gblock}
	\vspace{1cm}
	\begin{gblock}{Die gegliederte und autogerechte Stadt}
		Der Fokus wird auf den Bau von günstigen Sozialwohnungen gelegt.
		Die Stadt wurde in einzelne Wohnungs- und Nutzungsbereiche gegliedert und durch Grünzüge aufgelockert.
		Großer Wert wurde ebenfalls auf den Verkehr gelegt, sodass die Stadtplanung meist auf eine Verkehrsplanung hinauslief.
		Sämtliche Planungsmaßnahmen waren in erster Linie für den ungehinderten Verkehrsfluss des Autos.
		Das Straßennetz sollte ideal für den Individualverkehr ausgelegt sein.
		Durch den stetig wachsenden Verkehr, kamen die Straßennetze allerdings schnell an ihre Kapazitätsgrenzen, wodurch sich schlussendlich von diesem Konzept abgewandt wurde.
	\end{gblock}
	\vspace{1cm}
	\begin{gblock}{Urbanität durch Dichte}
		Am Stadtrand entstanden Großwohnsiedlungen, in welchen oft Freizeit- und Versorgungseinrichtungen vorzufinden waren.
		Durch den Massenwohnungsbau konnte die Wohnungsnot schnell überwunden werden.
		Die Hochhäuser wurden mit vorgefertigten Betonteilen gebaut, wodurch Kosten gespart werden konnten.
		Die Bauweise gefiel allerdings vielen Bewohnern nicht wirklich.
	\end{gblock}



\end{document}
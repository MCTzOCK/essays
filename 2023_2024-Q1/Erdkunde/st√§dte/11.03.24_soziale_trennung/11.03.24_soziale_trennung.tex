\documentclass[12pt,a4paper]{report}
\usepackage[T1]{fontenc}
\usepackage[utf8]{inputenc}
\usepackage{charter}
\usepackage{ngerman}
\usepackage[left=2cm,right=2cm,top=2cm,bottom=2cm]{geometry}

\begin{document}
	\noindent
	\Large
	Sozialräumliche Gliederung in München
	\large
	\begin{itemize}
		\item Besonders in den zentralen Stadtteilen von München ist die Einwohnerdichte deutlich höher, als in den außen liegenden Teilen.
		Diese liegt meist zwischen 9.000 und 12.000 oder teilweise sogar bei über 12.000 Einwohner / km$^2$.
		In den äußeren Stadtteilen kann die Einwohnerdichte auch deutlich unter 3000 Einwohner / km$^2$ fallen.
		\item Der Anteil an Ausländern ist besonders in den südwestlichen Vierteln mit mehr als 28 \% deutlich höher, als in den nördlichen und östlichen Stadtteilen Münchens.
		\item Der Kaufkraftindex liegt in allen Stadtteilen, mit Ausnahme von Feldmoching-Hasenbergl, Schwanthalerhöhe und Milbertshofen am Hart, über 90.
		In einigen Teilen, wie zum Beispiel Bogenhausen, liegt der Kaufkraftindex sogar über 105.
		\item Im Norden Münchens ist der Anteil der ALG II Empfänger im Vergleich zu den anderen Teilen besonders hoch.
		Im Zentrum und im Osten ist der Anteil mit unter 4,0 \% am geringsten.
		Der Westen Münchens liegt im Durchschnitt zwischen 4,1 \% und 6,2 \%.
		\item Der Anteil der Einpersonenprivathaushalte ist lediglich im Zentrum Münchens extrem hoch.
		In den umliegenden Bezirken ist dieser Anteil jeweils relativ gering.
	\end{itemize}
	\newpage
	\noindent
	\Large Segregationsprozesse in München
	\large
	\paragraph{Ausgangslage:}
	\begin{itemize}
		\item Immobilien sind sehr teuer
		\item Zuwachs der Bevölkerung bis 2040 um 20 \%
	\end{itemize}
	\paragraph{M8:}
	\begin{enumerate}
		\item Familien mit Kindern ziehen aus München weg $\to$ Kaum Wohnraum für Familien vorhanden $\to$ falls doch, dann extrem teuer.
		\item teilweise Studenten + junge Arbeitnehmer ohne Kinder $\to$ größtenteils Akademiker $\to$ verdienen gut und benötigen wenig Platz
		\item Familien mit Kindern ziehen (wieder) weg $\to$ auch Akademiker
		\item Menschen in Führungspositionen, die schon viel Berufserfahrung haben
		\item Rentner müssen München verlassen
	\end{enumerate}
\end{document}
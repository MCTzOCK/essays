\documentclass[12pt]{report}
\usepackage[utf8]{inputenc}
\usepackage[T1]{fontenc}
\usepackage{ngerman}
\usepackage[left=0cm,right=1cm,top=2cm,bottom=2cm,landscape]{geometry}
\usepackage{tabularx}
\usepackage{charter}
\usepackage{custompkg}

\begin{document}
	\bslinespacing{1.25}
	\thispagestyle{empty}
	\begin{tabularx}{\textwidth}{|X|X|X|X|X|}
		\hline
		\textbf{Typisierungskriterien} & \textbf{Grundriss} & \textbf{Siedlungsmittelpunkt} & \textbf{Straßennetz / Verkehrsnetz} & \textbf{Sonstige Merkmale} \\
		\hline
		\textbf{Römische Stadt} & Schachbrettartig; Schutz durch Ummauerung; sozialräumliche Gliederung&meist Forum (großer Platz)& Wasserversorgung und Abwasserentsorgungsleitungen; Einrichtungen des öffentlichen Lebens (Forum, Thermen); Gerichtswesen; Verwaltung; Handlung; Fernstraßennetz & Früher meist Militärstützpunkte \\
		\hline
		\textbf{Mittelalterliche Stadt (8.-15. Jh.)} &unregelmäßiges Muster; Stadtmauern; Burg/Festung für Verteigungszwecke;&Marktplatz umgeben von Einrichtungen (Rathaus, Kirche, Geschäfte)&unregelmäßig; dynamisch; an natürliche Gegebenheiten (Flüsse, Hügel) angepasst.& \\
		\hline
		\textbf{Absolutismus / Residenzstadt (16.-18. Jh.)} &Zentraler Palastkomplex umgeben von repräsentativen Gärten und Parks; Alleen und symmetrische Strukturen&Palastkomplex&Breite Alleen zum Palast hin; umliegende kleiner Straßen nach einem festen Muster angelegt& \\
		\hline
		\textbf{Industrialisierung / Industriestadt (19. Jh.)} &Schwerpunkt auf Fabriken und Industrieanlagen; angeordnet an Flüssen oder Eisenbahnstrecken&Industrielle Kerngebiete&Pragmatische Herangehensweise für den Transport von Resourcen; rechtwinkliges Muster& \\
		\hline
		
	\end{tabularx}
\end{document}
\documentclass[12pt, a4paper]{report}
\usepackage[utf8]{inputenc}
\usepackage[T1]{fontenc}
\usepackage{german}
\usepackage[left=2cm,right=2cm,top=2cm,bottom=2cm]{geometry}
\usepackage{charter}
\usepackage{custompkg}
\usepackage{eurosym}

\begin{document}
	\bslinespacing{1.5}
	\title{Probeklausur Nr. 2}
	\author{Ben Siebert}
	\date{\today}
	\maketitle
	
	\paragraph{Aufgabe 1} \mbox{} \\
	Die Niederlande liegen auf der Nordhalbkugel auf dem europäischen Kontinent.
	Sie sind im Nordwesten des Kontinents verordnet.
	Im Norden grenzen die Niederlande an die Nordsee.
	Die Nachbarländer sind Deutschland im Osten und Belgien im Süden.
	Die Niederlande liegen bei etwa 52,63708° N, 5,67359° O.
	Die Hauptstadt der Niederlande ist Amsterdam, welche im Westen des Landes verordnet ist.
	Das Klima der Niederlande ist typisch für Mittel- bzw. Westeuropa.
	Es gibt relativ kalte Wintermonate und relative warme Sommermonate.
	Der Niederschlag ist gleichmäßig über das Jahr verteilt und beträgt insgesamt 815,5 mm.
	Die Durchschnittliche Temperatur liegt bei 10°C. \\
	Der Boden der Niederlande kann in zwei große Regionen unterteilt werden.
	Zum einen Marsch Boden, welcher im Westen bis Nordosten verordnet ist und zum anderen Podsol Böden, welche im Osten des Landes vertreten sind.
	Marsch muss vor der landwirtschaftlichen Nutzung trockengelegt werden, ist dafür aber auch ertragreich.
	Podsol hingegen muss nicht besonders behandelt werden, bevor er landwirtschaftlich genutzt werden kann, ist dafür allerdings ertragsarm (M2). \\
	Die Niederlande haben eine Bevölkerung von 16.62 Mio. Menschen.
	Hierbei liegt die Bevölkerungsdichte bei 400 Menschen pro Quadratkilometer.
	Dies sind fast 200 Menschen mehr, als in Deutschland, wo sie bei 229 liegt.
	Insgesamt wird in den Niederlanden eine Fläche von 1,9 Mio. ha für die Landwirtschaft verwendet.
	Das ist nur ein Bruchteil, der in Deutschland verwendeten Fläche, wo es über 16 Mio. ha sind.
	Allerdings ist der Bodenpreis für Agrarland in den Niederlanden deutlich höher als in Deutschland.
	Ein ha Land kostet 48.000 \euro, während es in Deutschland nur 11.000 \euro\ sind.
	Ebenfalls ist die landwirtschaftliche Fläche pro Einwohner mit 0.11ha deutlich geringer als in Deutschland mit 0.2ha.
	Die Erwerbstätigen im Agrarsektor ist mit 2.8\%\ im Vergleich zu Deutschland doppelt so hoch. \\
	Es gibt insgesamt über 72.000 landwirtschaftliche Betriebe in den Niederlanden, welche jeweils etwa 121.000 \euro\ Burttowertschöpfung haben.
	Pro Jahr werden 72,2 Mrd. \euro\ an Agrargütern exportiert, wovon ein Großteil (57,2 Mrd. \euro) nach Europa geliefert wird (M3).
	
	\newpage
	\paragraph{Aufgabe 2} \mbox{} \\
	Bei der Spezialisierung fokussiert sich ein Betrieb auf eine bestimmte Tätigkeit, wie zum Beispiel die Erzeugung von Rindfleisch.
	Hierbei können viele Vorgänge vereinfacht werden, da diese nur noch für einen bestimmten Rohstoff funktionieren müssen und nicht auf mehrere Aspekte anwendbar sein müssen.
	Bei der Intensivierung wird durch hohen Einsatz von Technik und anderen Mitteln auf einer begrenzten Fläche einen möglichst hohen Ertrag zu erzielen. \\
	Die Landwirtschaft in den Niederlanden ist so produktiv, da sie sehr auf den sog. \dq Gewächsthausanbau\dq\ ausgelegt ist.
	Hierbei wird der $CO_2$-Anteil in der Luft künstlich auf die optimalen 800 ppm angehoben, wodurch die Pflanzen bis zu 40 \%\ schneller wachsen (M8).
	Die Hauptproduktionsstruktur in den Niederlanden ist der Gewächshausanbu.
	Mit diesen Gewächshäusern kommen meist Blumen-, Gemüse- und Obstversteigerungen, bei welchen die produzierten Güter meistbietend versteigert werden.
	Diese Einrichtungen sind zum Großteil in sogenannten Greenports verordnet, welche eine wirtschaftsnahe Infrastruktur bieten.
	Ebenfalls sind diese Bereiche sehr stark technisiert und bieten meist Platz für Forschung und Ausbildung.
	Bei jedem Greenport wird außerdem auf eine gute Transportanbindung geachtet.
	Die Landwirtschaft in den Niederlanden ist also deutlich organisierter, als in anderen Ländern und erhält große Förderung vom Staat, welcher ebenfalls für die Infrastruktur vor Ort sorgen.
	Klassische Freilandproduktion ist nicht so verbreitet (M4).
	Die Region Westland ist alleine für einen großen Teil der gesamten Produktionsfläche der Niederlande verantwortlich.
	Von den insg. etwa 10.000ha, befinden sich alleine 2500ha in Westland.
	\newpage
	\paragraph{Aufgabe 3}
	
\end{document}
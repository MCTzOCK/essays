\documentclass[11pt,a4paper]{report}

\usepackage[T1]{fontenc}
\usepackage[utf8]{inputenc}
\usepackage{ngerman}
\usepackage{custompkg}
\usepackage{charter}
\begin{document}

	\bslinespacing{1.5}
	\bsremovechaptertitle
	
	\title{Fleischproduktion}
	\author{Ben Siebert}
	\date{20. November 2023}
	
	\maketitle
	
	\chapter{Aufgabe 1a}
	
	\paragraph{Im} Allgemeinen ist festzustellen, dass die gesamte Fleischproduktion in den letzten 25 Jahren gestiegen ist.
	Die größte Steigung ist bei Geflügel zu erkennen.
	Diese lag 1995 noch bei etwa 60 Mio. t, stieg allerdings bis 2020 relativ linear auf ca. 120 Mio. t an.
	Auch die Entwicklung der Produktion von Schweinefleisch ist, wenn auch nicht so stark, von 80 Mio. t auf ebenfalls etwa 120 Mio. t gestiegen.
	Bei Rindfleisch ist die Entwicklung nicht besonders stark.
	Diese verlief seit 1995 relativ flach und stieg lediglich von etwa 60 Mio. t auf etwa 80 Mio. t an. (M3)
	\paragraph{Global} gesehen ist zu erkennen, dass die gesamte Fleischproduktion ebenfalls stark angestiegen ist.
	Im Bezug auf Rind-, Schweine- und Geflügelfleisch wurden 1962 nur 64.465.000 t produziert.
	1990 hingegen waren es bereits 164.344.000 t.
	Bis zum Jahr 2012 stieg die Produktion ebenfalls stark an und lag schließlich bei 278.047.000 t. (M2)
	Die Rindfleisch Produktion ist global zwischen 1962 und 2012 von 29.203.000 t auf 75.422.000 t angestiegen, hat sich also insgesamt mehr als verdoppelt. Auch bei der Produktion von Schweinefleisch sieht es nicht viel anders aus. Diese ist von 1962 bis 2012 von 26.056.000 t auf 109.122.000 t angestiegen.
	Die Geflügelfleisch Produktion ist von 1962 bis 2021 um 1.374 \% gestiegen. (von 9.206.000 t auf 126.502.000 t)
	
	\newpage
	
	\chapter{Aufgabe 2}
	Es ist zu erkennen, dass die Hühnerfleischproduktion seit 1962 extrem angestiegen ist.
	Besonders in Asien ist die Erzeugungsmenge extrem gestiegen.
	So lag diese 1962 noch bei etwa 2 Mio. t und stieg schon 1992 auf etwa 10 Mio. t an.
	Bis 2012 hat die sich Produktion in Asien etwa verdreifacht und lag 2012 bei etwa 30 Mio. t.
	In Europa hingegen stieg die Produktion von 1962 bis 1990 zwar auch erst an, sank allerdings anschließend bis etwa 2000.
	Bis 2012 stieg die Erzeugungsmenge allerdings wieder auf 15 Mio. t an.
	Generell ist die Produktion allerdings auf der ganzen Welt extrem angestiegen (M1).
	Hühner brauchen im Vergleich zu anderen Tieren, wie Schweinen und Rindern sehr wenig Futtermenge für die Erzeugung von einem kg Fleisch.
	Hühner benötigen 1,6 kg, während Schweine bereits 2,8 kg benötigen.
	Rinder benötigen mit 6-8 kg am meisten Futter für die Erzeugung für einen Kilogramm Fleisch.
	Der größte Produzent von Futtermengen sind seit 1962 die Vereinigten Staaten von Amerika.
	Diese haben 1962 noch einen Anteil von 33,3 \% gehabt, welcher allerdings bis 2012 auf nur noch 18,4 \% gesunken ist.
	1962 lagen sie 23 \% vor dem zweitgrößten Produzenten der UdSSR.
	2012 sind es nur noch 5 \% Unterschied zum zweitgrößten Produzenten China.
\end{document}
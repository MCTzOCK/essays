\documentclass[a4paper, 12pt]{report}

\usepackage{ngerman}
\usepackage[T1]{fontenc}
\usepackage[utf8]{inputenc}
\usepackage{bookman}
\usepackage{custompkg}
\usepackage[left=2cm, top=2cm, right=2cm, bottom=2cm]{geometry}

\begin{document}
	\bslinespacing{1.5}
	\bsremovechaptertitle
	\chapter{Lokalisierung}
	
	Das Oldenburger Münsterland befindet sich auf der Nordhalbkugel auf dem Kontinent Europa.
	Es liegt in Zentraleuropa in Deutschland.
	Nachbarländer vom Deutschland sind Dänemark im Norden, Frankreich, die Niederlande, Belgien und Luxemburg im Westen, Polen, Tschechien im Osten und die Schweiz und Österreich im Süden.
	Deutschland besitzt zwei Meere im Norden, zum einen die Nordsee, welche an den Nordwesten des Landes grenzt und die Ostsee, welche an den Nordosten grenzt.
	Es erstreckt sich von 47 bis 55 Grad nördlicher Breite und von etwa 6 bis 15 Grad östlicher Länge.
	Das Oldenburger Münsterland liegt im Bundesland Niedersachsen, welches im Norden Deutschlands auffindbar ist.
	Es befindet sich südlich von Oldenburg und nördlich von Osnabrück und erstreckt sich von etwa 52.457563 Grad nördliche Breite und östlicher Länge 8.099087 bis 53.173256 Grad nördlicher Breite und 7.742539 Grad östlicher Länge.
	\\\\
	In der Region stehen durch die geringe Bevölkerungsdichte und das vorliegende Flachland große Flächen für das Agrobusiness zu verfügung.
	Obwohl es einige Städte im Oldenburger Münsterland gibt, ist die ländliche Fläche deutlich größer und nicht durch Großstädte begrenzt.
	Ebenfalls ist der Anbau von Futtermittel in der Region großflächig möglich.
	Die Autobahnanbindungen ermöglichen Transport in den Norden und Süden Deutschlands.
	Dies wird vor allem durch die A1 bzw. die A29 ermöglicht.
	Der Anschluss an die Häfen in Bremenhaven bzw. Hamburg ermöglicht den Warentransport Übersee.
	Ebenfalls ist der Lufttransport durch den Flughafen in Hamburg möglich.
	\\\\
	Es gibt im Oldenburger Münsterland viele landwirtschaftlich-orientierte Unternehmen, unter denen sowohl Viehzüchter, als auch Futterproduzenten sind. 
	Hierbei wäre eine Kooperation mit den Produzenten des Futtermittels durchaus denkbar.
	Ebenfalls sind einige Forschungseinrichtungen in der Region ansässig, sodass auch hier eine Kooperation im Bezug auf die Technisierung des Betriebs möglich ist.
	Auch die ärztliche Versorgung ist in der Region vorhanden.
	
	
	
	
\end{document}
\documentclass[12pt,a4paper]{report}
\usepackage[T1]{fontenc}
\usepackage[utf8]{inputenc}
\usepackage{charter}
\usepackage{ngerman}
\usepackage[left=2cm,right=2cm,top=2cm,bottom=2cm]{geometry}

\renewcommand\thesection{\arabic{section}.} 

\begin{document}
	\section{Die Auferstehung Jesu - ein historisches Ereignis?}
	\begin{itemize}
		\item[$\to$] historische \dq Wahrheit\dq
		\item[$\to$] rekonstruiert aus Quellen und Überresten
		\item[$\to$] Ergebnis muss intersubjektiv akzeptabel sein
	\end{itemize}
	Darauf folgt, dass die Auferstehung Jesu keine historische, sondern eine Glaubensaussage ist.
	Allerdings ist historisch verbürgt:
	\begin{itemize}
		\item Jesu Anhänger verkünden nach seinem Tod, er sei ihnen begegnet.
	\end{itemize}
\end{document}
\documentclass[11pt, a4paper]{report}
\usepackage{ngerman}
\usepackage[T1]{fontenc}
\usepackage[utf8]{inputenc}
\usepackage{bookman}
\usepackage[left=2cm,right=2cm,top=2cm,bottom=2cm]{geometry}
\usepackage{custompkg}

\begin{document}
	\bsremovechaptertitle
	\bslinespacing{1.5}
	\title{Biographien und Aufgaben}
	\date{\today}
	\author{Ben Siebert \and Alex D\"oppers}
	\maketitle
	\tableofcontents
	\chapter{Hans Jonas}
	\paragraph{Generelles:}
	\begin{itemize}
		\item Geboren am 10.05.1903 in M\"onchengladbach
		\item Gestorben am 05.02.1993 in New Rochelle im Bundesstaat New York
		\item Einmal verheiratet (mit Lore Jonas; 1943-1993)
		\item Er war ein deutscher Philosoph
		\item Lebte den Großteil seines Lebens in den Vereinigten Staaten von Amerika
	\end{itemize}
	\paragraph{Werke (Auswahl):}
	\begin{itemize}
		\item \dq Das Prinzip Verantwortung\dq - 1979
		\item \dq The gnostic religion\dq - 1958
		\item \dq Gnosis und spa\"atantiker Geist\dq - 1988
		\item \dq The phenomenon of life: toward a philosophical biology\dq - 1966
	\end{itemize}
	\chapter{Dietrich Bonhoeffer}
	\paragraph{Generelles:}
	\begin{itemize}
		\item Geboren am 04.02.1906 in Breslau
		\item Kam aus einer wohlhabenden Familie
		\item Studierte Theologie in T\"ubingen und Berlin und promovierte in Berlin \"uber das Thema Gemeinschaft
		\item entschiedener Gegner des Nationalsozialismus
		\item Gestorben am 09.04.1944 in Flossenb\"urg auf Grund einer Hinrichtung
	\end{itemize}
	\paragraph{Werke (Auswahl):}
	\begin{itemize}
		\item \dq Nachfolge\dq - 1937 - ethische Beleuchtung der Prinzipien der Nachfolgen Christi
		\item \dq Ethik\dq - 1949 - moralische Fragen und Verantwortung
	\end{itemize}
	\chapter{Emile Chartier}
	\paragraph{Generelles:}
	\begin{itemize}
		\item Geboren am 3. M\"arz 1868 in Mortagne-au-Perche, Frankreich.
		\item Gestorben am 2. Juni 1951.
		\item Er war ein franz\"osischer Philosoph, Schriftsteller und Lehrer, der eine bedeutende Rolle im Bildungswesen spielte
		\item Alain war ein entschiedener Gegner des Krieges und setzte sich für den Pazifismus ein, insbesondere während des Ersten Weltkriegs.
		\item Chartier hatte einen Einfluss auf die Reformen im französischen Bildungssystem und betonte die Notwendigkeit einer demokratischen Erziehung.
	\end{itemize}
	\chapter{Aufgabe 2)}
	Die Aussage begründet die Aussage des schwachen Gottes, indem sie das Gottesbild auf Jesus Christus bezieht, dessen Lebensweg in der Bibel geschildert wird.
	Sowohl die Geburt (der Anfang), als auch der Tod (das Ende) war von extremer Schwäche gepr\"agt, da er zwischen Dieben und Verbrechern hingerichtet wurde, bzw. zwischen Tieren geboren worden ist.
	\chapter{Aufgabe 3}
	Gott zeigt sich den Menschen als schwach, damit sie sich mit ihm verbunden fühlen und ihn nicht als übergeordnete Instanz wahrnehmen.
	Hierdurch möchte Gott einen Kontakt auf Augenhöhe herstellen.
	\chapter{Aufgabe 4}
	Sehr geehrte Herr Comte-Sponville,
	\\
	wir schreiben Ihnen, da wir uns über ihre Passage \dq dessen Anfang und Ende Extreme der Schwäche sind: Krippe und Kreuzweg\dq\  äußern möchten.
	Zunächst war die Hinrichtung Jesu nicht sein Ende, da er schließlich den Tod besiegte und wieder auferstanden ist.
	Des Weiteren beweist ein Tod zwischen Dieben viel mehr Stärke als Schwäche, denn immerhin schaffte er es, sich hinrichten zu lassen und starb schließlich symbolisch für alle Menschen, die zu seiner Zeit lebten und jemals leben werden.	Dies gilt ebenfalls f\"ur seine Geburt, denn jemand, der in schlechten Verh\"altnissen aufw\"achst und es trotzdem zu etwas bringt, gilt in unserer Gesellschaft als sehr stark.
	\\
	Mit freundlichen Grüßen,
	\\
	Alex und Ben
\end{document}
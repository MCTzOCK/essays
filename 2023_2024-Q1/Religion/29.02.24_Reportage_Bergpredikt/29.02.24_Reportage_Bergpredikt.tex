\documentclass[12pt,a4paper,twocolumn]{report}
\usepackage[T1]{fontenc}
\usepackage[utf8]{inputenc}
\usepackage{charter}
\usepackage{ngerman} \usepackage[left=2cm,right=2cm,top=2cm,bottom=2cm]{geometry}

\begin{document}
	\noindent
	\Large
	\textbf{Reportage Bergpredikt}
	\\[0.2cm]
	\large
	\noindent
	Ich bin einer von vielen, die heute auf diesen Berg gekommen sind, um Jesus von Nazareth zu hören.
	Er ist ein Wanderprediger, der in ganz Galiläa bekannt geworden ist durch seine Heilungen und Wunder.
	Er spricht von einem neuen Reich Gottes, das nahe ist, und fordert die Menschen auf, sich zu bekehren und zu glauben.
	Ich habe mich unter die Menge gemischt, die ihm auf den Berg gefolgt ist.
	Er setzt sich auf einen erhöhten Platz, und seine  engsten Jünger treten zu ihm.
	Dann beginnt er zu reden.
	Seine Stimme ist klar und ruhig, aber voller Autorität.
	% Er spricht nicht wie die Schriftgelehrten und Pharisäer, die das Gesetz auslegen, sondern wie einer, der selbst das Gesetz ist.
	Jesus beginnt mit einer Reihe von Seligpreisungen, die mich überraschen.
	Laut ihm sind alle selig, die arm sind, die trauern, die hungern, die sich nach Gerechtigkeit sehnen, die barmherzig sind, die ein reines Herz besitzen, die Frieden stiften und die verfolgt werden.
	Anschließend meint er, dass alle das Himmelreich erben werden und dass sie getröstet, gesättigt, barmherzig behandelt, und Gottes Kinder genannt werden.
	Er sagt, dass sie fröhlich und jubelnd sein sollen, wenn sie um seinetwillen leiden.
	Er zeiht einen Vergleich zwischen den Menschen und dem Salz bzw. dem Licht der Erde und sagt, dass wir unser Licht nicht verstecken sollen, sondern leuchten lassen sollen, damit die Menschen unsere guten Werke sehen.
	Des Weiteren gibt er an, dass er nicht gekommen ist, um das Gesetz und die Propheten aufzulösen, sondern zu erfüllen.
	Er gibt zu bedenken, dass dieses Gesetz nicht vergehen wird, bis nicht alles passiert ist.
	Anschließend äußert er Kritik a den Schriftgelehrten und den Pharisäer, indem er sagt, dass unsere Gerechtigkeit besser sein muss, als die dieser Menschen, da wir sonst nicht ins Himmelreich kommen würden.
	Dann geht er ins Detail.
	Jesus erklärt, was das Gesetz wirklich bedeutet, und wie wir es nicht nur äußerlich, sondern auch innerlich halten sollen.
	Laut ihm genügt es nicht, nicht zu töten, sondern 
	uns auch nicht ärgern sollen.
	Jesus sagt, dass wir uns mit unserem Bruder versöhnen sollen, bevor wir Gott unsere Gabe bringen und dass es nicht genügt, nicht die Ehe zu brechen, sondern dass wir auch nicht begehren sollen.
	Ebenfalls sollen wir nicht falsch schwören, sondern einfach Ja oder Nein sagen.
	Am Ende seiner Prädikt gibt er zu bedenken, dass man Böses nicht mit Bösem vergelten soll und man nicht nur seine Freunde, sondern auch seine Feinde lieben soll und für alle beten soll.
\end{document}
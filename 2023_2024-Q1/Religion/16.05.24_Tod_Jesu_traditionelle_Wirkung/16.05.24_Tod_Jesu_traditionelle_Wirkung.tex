\documentclass[12pt,a4paper]{report}
\usepackage[T1]{fontenc}
\usepackage[utf8]{inputenc}
\usepackage{charter}
\usepackage{ngerman}
\usepackage[left=2cm,right=2cm,top=2cm,bottom=2cm]{geometry}

\renewcommand\thesection{\arabic{section}.} 

\begin{document}
	\section{Die Heilsbedeutung des Todes Jesu - traditionelle Erklärung}
	(Augustinus, 4. Jh. n. Chr.) \\
	\begin{enumerate}
		\item Gott
		\item Sündenfall (Menschen essen von dem Baum)
		\item[$\to$]Beziehung zwischen Gott und Mensch gestört
		\item[$\to$]\dq Erbsünde\dq
		\item[$\to$] Weitergabe des Sündenfalls an jeden Menschen
		\item[$\to$] Sündenfall führt zur Hölle 
		\item[$\to$] Gottes Sohn stirbt am Kreuz
		\item[$\to$] Versöhnung zwischen Gott und Mensch
	\end{enumerate}
	\paragraph{Problematik / Einwände:}
	\begin{itemize}
		\item Sünde kann nicht vererbbar sein $\to$ individuelle Selbstverantwortlichkeit (MODERNE)
		\item das \dq erste Menschenpaar\dq\ hat nicht existiert / Sündenfall ist kein historisches Ereignis ($\to$ historisch-kritische Sicht auf Bibel $\to$ MODERNE)
		\item problematisches Gottesbild: Gott \textbf{ist} Liebe (1 Joh.)
	\end{itemize}
\end{document}
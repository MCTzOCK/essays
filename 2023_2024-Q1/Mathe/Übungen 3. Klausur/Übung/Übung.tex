\documentclass[12pt,a4paper]{report}
\usepackage[T1]{fontenc}
\usepackage[utf8]{inputenc}
\usepackage{charter}
\usepackage{ngerman} \usepackage[left=2cm,right=2cm,top=2cm,bottom=2cm]{geometry}
\usepackage{amsmath}
\usepackage[dvipsnames]{xcolor}
\usepackage{tabularx}

\newcommand{\richtig}[1]{\color{ForestGreen}\textbf{
	Laut Lösungen richtig (S. #1)
}\color{black}}

\newcommand{\falsch}[1]{\color{BrickRed}\textbf{
	Laut Lösungen falsch (S. #1)
}\color{black}}

\begin{document}
	\noindent
	\Large S. 26 Nr. 3
	\large
	\noindent
	\paragraph{a)} \mbox{} \\
	\begin{align*}
		f(x) = 0.5 \cdot 2^x &\to \text{Graph D} \\
		f(x) = 2^x + 2 &\to \text{Graph B} \\
		f(x) = 0.5 \cdot 2^{x+1} &\to \text{Graph A} \\
		f(x) = 2^{-x} &\to \text{Graph E} \\
		f(x) = 0.5 \cdot 2^x + 1 &\to \text{Graph C} \\
	\end{align*}
	\richtig{135} \\[0.5cm]
	\Large S. 26 Nr. 4
	\large
	\noindent
	\begin{align*}
		x^{\frac{1}{2}} = \frac{1}{2} &\to x =\frac{1}{4} \\
		(\frac{1}{2})^{2x} = 64 &\to x= -3 \\
		x^{-8} = 256 &\to x = \frac{1}{2} \\
		(\frac{1}{4})^x + 3 = 5 &\to x = -\frac{1}{2} \\
		x^{\frac{3}{4}} = 27 &\to x = 81 \\
		6 \cdot 2^{3-x} = 24 &\to x = 1
 	\end{align*}
 	\richtig{135} \\[0.5cm]
 	\Large S. 26 Nr. 7
	\large
	\noindent
	\paragraph{a)}
	\begin{align*}
		\text{Gesamte prozentuale Steigerung} &= (1+0.4)^7 \\
		&= 1054 \% \\
		\text{Abzug der ursprünglichen 100 \%} &\to 1054 \% - 100 \% \\
		&= 954 \%
	\end{align*}
	\richtig{135}
	\paragraph{b)}
	\begin{align*}
		\text{10 \%:}& \\
		(1 + 0.4)^x &= 1.10 \\
		x&\approx 0.2832 \\
		\text{100\%:}& \\
		(1+0.4)^x &= 2 \\
		x &\approx 2.06
	\end{align*}
	\richtig{135}
	\paragraph{c)}
	\begin{align*}
		10 \cdot 1.4^x &= 2000 \\
		x &\approx 15.74
	\end{align*}
	\richtig{135}
	\\[0.5cm]
	\Large S. 27 Nr. 1
	\large
	\noindent \\[0.5cm]
	\begin{tabularx}{\linewidth}{|X|X|X|X|}
		\hline
		$f(x)$ & $e^x + 3x$ & $2e^x + 0.5x^2$ & $-4e^x + \frac{1}{3}x^4$ \\
		\hline
		$f'(x)$ & $e^x + 3$ & $2e^x + x$ & $-4e^x + \frac{4}{3}x^3$ \\
		\hline
		$f''(x)$ & $e^x$ & $2e^x + 1$ & $-4e^x + 4x^2$ \\
		\hline
	\end{tabularx}
	\\[1cm]
	\noindent
	\begin{tabularx}{\linewidth}{|X|X|X|}
		\hline
		$f(x)$ & $sin(x) - e^x$ & $\frac{1}{3}\cdot(e^x+x^2)$ \\
		\hline
		$f'(x)$ & $cos(x) - e^x$ & $\frac{e^x + 2x}{3} $\\
		\hline
		$f''(x)$ & $-sin(x) - e^x$ & $\frac{e^x + 2}{3} $\\
		\hline
	\end{tabularx}
	\\[1cm]
	\richtig{135} \newpage
	\noindent
	\Large S. 29 Nr. 1
	\large
	\noindent
	\paragraph{b)} \mbox{} \\
	Die Verdoppelungszeit beträgt in etwa $T_V \approx 2$ Jahre \\[0.2cm]
	\noindent
	\richtig{136}
	\noindent \\[0.5cm]
	\Large S. 35 Nr. 1
	\large
	\noindent
	\paragraph{a)} \mbox{} \\
	\begin{align*}
		f(x) = (x^3 + 1)(x - 4) \\
		u =x^3 + 1;\ v = x-4;\\
		u'=3x^2;\ v'=1 \\
		f'(x) = 3x^2 \cdot (x-4) + 1\cdot (x^3 + 1) \\
		f'(x) = 3x^3 - 12x^2 + x^3 + 1 \\
		f'(x) = 4x^3 - 12x^2 + 1
	\end{align*}
	\paragraph{b)} \mbox{} \\
	\begin{align*}
		f(x) = (x+3)(2-x) \\
		u=x+3;\ v=2-x;\\
		u'=1;\ v'=-1 \\
		f'(x) = -1\cdot (x+3)+1 \cdot (2-x) \\
		f'(x) = -x - 3 + 2 - x \\
		f'(x) = -2x - 1
	\end{align*}
	\paragraph{c)} \mbox{} \\
	\begin{align*}
		f(x) = (x^4 - 2x^2 + 3)(x^2 - 4) \\
		u= x^4 - 2x^2 + 3;\ v=x^2-4 \\
		u'=4x^3 - 4x;\ v'=2x \\
		f'(x) = (4x^3 -4x) \cdot (x^2-4) + 2x \cdot (x^4 - 2x^2 + 3) \\
		f'(x) = 4x^5 - 16x^3 - 4x^3 + 16x + 2x^5 - 4x^3 + 6x \\
		f'(x) = 4x^5 + 2x^5 - 16x^3 - 4x^3 - 4x^3 + 16x + 6x \\
		f'(x) = 6x^5 - 24x^3 + 22x
	\end{align*}
	\richtig{140}
	\noindent \\[0.5cm]
	\Large S. 35 Nr. 6
	\large
	\noindent
	\paragraph{a)}
	\begin{align*}
		f(x) = (3x - 2) \cdot \sqrt{x}; x_0 = 4 \\
		u = (3x -2);\ v=\sqrt{x} \\
		u'= 3;\ v'=\frac{1}{2}\sqrt{x}; \\
		f'(x) = u' \cdot v + v' \cdot u \\
		f'(x) = 3 \cdot \sqrt{x} + \frac{1}{2\sqrt{x} }\cdot (3x - 2) \\
		f'(x_0) \\
		f'(4) = 3 \cdot \sqrt{4} + \frac{1}{2\sqrt{4}} \cdot (3\cdot4 - 2) \\
		f'(4) = 6 + \frac{1}{4} \cdot 10 \\
		f'(4) = 6 + 2.5 \\
		f'(4) = 8.5 \\
		\Rightarrow m=8.5 \\
		t(x) = 8.5x+b \\
		f(x_0)\\
		f(4) = (3\cdot 4-2)\cdot \sqrt{4} \\
		f(4) = 20 \\
		\Rightarrow 20 = 8.5 \cdot 4 + b \\
		\Leftrightarrow 20 = 34 + b\\
		\Leftrightarrow b = 14 \\
		\Rightarrow t(x) = 8.5x + 14
	\end{align*}
	\richtig{141}
	\newpage
	\paragraph{b)}
	\begin{align*}
		f(x) = \frac{1}{2}x \cdot e^x;\ x_0 = 2\\
		u = \frac{1}{2}x;\ v = e^x; \\
		u' = \frac{1}{2};\ v' = e^x \\
		f'(x) = \frac{1}{2}x \cdot e^x + e^x \cdot \frac{1}{2} \\
		f'(x_0) \\
		f'(2) = \frac{1}{2} \cdot 2 \cdot e^2 + e^2 \cdot \frac{1}{2} \\
		f'(2) = 1 \cdot e^2 + e^2 \cdot \frac{1}{2} \\
		f'(2) = \frac{3}{2}e^2 \\
		\Rightarrow m = \frac{3}{2}e^2 \\
		t(x) = \frac{3}{2}e^2\cdot x + b \\
		f(x_0) \\
		f(2) = \frac{1}{2} \cdot 2 \cdot e^2 \\
		f(2) = e^2 \\
		\Rightarrow P(e^2|\frac{3}{2}e^2) \\
		\Rightarrow e^2 = \frac{3}{2}e^2 \cdot 2 + b \\
		\Leftrightarrow e^2 = \frac{6}{2}e^2 + b \\
		\Leftrightarrow e^2 = 3e^2 + b \\
		\Leftrightarrow -2e^2 = b \\
		\Rightarrow t(x) = \frac{3}{2}e^2\cdot x-2e^2
	\end{align*}
	\richtig{141}
	\newpage
	\paragraph{c)}
	\begin{align*}
		f(x) = \sqrt{x} \cdot (2x^2 - 4);\ x_0 = 2 \\
		u = \sqrt{x};\ v = (2x^2 - 4);\\
		u'=\frac{1}{2\sqrt{x}};\ v' = 4x \\
		f'(x) = \sqrt{x} \cdot 4x + \frac{1}{2\sqrt{x}} \cdot (2x^2 - 4) \\
		f'(x_0)\\
		f'(2) = \sqrt{2} \cdot 4\cdot 2 + \frac{1}{2\sqrt{2}} \cdot (2\cdot 2^2 - 4) \\
		f'(2) = \sqrt{2} \cdot 8 + \frac{1}{2\sqrt{2}} \cdot 4 \\
		f'(2) = \sqrt{2} \cdot 8 + \frac{4}{2\sqrt{2}} \\
		f'(2) = \sqrt{2} \cdot 8 +\sqrt{2} \\
		f'(2) = 9\sqrt{2} \\
		\Rightarrow m = 9\sqrt{2} \\
		\Rightarrow t(x) = 9\sqrt{2}x + b \\
		f(x_0) \\
		f(2) = \sqrt{2} \cdot (2\cdot 2^2 - 4) \\
		f(2) = \sqrt{2} \cdot (4) \\
		f(2) = 4\sqrt{2} \\
		\Rightarrow P(2|4\sqrt{2}) \\
		\Rightarrow 4\sqrt{2} = 9\sqrt{2}\cdot 2 + b \\
		\Leftrightarrow 4\sqrt{2} = 18\sqrt{2} + b \\
		\Leftrightarrow -14\sqrt{2} = b \\
		\Rightarrow t(x) = 9\sqrt{2}x - 14\sqrt{2}
	\end{align*}
	\richtig{141}
	\newpage
	\noindent
	\Large
	S. 102 Nr. 9
	\large
	\paragraph{a)}
	\begin{align*}
		f(x) &= ca^x \\
		P(0|6)\ &\land\ Q(3|\frac{2}{9}) \\
		\Rightarrow c &= 6 \\
		\Rightarrow f(x) &= 6a^x \\
		f(3) &= \frac{2}{9} \\
		\Leftrightarrow 6a^3 &= \frac{2}{9}\ |\div 6 \\
		\Leftrightarrow a^3 &= \frac{\frac{2}{9}}{6} \\
		\Leftrightarrow a^3 &= \frac{2}{9} \cdot \frac{1}{6} \\
		\Leftrightarrow a^3 &= \frac{1}{27} \\
		\Leftrightarrow a &= \frac{1}{3} \\
		\Rightarrow f(x) &= 6\cdot (\frac{1}{3})^x
	\end{align*}
	\richtig{147}
	\paragraph{b)}
	\begin{align*}
		f(x) &= 162 \\
		\Leftrightarrow 6 \cdot (\frac{1}{3})^x &= 162 \\
		\Leftrightarrow (\frac{1}{3})^x &= 27 \\
		\Leftrightarrow x &= \log_{\frac{1}{3}}(27) \\
		\\
		f(x) &= 2\sqrt{3} \\
		\Leftrightarrow 6 \cdot (\frac{1}{3})^x &= 2\sqrt{3} \\
		\Leftrightarrow (\frac{1}{3})^x &= \frac{2\sqrt{3}}{6} \\
		\Leftrightarrow (\frac{1}{3})^x &= \frac{\sqrt{3}}{3} \\
		\Leftrightarrow x &= \log_{\frac{1}{3}}(\frac{\sqrt{3}}{3})
	\end{align*}
	\richtig{417}
	\newpage
	\noindent
	\Large S. 139 Nr. 6
	\large
	\paragraph{a)}
	\begin{align*}
		f(x) &= (\frac{1}{2}x + 5)^2 \\
		u = x^2&;\ v=\frac{1}{2}x + 5\\
		u' = 2x&;\ v'=\frac{1}{2} \\
		u&\circ v\\
		f'(x) &= u'(v(x)) \cdot v'(x) \\
		\Leftrightarrow f'(x) &= 2\cdot (\frac{1}{2}x + 5) \cdot \frac{1}{2} \\
		\Leftrightarrow f'(x) &= \frac{1}{2}x + 5
	\end{align*}
	\richtig{426}
	\paragraph{b)}
	\begin{align*}
		f(x) &= (3-x)^5 \\
		u=x^5&;\ v=3-x\\
		u'=5x^4&;\ v'=-1 \\
		u&\circ v\\
		f'(x) &= u'(v(x)) \cdot v'(x) \\
		\Leftrightarrow f'(x) &= 5 \cdot (3-x)^4 \cdot (-1) \\
		\Leftrightarrow f'(x) &= -5 (3-x)^4
	\end{align*}
	\richtig{426}
	\paragraph{c)}
	\begin{align*}
		f(x) &= e^{5x + 6} \\
		u= e^x&;\ v=5x+6\\
		u'=e^x&;\ v'=5\\
		u&\circ v \\
		f'(x) &= u'(v(x)) \cdot v'(x) \\
		\Leftrightarrow f'(x) &= e^{5x + 6} \cdot 5 \\
		\Leftrightarrow f'(x) &= 5e^{5x + 6}
	\end{align*}
	\richtig{426}
\end{document}
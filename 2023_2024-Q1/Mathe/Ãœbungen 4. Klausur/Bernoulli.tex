\documentclass[12pt,a4paper]{report}
\usepackage[T1]{fontenc}
\usepackage[utf8]{inputenc}
\usepackage{charter}
\usepackage{ngerman}
\usepackage[left=2cm,right=2cm,top=2cm,bottom=2cm]{geometry}
\usepackage{amsmath}

\renewcommand\thesection{\arabic{section}.} 

\begin{document}
	\section{S. 284 Nr. 8}
	\paragraph{a)}
	\begin{align*}
		P(X=2) &= \binom{6}{2} \cdot 2 \cdot (1-0.75)^{6-2} \\
		P(X=2) &= \binom{6}{2} \cdot 2 \cdot (0.25)^4 \\
		P(X=2) &= 0.032958984375 \approx 32.9\ \%
	\end{align*}
	\paragraph{b)}
	\begin{align*}
		P(Y=2) &= \binom{6}{2} \cdot 2 \cdot (1-0.25)^{6-2} \\
		P(Y=2) &= \binom{6}{2} \cdot 2 \cdot (0.75)^4 \\
		P(Y=2) &= 29.66\ \%
	\end{align*}
	\paragraph{c)}
	\begin{align*}
		P(X < 3) &= 3.7\ \%
	\end{align*}
	\paragraph{d)}
	\begin{align*}
		P(Y > 2) &= 46.6\ \%
	\end{align*}
	\section{S. 284 Nr. 9}
	\paragraph{a)}
	\begin{align*}
		P(X > 8) &= 67.7\ \%
	\end{align*}
	\paragraph{b)}
	\begin{align*}
		P(X = 5) &= 31.14\ \% \\
		P(X = 6) &= 31.14\ \% \\
		31.14\ \% + 31.14\ \% &= 62.28\ \%
	\end{align*}
	\section{S. 289 Nr. 7}
	\paragraph{a)}
	\begin{align*}
		P(X=4) &= 18.7\ \% \\
		P(X \leq 4) &= 83.57\ \%
	\end{align*}
	\paragraph{b)}
	Man kann so $P(X\geq 3)$ auch so berechnen, da $1-P(X \leq 2)$ die Gegenwahrscheinlichkeit ist.
	\paragraph{c)}
	\begin{align*}
		P(1 \leq X \leq 5) &= 90.37\ \%
	\end{align*}
\end{document}
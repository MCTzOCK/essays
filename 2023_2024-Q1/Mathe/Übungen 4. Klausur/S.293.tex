\documentclass[12pt,a4paper]{report}
\usepackage[T1]{fontenc}
\usepackage[utf8]{inputenc}
\usepackage{charter}
\usepackage{ngerman}
\usepackage[left=2cm,right=2cm,top=2cm,bottom=2cm]{geometry}
\usepackage{amsmath}

\renewcommand\thesection{\arabic{section}.} 

\begin{document}
	\section{S. 293 Nr. 10}
	\paragraph{a)}
	\begin{align*}
		P(X \geq 50) &= 0.5397 \approx 53.97\ \% \\
		P(45 \leq X \leq 55) &= 72.87 \approx 72.87\ \%
	\end{align*}
	\paragraph{b)} \mbox{} \\
	Wahrscheinlichkeit \textbf{keinen} Jungen zu bekommen beträgt: 
	\begin{align*}
		1-p &= 0.5 \\
		-p &= -0.5 \\
		p &= 0.5
	\end{align*}
	Wahrscheinlichkeit, dass alle Kinder weiblich sind:
	\begin{align*}
		P(X = 100) &= 0.5^{n}
	\end{align*}
	Wahrscheinlichkeit für mindestens einen Jungen:
	\begin{align*}
		p &= 1-0.5^{n}
	\end{align*}
	Daraus folgt:
	\begin{align*}
		1-0.5^{n} &\geq 0.99 \\
		0.5^{n} &\geq 0.01
	\end{align*}
	
	\section{S. 293 Nr. 11}
	\begin{align*}
		P(X > 1) &\leq 0.1 \\
		P(X \leq 1) &\geq 0.99 
	\end{align*}
\end{document}
\documentclass[11pt, a4paper]{report}

\usepackage{custompkg}
\usepackage[utf8]{inputenc}
\usepackage[T1]{fontenc}
\usepackage{ngerman}
\usepackage[left=2cm,right=2cm,top=2cm,bottom=2cm]{geometry}
\usepackage{charter}
\usepackage{amsmath}
\usepackage{mathbbol}
\usepackage{amssymb}

\begin{document}
	\title{Exponentialfunktionen Übung (Gruppe 2)}
	\author{Ben Siebert \and Tanel Malak \and Moritz Junkermann}
	\date{\today}
	\maketitle
	\tableofcontents
	\bsremovechaptertitle
	\bslinespacing{1.5}
	\chapter{S. 140 Nr. 10}
	Gegeben ist die Funktion f mit $f(x) = 2x \times e^{x^2-2}$
	\section{Teilaufgabe a)}
	Zu überprüfende Stammfunktion: $F(x) = e^{x^2-2}$ \\
	Gegebene Stammfunktion ableiten: \\
	$u = e^x;\ v = x^2 - 2;\ u' = e^x;\ v' = 2x\\
	f(x) = u'(v(x)) \times v'(x) \\
	f(x) = e^{x^2 - 2} \times 2x \\
	$
	Die Stammfunktion $F(x) = e^{x^2-2}$ ist die korrekte Stammfunktion der Funktion $f(x) = 2x \times e^{x^2-2}$
	\section{Teilaufgabe b)}
	Intervall $I\Bigl[0;2\Bigl]$ \\
	$
	\int_{0}^{2}f(x) dx \Bigl[F(x) = e^{x^2-2}\Bigl]\\
	\int_{0}^{2}f(x) dx = F(2) - F(0) \\
	\Leftrightarrow e^{2^2-2} - e^{0^2-2} \\
	\Leftrightarrow e^{2} - e^{-2} \\
	\xrightarrow[]{CAS}\ \approx 7.25372 \\
	$
	Die Fläche, die von dem Graphen der Funktion $f$, der x-Achse und der Geraden $x=2$ eingeschlossen wird, beträgt in etwa $e^2 - e^{-2}$.
	\section{Teilaufgabe c)}
	Durch die Achsensymmetrie der Funktion $f(x)$ gleicht sich der Anteil der Integrale unterhalb und oberhalb der x-Achse aus, sodass das Intervall von $-a$ bis $a$ immer Null ist.
	\chapter{S. 142 Nr. 24}
	\section{Teilaufgabe a)}
	Ansatz:
	\\
	$\int_{1}^{e} (\frac{3}{x} + x) dx = F(e) - F(1)$
	\\
	Stammfunktion bilden:
	\\
	$\int_{1}^{e} (\frac{3}{x} + x) dx \Bigl[F(x) = 3 \times ln(|x|) + \frac{x^2}{2}\Bigl]$
	\\
	Integral bestimmen (und vereinfachen):
	\\
	$\int_{1}^{e} (\frac{3}{x} + x) dx = F(e) - F(1) \\
	\Leftrightarrow (3 \times ln(|e|) + \frac{e^2}{2}) - (3 \times ln(|1|) + \frac{1^2}{2}) \\
	\Leftrightarrow (3 \times ln(|e|) + \frac{e^2}{2}) - (3 \times 0 + \frac{1}{2}) \\
	\Leftrightarrow (3 \times ln(|e|) + \frac{e^2}{2}) - \frac{1}{2} \\
	\Leftrightarrow (3 \times 1 + \frac{e^2}{2}) - \frac{1}{2} \\
	\Leftrightarrow (3 + \frac{e^2}{2}) - \frac{1}{2} \\
	\Leftrightarrow \frac{6}{2} + \frac{e^2}{2} - \frac{1}{2} \\
	\Rightarrow \frac{6 + e^2 - 1}{2} \\
	\xrightarrow[]{CAS}\ \approx 6.19
	$
	\section{Teilaufgabe b)}
	Ansatz:
	\\
	$\int_{1}^{e} (\frac{2 + 4x}{x}) dx = F(e) - F(1)$
	\\
	Stammfunktion bilden:
	\\
	$
	\int_{1}^{e} (\frac{2 + 4x}{x}) dx\ \Rightarrow f(x) = 2 + 4x \times \frac{1}{x}\ \Bigl[F(x) = 2x + 2x^2 \times ln(|x|)\Bigl]
	$
	\\
	Integral berechnen (und vereinfachen):
	\\
	$
	\int_{1}^{e} (\frac{2 + 4x}{x}) dx = F(e) - F(1) \\
	\Leftrightarrow (2e + 2e^2 \times ln(|e|)) - 2 + 2 \times ln(|1|) \\
	\Leftrightarrow (2e + 2e^2 \times 1) - 2 + 2 \times 0 \\
	\Leftrightarrow (2e + 2e^2) - 2 \\
	\Leftrightarrow 2e^2 + 2e - 2 \\
	\xrightarrow[]{CAS}\ \approx 18.21
	$
	\section{Teilaufgabe c)}
	Ansatz:
	\\
	$\int_{e}^{e^2} (\frac{2x - 5}{x^2}) dx\ = F(e^2) - F(e)$
	\\
	Stammfunktion bilden:
	\\
	$
	\int_{e}^{e^2} (\frac{2x - 5}{x^2}) dx \Rightarrow f(x) = 2x - 5 \times \frac{1}{x^2} = 2x - \frac{5}{x^2}\ \Bigl[F(x) = x^2 + \frac{5}{x}\Bigl]
	$
	\\
	Integral berechnen (und vereinfachen):
	\\
	$
	F(e^2) - F(e) \\
	\Leftrightarrow e^4 + \frac{5}{e^2} - e^2 + \frac{5}{e} \\
	\xrightarrow[]{CAS}\ \approx 49.72
	$
	
	\chapter{S. 145 Nr. 5g}
	Funktion: $f(x) = 2x \times e^{2x-7}$ \\
	\textbf{Nullstellen}: \\
	$f(x) = 0 \\
	2x \times e^{2x-7} = 0$
	\\
	$\to x_0 = 0$, da $2x$ nur $0$, wenn $x=0$ ist und $e^x$ niemals gleich $0$ sein kann.
	\\
	\textbf{Symmetrie}:
	\\
	$f(-x) = 2 \times (-x) \times e^{2 \times (-x) -7}$
	\\
	Es ist zu sehen, dass $f(-x) \ne f(x)$ bzw. $-f(x)$ ist, sodass der Graph der Funktion $f(x)$ keine Symmetrie besitzt. 
	\\
	\textbf{Ableitungen:}
	\\
	$
	u = 2x;\ v = e^{2x - 7};\ u' = 2; \\
	s = e^x;\ t = 2x - 7; \ s' = e^x;\ t' = 2; \\
	v' = e^{2x - 7} \times 2 \\
	f'(x) = u' \times v + v' \times u \\
	f'(x) = 2 \times e^{2x - 7}+ e^{2x-7} \times 2 \times 2x \\
	f'(x) = e^{2x-7} \times (2 + 2 \times 2x) \\
	f'(x) = e^{2x-7} \times (2 + 4x) \\
	\\
	f''(x) = u_2' \times v_2 + u_2 \times v_2'\\
	u_2 = 2 + 4x;\ u_2' = 4;\ v_2 = e^{2x - 7}; \\
	s = e^x;\ t = 2x - 7;\ s' = e^x;\ t' = 2; \\
	v' = e^{2x - 7} \times 2 \\
	f''(x) = 4 \times e^{2x - 7} + (2x + 4) \times e^{2x - 7} \times 2 \\
	f''(x) = e^{2x - 7} \times (4 + (2x + 4) \times 2) \\
	f''(x) = e^{2x - 7} \times (4 + 4x + 8) \\
	f''(x) = e^{2x - 7} \times (12 + 4x) \\
	$
	\\
	\textbf{Verhalten nahe Null:}
	\\
	Das Verhalten nahe Null kann nicht bestimmt werden, da die Funktion keine Summanden enthält und so der geringste Exponent von $x$ nicht bestimmt werden kann.
	\\
	\textbf{Extremstellen:}
	\\
	notwendige Bedingung für EST: $f'(x) = 0$ \\
	$
	e^{2x-7} \times (2 + 4x) = 0 \\
	\Leftrightarrow 2 + 4x = 0\ \ \Bigl|\div 4 \\
	\Leftrightarrow 0.5 + x = 0\ \Bigl|-0.5\\
	\Leftrightarrow x = -\frac{1}{2} \\
	x_1 = -\frac{1}{2}
	$
	\\
	hinreichende Bedingung für EST: $f'(x) = 0\ \land\ f''(x) \ne 0$ \\
	$
	f''(-\frac{1}{2}) = e^{2x - 7} \times (12 + 4x) \\
	\Leftrightarrow f''(-\frac{1}{2}) > 0 \\
	\Rightarrow Tiefpunkt
	$
	\\
	Y-Wert bestimmen: \\
	$
	f(\frac{1}{2}) = 2x \times e^{2x-7} \\
	\Leftrightarrow 2 \times (-\frac{1}{2}) \times e^{2 \times (-\frac{1}{2}) - 7} \\
	\Leftrightarrow -1 \times e^{-8}
	$
	\\
	Die Funktion $f(x)$ besitzt einen Tiefpunkt $TP(\frac{1}{2}\Bigl|-e^{-8})$
	\\
	\textbf{Randverhalten:}
	\\
	Das Randverhalten kann nicht bestimmt werden, da die 	Funktion keine Summanden enthält und so der größte 	Exponent von $x$ nicht bestimmt werden kann.
		\\
	\textbf{Wendestellen:}
	\\
	\textbf{Monotinie:}
	\\
	
\end{document}
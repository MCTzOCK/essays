\documentclass[a4paper, 12pt]{report}
\usepackage[T1]{fontenc}
\usepackage[utf8]{inputenc}
\usepackage{charter}
\usepackage{ngerman}
\usepackage[left=2cm,right=2cm,top=2cm,left=2cm]{geometry}
\usepackage{graphicx}
\usepackage{amsmath}

\begin{document}
	\noindent
	\Large S. 146 Nr. 11
	\large
	\paragraph{a)} \mbox{} \\
	Der Y-Achsenabschnitt ist immer der Gleiche.
	\\
	\textbf{Nullstellen:} \\
	Der Graph hat für $t \leq 0$ eine Nullstelle. Für $t > 0$ hat er entweder keine oder zwei Nullstellen. Für $t = e$ besitzt der Graph eine Nullstelle.
	\\
	\begin{align*}
		f_t(x) &= 0 \\
		e^x - tx &= 0
	\end{align*}
	\textbf{Extremstellen:} \\
	Der Graph hat für $t > 0$ immer einen Tiefpunkt. Für $t \leq 0$ besitzt der Graph keine Extremstellen.
	\begin{align*}
		f_t'(x) &= e^x - t\\
		f_t'(x) &= 0 \\
		e^x - t &= 0 \\
		x &= ln(t)\ \land\ t > 0
	\end{align*}
	Es gibt Extremstellen, wenn $t > 0$ ist.
	\paragraph{b)} \mbox{} \\
	notwendige Bedingung für EST: $f_t'(x) = 0$
	\begin{align*}
		f_t'(x) &= e^x - t\\
		f_t'(x) &= 0 \\
		e^x - t &= 0 \\
		x &= ln(t)\ \land\ t > 0
	\end{align*}
	hinreichende Bedingung für EST: $f_t'(x) = 0\ \land\ f_t''(x) \ne 0$
	\begin{align*}
		f_t''(x) &= e^x \\
		f_t''(ln(t)) &= e^{ln(t)} \\
		f_t''(ln(t)) &> 0 \Rightarrow TP
	\end{align*}
	\textbf{Koordinaten:}
	\begin{align*}
		f_t(ln(t)) &= t - t \cdot ln(t)
	\end{align*}
	$TP \Bigl(ln(t)\Bigl|t - t \cdot ln(t) \Bigl)$
	\newpage
	\noindent
	\textbf{Ortskurve:}
	\begin{align*}
		x &= ln(t) \\
		t &= e^x
	\end{align*}
	$t$ in Y-Koordinate einsetzen.
	\begin{align*}
	y &= e^x - e^x \cdot ln(e^x) \\
	y &= (1-x) \cdot e^x	 \\
	& \Rightarrow g(x) = (1-x) \cdot e^x
	\end{align*}
	\Large S. 146 Nr. 12
	\\
	\paragraph{a)}
	\large
	\begin{align*}
		f_k(x) &= (2x + 3k) \cdot e^{(x+1)} \\
		f_k'(x) &= 5.4365 \cdot (x + 8.5) \cdot e^x \\
		f_k''(x) &= 5.4365 \cdot (x + 9.5) \cdot e^x \\
	\end{align*}
	\textbf{Extremstellen:} \\
	notwendige Bedingung für EST: $f_k'(x) = 0$
	\begin{align*}
		f_k'(x) &= 0 \\
		x &= \frac{-(3 \cdot k + 2)}{2}
	\end{align*}
	hinreichende Bedingung für EST: $f_k'(x) = 0\ \land f_k''(x) \neq 0$
	\begin{align*}
		f_k''(\frac{-(3 \cdot k + 2)}{2}) &= 0.001106\ > 0 \\
		&\Rightarrow TP
	\end{align*}
	Y-Koordinate bestimmen: $f_k(\frac{-(3 \cdot k + 2)}{2})$ \\
	\begin{align*}
		f_k(\frac{-(3 \cdot k + 2)}{2})= -2 \cdot e^{\frac{-3 \cdot k}{2}}
	\end{align*}
	Tiefpunkt: $
	TP\Bigl(\frac{-(3 \cdot k + 2)}{2}\Bigl| -2 \cdot e^{\frac{-3 \cdot k}{2}}\Bigl)$
	\newpage\noindent
	\textbf{Wendestellen:}
	\\
	notwendige Bedingung für WST: $f_k''(x) = 0$ \\
	\begin{align*}
		f_k''(x) &= 0 \\
		x &= \frac{-(3 \cdot k + 4)}{2} \\
	\end{align*}
	hinreichende Bedingung für WST: $f''_k(x) = 0\ \land\ f'''_k(x) \ne 0$\\
	\begin{align*}
		f'''_k(\frac{-(3 \cdot k + 4)}{2}) &= 2 \cdot e^{\frac{-3 \cdot k}{2} - 1}
	\end{align*}
	\paragraph{b)}
	\begin{align*}
		4 &= f_k(-1) \\
		\Leftrightarrow 4 &= (2 \cdot (-1) + 3k) \cdot e^{(-1+1)} \\
		\Leftrightarrow 4 &= (-2 + 3k) \cdot e^0 \\
		\Leftrightarrow 4 &= -2 + 3k\ \Bigl|+2 \\
		\Leftrightarrow 6 &= 3k\ \Bigl|\div 3 \\
		\Leftrightarrow k &= 2
	\end{align*}
	
	\paragraph{c)} \mbox{} \\
	Bei einer Erhöhung von $k$, wird die Extremstelle nach links verschoben und nähert sich der $x$-Achse an.
	
	\paragraph{d)} \mbox{} \\
	\begin{align*}
		x &= \frac{-(3 \cdot k + 2)}{2} \\
		k &= \frac{-2 \cdot (x + 1)}{3} \\
		y &=-2 \cdot e^{
			\frac{
				-3 \cdot \frac{-2 \cdot (x + 1)}{3}
			}{2}
		} \\
		y &= -2 \cdot e^{x + 1} \\
		g(x) &= -2 \cdot e^{x + 1}
	\end{align*}
	\Large S. 154 Nr. 9
	\\
	\large
	\begin{align*}
		f_k(x) &= x - k \cdot e^x \\
		f_k'(x) &= 1 - k \cdot e^{x} \\
		f_k''(x) &= -k \cdot e^{x}
	\end{align*}
	\paragraph{a)} \mbox{} \\
	Für $k \leq 0$ und $x \to \infty$ strebt $f(x)$ ebenfalls $\infty$ an. \\
	Für $k \leq 0$ und $x \to -\infty$ strebt $f(x)$ ebenfalls $-\infty$ an. \\
	Für $k > 0$ und $x \to \infty$ strebt $f(x)$ ebenfalls $-\infty$ an. \\
	Für $k > 0$ und $x \to -\infty$ strebt $f(x)$ ebenfalls $-\infty$ an. \\
	(Asymptote)
	\paragraph{b)} \mbox{} \\
	\begin{align*}
		f_{k_1}(x) &= f_{k_2}(x) \\
		x - k_1 \cdot e^x &= x - k_2 \cdot e^x 
	\end{align*}
	Hierdurch würde $k_1 = k_2$ gelten, was durch die Aufgabenstellung revidiert wird, da $k_1 \ne k_2$ gilt.
	
	\paragraph{c)} \mbox{} \\
	notwendige Bedingung für EST: $f'_k(x) = 0$ \\
	\begin{align*}
		f'_k(x) &= 0 \\
		\xrightarrow{CAS} x &= ln(\frac{1}{k})\ \land\ k > 0
	\end{align*}
	hinreichende Bedingung für EST: $f_k'(x) = 0\ \land f_k''(x) \ne 0$ \\
	\begin{align*}
		f_k''(ln(\frac{1}{k})) &= - 1 \\
		-1 < 0 &\to \text{Hochpunkt}
	\end{align*}
	Y-Wert: $f(ln(\frac{1}{k})) = ln(\frac{1}{k}) - 1$ \\[0.5cm]
	Hochpunkt: $\text{HP}(ln(\frac{1}{k})| ln(\frac{1}{k}) - 1)$ \\
	Der Graph besitzt für $k > 0$ immer einen Hochpunkt. Für $k < 0$ existieren keine Extremstellen.
	
	\paragraph{d)} \mbox{} \\
	\begin{align*}
		x &= ln(\frac{1}{k}) \\
		k &= e^{-x}
	\end{align*}
	In Y-Koordinate einsetzen:
	\begin{align*}
		y &= x - 1 \\
		g(x) &= x - 1
	\end{align*}
	
	\paragraph{e)} \mbox{} \\
	\begin{align*}
		\xrightarrow{CAS} F_k(x) &= \frac{x^2}{2} - k \cdot e^x
	\end{align*}
	\paragraph{f)} \mbox{} \\
	\begin{align*}
		\int_{a}^{0} (f_k(x) - g(x)) dx &= [F(x) = ]
	\end{align*}
	

\end{document}
\documentclass[25pt, a3paper]{tikzposter}
\title{Exponentialfunktionen}
\author{Ben Siebert}
\date{\today}
\institute{Gymnasium Holthausen Hattingen}

\usetheme{Wave}
\tikzposterlatexaffectionproofoff

% Packages
\usepackage{ngerman}
\usepackage{charter}
\usepackage[utf8]{inputenc}
\usepackage[T1]{fontenc}
\usepackage[fleqn]{mathtools}
\usepackage{amssymb}
\usepackage{amsthm}
\usepackage{amsmath}
\usepackage{multicol}
\usepackage{graphicx}
\usepackage{pgfplots}


\begin{document}
\maketitle
\begin{columns}
    \column{0.5}
    \block{Einleitung}{
        Eine Exponentialfunktion ist eine Funktion, bei der die Variable im Exponenten steht. \\
        \begin{center}
            $f(x) = a^x \Bigl[a \in \mathbb{R}\Bigl]$
        \end{center}
        \vspace{1cm}
        Eine solche Funktion ist nicht symmetrisch zu einer Achse. \\
        \begin{tikzpicture}[scale=1.5]
            \begin{axis}[
                    axis lines = center,
                    xlabel = $x$,
                    ylabel = {$f(x) = 2^x$},
                    xmin = -5,
                    xmax = 5,
                    ymin = -5,
                    ymax = 5,
                    xtick = {-5,5},
                    ytick = {-5,5},
                ]
                \addplot [
                    domain=-5:5,
                    samples=100,
                    color=red,
                ]
                {2^x};
            \end{axis}
        \end{tikzpicture}
    }
    \column{0.5}
    \block{e-Funktion}{
        Die e-Funktion ist eine spezielle Exponentialfunktion, bei der die Basis die Eulersche Zahl $e$ ist. \\
        \begin{center}
            $f(x) = e^x$
        \end{center}
        \vspace{1cm}
        Die e-Funktion ist die Umkehrfunktion des natürlichen Logarithmus. \\
        \begin{center}
            $f(x) = \ln(x)$
        \end{center}
        \vspace{1cm}
        Die e-Funktion ist die einzige Exponentialfunktion, deren Ableitung gleich der Funktion selbst ist. \\
        \begin{center}
            $f(x) = e^x \Rightarrow f'(x) = e^x$
        \end{center}
        \vspace{1cm}
        Die e-Funktion ist die einzige Exponentialfunktion, deren Stammfunktion gleich der Funktion selbst ist. \\
        \begin{center}
            $f(x) = e^x \Rightarrow F(x) = e^x$
        \end{center}
        \vspace{1cm}
        \textbf{Grenzwerte der e-Funktion}: \\
        \begin{center}
            $\lim\limits_{x \to \infty} e^x = \infty$ \\
            $\lim\limits_{x \to -\infty} e^x = 0$
        \end{center}
        \textbf{Symmetrie der e-Funktion}: \\
        \begin{center}
            $e^{-x} = \frac{1}{e^x}$
        \end{center}
    }
\end{columns}
\begin{columns}
    \column{0.5}
    \block{Rechenregeln}{
        \begin{itemize}
            \item $a^x \cdot a^y = a^{x+y}$
            \item $\frac{a^x}{a^y} = a^{x-y}$
            \item $(a^x)^y = a^{x \cdot y}$
            \item $a^0 = 1$
            \item $a^{-x} = \frac{1}{a^x}$
            \item $a^{\frac{1}{2}} = \sqrt{a}$
            \item $a^{\frac{1}{x}} = \sqrt[x]{a}$
        \end{itemize}
    }
    \column{0.5}
    \block{Wachstum}{
        \begin{itemize}
            \item $a > 1 \Rightarrow$ Funktion wächst
            \item $a = 1 \Rightarrow$ Funktion konstant
            \item $0 < a < 1 \Rightarrow$ Funktion fällt
            \item $a = 0 \Rightarrow$ Funktion konstant
        \end{itemize}
    }
\end{columns}
\begin{columns}
    \column{0.5}
    \block{Ableitungsregeln für Produkte}{
        \begin{itemize}
            \item $f(x) = u(x) \cdot v(x)$
            \item $f'(x) = u'(x) \cdot v(x) + u(x) \cdot v'(x)$
        \end{itemize}
        Beispiel: \\
        \begin{center}
            $f(x) = x^2 \cdot e^x$ \\
            $f'(x) = 2x \cdot e^x + x^2 \cdot e^x$
        \end{center}
    }
\end{columns}
\end{document}
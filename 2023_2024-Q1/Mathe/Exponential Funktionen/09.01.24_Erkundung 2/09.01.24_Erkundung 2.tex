\documentclass[11pt, a4paper]{report}

\usepackage[T1]{fontenc}
\usepackage[utf8]{inputenc}
\usepackage{ngerman}
\usepackage{amsmath}
\usepackage{charter}
\usepackage{custompkg}
\usepackage[left=2cm,right=2cm,top=2cm,bottom=2cm]{geometry}
\usepackage{tabularx}
\usepackage{amsmath}
\usepackage{hyperref}

\begin{document}
	\bslinespacing{1.5}
	\bsremovechaptertitle
	\title{Exponentialfunktionen Erkundung}
	\author{Julina Elfert \and Tanel Malak \and Ben Siebert \and Moritz Junkermann}
	\date{\today}
	\maketitle
	\tableofcontents
	\chapter{Ableitungsregeln für Produkte}
	\section{Teilaufgabe a)}
	\begin{tabularx}{\textwidth}{|X|X|X|}
	 \hline
	 $f(x) = u(x) \times v(x)$ & $u'(x) \times v(x)$ & $u(x) \times v'(x)$ \\
	 \hline
	 $x^6 = x^1 \times x ^5$ & $1 \times x^5 = x^5$ & $x^2 \times 5x^4 = 5x^5$ \\
	 \hline
	 $x^6 = x^2 \times x^4$ & $2x \times x^4 = 2x^5$ & $x^2 \times 4x^3 = 4x^5$ \\
	 \hline
	 $x^6 = x^3 \times x^3$ & $3x^2 \times x^3 = 3x^5$ & $x^3 \times 3x^2 = 3x^5$ \\
	 \hline
	 $x^6 = x^4 \times x^2$ & $4x^4 \times x^2 = 4x^5$ & $x^4 \times 2x = 2x^5$ \\
	 \hline
	 $x^6 = x^5 \times x^1$ & $5x^4 \times x^1 = 5x^5$ & $x^5 \times 1 = x^5$ \\
	 \hline
	\end{tabularx}
	\section{Teilaufgabe b)}\label{sec:b_rule}
	Die Formel lautet: $f'(x) = \Bigl(u'(x) \times v(x)\Bigl) + \Bigl(u(x) \times v'(x)\Bigl)$
	\section{Teilaufgabe c)}
	$f(x) = 3x^4 = 3x^2 \times x^2 = u(x) \times v(x)$ \\
	\paragraph{Anwendung der Regel (siehe \ref{sec:b_rule})} \mbox{} \\
	$
	6x \times x^2 + 3x^2 \times 2x \\
	= 6x^3 + 3x^3 \\
	= 12x^3 \\
	$
	Hierdurch ist bewiesen, dass die Regel (siehe \ref{sec:b_rule}) stimmt.
	
	
	\chapter{Produktregel-Zettel}
	
	\paragraph{Richtige Reihenfolge:} \mbox{} \\
	
	\begin{enumerate}
		\item $\frac{f(x_0 + h) - f(x_0)}{h}$
		\item $= \frac{u(x_0 + h) \times v(x_0 + h) - u(x_0) \times v(x_0)}{h}$
		\item $= \frac{(u(x_0 + h) - u(x_0)) \times v(x_0) + u(x_0 + h) \times (v(x_0 + h) - v(x_0))}{h}$
		\item $= \frac{u(x_0 + h) - u(x_0)}{h} \times v (x_0) + u(x_0 + h) \times \frac{v(x_0 +h) - v(x_0)}{h}$
	\end{enumerate}
	
	Für $h \to 0 = -u(x_0)*v(x_0)$
	
	\chapter{Funktionen in Funktionen}
	\section{Arbeitsauftrag A)}
	\subsection{Teilaufgabe 1)}
	Funktionen: $f(x) = x^2;\ g_1(x) = (x - 1)^2$ \\
	Die Nullstellen von $f$ und die der Ableitung $f'$ sind identisch.
	Man konnte sehen, dass $g$ identisch mit $f$ nur um eine Einheit verschoben ist.
	Die Ableitung von $g$ und von $f$ verlaufen parallel.
	Beide Ableitungen verlaufen durch die Scheitelpunkt der jeweiligen Funktion.
	\subsection{Teilaufgabe 2)}
	Auf Grund der in Teilaufgabe eins erarbeiteten Ergebnisse, ist die Ableitung von $g$ um $b$ nach unten verschoben.
	\section{Arbeitsauftrag B)}
	\subsection{Teilaufgabe 1)}
	Funktionen: $f(x) = x^2;\  h_1(x) = (2x)^2 = 4x^2$ \\
	Ableitungen: $f'(x) = 2x;\ h_1'(x) = 2x \times 4 = 8x$ \\
	Man erhält die Ableitung $h_1'(x)$ aus der Funktion $f'(x)$, indem man vier mit der Ableitung $h_1'(x)$ multipliziert.
	\subsection{Teilaufgabe 2)}
	Funktion: $h(x) = f(a \times x)$ \\
	Um den Ableitungsgraphen von $h(x)$ aus $f'(x)$ zu erhalten, muss man diesen um den Faktor $a$ multiplizieren (Strecken oder Stauchen).
	\section{Arbeitsauftrag C)}
	\paragraph{Allgemeine Regel} Die Ableitung einer Funktion $k(x) = f(ax - b)$ kann mit Hilfe des Ableitungsgraphen $f(x) = x^2$ erhalten werden, wenn dieser um $b$ nach unten verschoben und mit $a$ multipliziert wird.
	\chapter{Die Ableitung von Verkettungen untersuchen}
	\section{Teilaufgabe a)}
	\begin{tabularx}{\textwidth}{|X|X|}
	\hline
	\textbf{Schnipsel} & \textbf{Zusammenhang} \\
	\hline
	$u(v) = v^2;\ u'(v) = 2v$ & Ableitung \\
	\hline
	$v(x) = 3x + 1;\ v'(x) = 3$ & Ableitung \\
	\hline
	$f(x) = 9x^2 + 6x + 1;\ f(x) = (3x + 1)^2$ & Klammern aufgelöst \\
	\hline
	$f(x) = 9x^2 + 6x + 1;\ f(x) = (3x + 1)^2;\ f'(x) = 18x + 6$ & Ableitung von $f(x)$ \\
	\hline
	$u'(v(x)) = 2\times(3x + 1);\ v(x) = 3x + 1;\ u'(v) = 2v $ & Eingesetzt\\
	\hline
	$f(x) = (3x + 1)^2$ & $u(v(x))$ eingesetzt \\
	\hline
	\end{tabularx}
	\section{Teilaufgabe b)}
	Funktionen: $f(x) = (x + 2)^2;\ g(x) = (e^x - 1)^2$ \\
	$g(x) = (e^x - 1)^2 \Rightarrow g(x) = (e^x)^2  - e^x - e^x + 1 = (e^x)^2 - 2e^x + 1 = e^{2x} - 2e^x + 1$
	\begin{enumerate}
		\item 1
		\item 2
		\item $f'(x) = 2x + 4$
		\item $f(x) = x^2 + 4x + 4$
		\item $f(x) = (x + 2)^2$
		\item $g(x) = (e^x - 1)^2$
		\item 7
		\item 8
	\end{enumerate}
	\chapter{Testwagen}

\end{document}
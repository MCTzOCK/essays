\documentclass[12pt, a4paper]{report}
\usepackage[T1]{fontenc}
\usepackage[utf8]{inputenc}
\usepackage{ngerman}
\usepackage{bookman}
\usepackage[left=2cm,right=2cm,top=2cm,bottom=2cm]{geometry}
\usepackage{custompkg}
\usepackage{multicol}

\begin{document}
	\bslinespacing{1.5}
	\paragraph{Seite 101 Nr. 2} \mbox{} \\
	a)
	$\\
	P(0|4)\ \ Q(4|0,5) \\
	f(x) = c \times a^x\\
	c = 4 \\ $
	Nach a auflösen: \\
	$
	\Rightarrow 4 \times a^4 = 0,5 \\
	\Leftrightarrow a^4 = \frac{0.5}{4} \\
	\Leftrightarrow a^4 = 0.125 \\
	\Leftrightarrow a = \sqrt[4]{0.125} \\
	\Leftrightarrow a = 0.594604 \\
	\Rightarrow f(x) = 4 \times 0.59^x
	$ \\
	x für f(x)=256: \\
	$f(x) =256 \\ x = 8$ \\\\
	b)
	$\\
	P(0|0.25)\ \ Q(6|16)\\
	f(x) = c \times a^x \\
	c = 0.25$ \\
	Nach a auflösen: \\
	$
	\Rightarrow 0.25 \times a^6 = 16 \\
	\Leftrightarrow a^6 = \frac{16}{0.25} \\
	\Leftrightarrow a^6 = 64 \\
	\Leftrightarrow a = \sqrt[6]{64} \\
	\Leftrightarrow a = 2
	\Rightarrow f(x) = 0.25 \times 2^x
	$ \\\\
	c)
	$\\
	P(0|512)\ \ Q(-3|8) \\
	f(x) = c \times a^x \\
	c = 512 \\
	$
	Nach a auflösen: \\
	$
	\Rightarrow 512 \times a^{-3} = 8 \\
	\Leftrightarrow a^{-3} = \frac{512}{8} \\
	\Leftrightarrow a^{-3} = 64 \\
	\Leftrightarrow a = \sqrt[-3]{64} \\
	\Leftrightarrow a = \frac{1}{4}
	$

	
	
\end{document}
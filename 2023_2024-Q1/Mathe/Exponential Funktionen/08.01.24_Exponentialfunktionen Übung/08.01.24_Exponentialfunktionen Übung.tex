\documentclass[12pt, a4paper]{report}

\usepackage[T1]{fontenc}
\usepackage[utf8]{inputenc}
\usepackage{ngerman}
\usepackage{bookman}
\usepackage[left=2cm,top=2cm,right=2cm,bottom=2cm]{geometry}
\usepackage{custompkg}
\usepackage{amsmath}
\usepackage{hyperref}

\begin{document}
	\title{Exponentialfunktionen Übung}
	\date{\today}
	\author{Julina Elfert \and Tanel Malak \and Moritz Junkermann \and Ben Siebert}
	\maketitle
	\tableofcontents
	\bsremovechaptertitle
	\bslinespacing{1.5}
	\chapter{S. 113 Nr. 2}
	\textbf{a)} \\
	$f(x) = 45 \\
	x = 4.97729$
	\\
	Die Halbwertszeit des Baumes ist nach 4.97729 Jahren erreicht. \\
	\textbf{b)} \\
	$f(10) = 22.3581$ \\
	Nach 10 Jahren beträgt die Wachstumgsgeschwindigkeit 22.3581$\frac{cm}{Jahr}$
	\\
	\textbf{c)} \\
	$f(x) = 50$ \\
	$f(4.22072) = 50$ \\
	Die Wachstumsgeschwindigkeit beträgt nach 4.22072 Jahren 50 $\frac{cm}{Jahr}$
	\\
	\textbf{d)} \\
	$F(x) = 646.264 - 646.264 \times 0.87^x$ \\
	\textbf{e)} \\
	$
	\int_{0}^{10} f(x) dx = \Bigl[646.264 - 646.264 \times 0.87^x\Bigl]_{0}^{10} = F(10) - F(0) = 9.419095
	$ \\
	Nach 10 Jahren ist der Baum um 485.717cm gewachsen. \\
	\textbf{f)} \\
	$
	\int_{0}^{20} f(x) dx = \Bigl[646.264 - 646.264 \times 0.87^x\Bigl]_{0}^{20} = F(20) - F(0) = 606.38 \\
	 = 606.38cm + 90cm = 696.38cm
	$ \\
	Nach 20 Jahren ist der Baum 696.38cm groß. \\
	\textbf{g)} \\
	$
	\frac{f(x_2) - f(x_1)}{x_2 - x_1} = \frac{f(20) - f(0)}{20 - 0} = \frac{5.54 - 0.9}{20} = 0.232
	$ \\
	Der Baum ist im Durchschnitt um 23.2cm pro Jahr gewachsen.
	\chapter{S. 113 Nr. 3}
	\textbf{a)} \\
	$f(3) = 1382.4 \\ f(-3) = 462.963$ \\
	Drei Stunden vor Beginn der Messung waren es 462.963 Bakterien.
	Drei Stunden nach Beginn der Messung bereits 1382.4 Bakterien. \\
	\textbf{b)} \\
	$f(x) = 1600 \\ f(3.80178) = 1600$ \\
	Nach 3.80178 Stunden gibt es exakt 1600 Bakterien. \\
	\textbf{c)} \\
	$f'(x) = 145.857 \times 1.2^x$ \\
	\textbf{d)} \\
	$f'(5) = 362.94 \\ f'(0) = 145.857$ \\
	Zu Beginn der Messung lag die Steigung der Bakterienanzahl bei 145.857 Bakterien / Stunde.
	Nach 5 Stunden lag die Steigung der Bakterienanzahl bei 362.94 Bakterien / Stunde.\\
	\textbf{e)} \\
	$f'(x) = 1000 \\ f'(10.559) = 1000$ \\
	Nach 10.5999 Stunden liegt die Steigung der Bakterienanzahl bei exakt 1000.
	\chapter{S. 114 Nr. 6}
	\textbf{a)} \\
	$B(x) = 4000 \times 0.8^x$ \\
	\textbf{b)} \\
	$B(10) = 429.497$ \\
	In 10m Tiefe beträgt die Beleuchtungsstärke 429.497 Lux \\
	\textbf{c)} \\
	$B(x) = 2000 \\ B(3.10628) = 2000 \\$
	Die Halbwertstiefe ist nach 3.10628 Metern erreicht. \\
	\textbf{d)} \\
	$B'(x) = -892.574 \times 0.8^x$ \\
	\textbf{e)} \\
	$B'(x) = -10 \\ B'(20.1284) = -10$ \\
	Bei 20.1284 Metern beträgt die Änderungsrate der Beleuchtungsstärke -10 $\frac{Lux}{Meter}$.
\end{document}
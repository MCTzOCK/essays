\documentclass[12pt,a4paper]{report}
\usepackage[T1]{fontenc}
\usepackage[utf8]{inputenc}
\usepackage{charter}
\usepackage{ngerman}
\usepackage[left=2cm,right=2cm,top=2cm,bottom=2cm]{geometry}
\usepackage{amsmath}

\begin{document}
	\noindent
	\Large S. 161 Nr. 8
	\large
	\paragraph{a)}
	\begin{align*}
		\int_{0}^{2} &\frac{6x}{\sqrt{2 + 3x^2}}dx \\
		g(x) &= 2 + 3x^2;\ f(z) = \frac{1}{\sqrt{z}} \\
		g'(x) &= 6x \\
		\text{Umrechnung der Grenzen:}&\ \text{Aus der Grenze 0 wird die Grenze}\  g(0) = 2 \\
		&\ \text{Aus der Grenze 2 wird die Grenze}\ g(2) = 14 \\
		\int_{2}^{14} \frac{1}{\sqrt{z}} dz &= \Bigl[2\sqrt{z}\Bigl]_{2}^{14} = 2\sqrt{14} - 2\sqrt{2}
	\end{align*}
	\paragraph{b)}
	\begin{align*}
		\int_{-1}^{1} &\frac{-2x}{(4-3x^2)^2} dx \\
		g(x) &= 4-3x^2;\ f(z) = \frac{1}{z^2} \\
		g'(x) &= -6x \\
		\text{Umrechnung der Grenzen:}&\ \text{Aus der Grenze -1 wird die Grenze}\ g(-1) = 1 \\
		&\ \text{Aus der Grenze 1 wird die Grenze}\ g(1) = 1 \\
		\frac{1}{3} \int_{1}^{1} \frac{1}{z^2} dz &= \frac{1}{3}\Bigl[-z^{-1}\Bigl]_{1}^{1} = 0
	\end{align*}
	\paragraph{c)}
	\begin{align*}
		\int_{0}^{1} &x^2\cdot e^{x^3+1} dx \\
		g(x) &= x^3 + 1;\ f(z) = e^z \\
		g'(x) &= 3x^2 \\
		\text{Umrechnung der Grenzen:}&\ \text{Aus der Grenze 0 wird die Grenze}\ g(0)=1 \\
		&\ \text{Aus der Grenze 1 wird die Grenze}\ g(1) = 2 \\
		\frac{1}{3}\int_{1}^{2}& e^z dz = \frac{1}{3}\Bigl[e^z\Bigl]_{1}^{2} = \frac{1}{3} e^2 - \frac{1}{3}e = \frac{1}{3}\cdot (e^2 - e)
	\end{align*}
	\paragraph{d)}
	\begin{align*}
		\int_{0}^{1} &x\cdot \sin(x^2)dx \\
		g(x) &= x^2;\ f(z) = \sin(z) \\
		g'(x) &= 2x \\
		\text{Umrechnung der Grenzen:}&\ \text{Aus der Grenze 0 wird die Grenze}\ g(0) = 0 \\
		&\ \text{Aus der Grenze 1 wird die Grenze}\ g(1) = 1 \\
		\int_{0}^{1} \sin(z) dz = \frac{1}{2} \Bigl[-\cos(z)\Bigl]_{0}^{1} &=-\frac{1}{2}\cos(1) + \frac{1}{2}\cos(0)=-\frac{1}{2} \cdot (\cos(1) + 1)
	\end{align*}
	\noindent
	\Large S. 161 Nr. 10
	\large
	\paragraph{a)}
	\begin{align*}
		\int_{1}^{3}& \frac{10}{(3x+1)^2} dx = \frac{10}{3} \int_{1}^{3} \frac{1}{(3x + 1)^2} \cdot 3 dx \\
		g(x) &= 3x + 1;\ f(z) = \frac{1}{z^2} \\
		g'(x) &= 3 \\
		\text{Umrechnung der Grenzen:}&\ \text{Aus der Grenze 1 wird die Grenze}\ g(1) = 4 \\
		&\ \text{Aus der Grenze 3 wird die Grenze}\ g(3) = 10  \\
		\frac{10}{3} \int_{4}^{10} & \frac{1}{z^2} dz = \frac{10}{3} \cdot \Bigl[-z^{-1}\Bigl]_{4}^{10} = \frac{10}{3} (-10^{-1}) - \frac{10}{3} (-4^{-1}) \\
		&= -\frac{1}{3} + \frac{2,5}{3} =\frac{1,5}{3} = 0,5
	\end{align*}
\end{document}
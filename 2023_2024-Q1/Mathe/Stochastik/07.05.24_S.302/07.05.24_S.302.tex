\documentclass[12pt,a4paper]{report}
\usepackage[T1]{fontenc}
\usepackage[utf8]{inputenc}
\usepackage{charter}
\usepackage{ngerman}
\usepackage[left=2cm,right=2cm,top=2cm,bottom=2cm]{geometry}
\usepackage{amsmath}

\begin{document}
	\section{Seite 302 Nr. 6}
	\begin{align*}
		n &= 100 \\
		\alpha &= 5\ \% \approx 0.05 \\
		p &= \frac{33}{100} \\
		H_0 &\to p_2 = 40\ \% \\
		H_1 &\to p_2 \ne 40\ \% \\
		\mu &= n \cdot p_2 = 100 \cdot 0.4 \approx 40 \\
		\sigma &= \sqrt{n\cdot p\cdot (1-p)} \\
		&=\sqrt{100\cdot 0.4 \cdot (1-0.4)} \\
		&=\sqrt{40 \cdot 0.6} \\
		&=\sqrt{24}\\
		\text{Sigma-Bereich}: & \approx 1.96\sigma \\
		\text{Rechte-Grenze}: & \mu + 1.96\sigma = 40 + 1.96 \cdot \sqrt{24}\approx 49.6019997917 \\
		\text{Linke-Grenze}: & \mu - 1.96\sigma = 40 - 1.96 \cdot \sqrt{24} \approx 30.3980002083 \\
		\text{Annahmebereich}: & [31;50]
	\end{align*}
	Der Stimmenanteil ist zwar ein bisschen zurückgegangen, jedoch liegt er trotzdem im Annahmebereich.
	\section{S. 302 Nr. 7}
	\begin{align*}
		n&=100 \\
		\alpha &= 5\ \% \\
		H_0&\to p=30\% \\
		H_1&\to p\ne30\%\\
		\mu&= n\cdot p = 100\cdot 0.3 \approx 30 \\
		\sigma &= \sqrt{n\cdot p\cdot (1-p)} \\
		&=\sqrt{100\cdot 0.3\cdot (1-0.3)} \\
		&=\sqrt{30\cdot (0.7)} \\
		&=\sqrt{21} \\
		\text{Sigma-Bereich}:& \approx 1.96\sigma \\
		\text{Rechte-Grenze}:& \mu + 1.96\sigma = 30 + 1.96\sqrt{21} \approx 38.9818483621 \\
		\text{Linke-Grenze}:& \mu - 1.96\sigma = 30 - 1.96\sqrt{21} \approx 21.01815163789 \\
		\text{Anahmebereich}: & [21;39]
	\end{align*}
	Die Hypothese 30\ \% Walnüsse kann aufrecht erhalten werden, da die 30\ \% im Annahmebereich liegen.
\end{document}
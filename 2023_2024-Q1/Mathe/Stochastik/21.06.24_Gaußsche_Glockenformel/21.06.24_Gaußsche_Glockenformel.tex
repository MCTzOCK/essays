\documentclass[12pt,a4paper]{report}
\usepackage[T1]{fontenc}
\usepackage[utf8]{inputenc}
\usepackage{charter}
\usepackage{ngerman}
\usepackage[left=2cm,right=2cm,top=2cm,bottom=2cm]{geometry}
\usepackage{amsmath}

\renewcommand\thesection{\arabic{section}.} 

\begin{document}
	\section{Gauß'sche Glockenformel A1}
	\paragraph{a)}
	Achsensymmetrisch, wenn $\mu=0$: $f(-x) = f(x)$
	\paragraph{b)}
	Extremstelle (Hochpunkt) bei $x=\mu$. $y$ ist größer, wenn $\sigma$ kleiner wird. $y$ ist kleiner, wenn $\sigma$ größer wird.
	\paragraph{c)}	
	Die Wendestelle ist bei $x = \pm\ \sigma$ und um $\mu$ verschoben, also bei $\mu \pm \sigma$
	\section{Gauß'sche Glockenformel A2}
	\paragraph{a)}
	\begin{align*}
		\varphi_{\mu,\sigma} (X) &= \frac{1}{\sigma\sqrt{2\pi}} \cdot e^{-\frac{(x-\mu)^2}{2\sigma^2}} \\
		\mu &= 0 \\
		\sigma &= 1 \\
		\varphi_{\mu,\sigma} (X) &= \frac{1}{\sqrt{2\pi}} \cdot e^{-\frac{(x)^2}{2}} \\
	\end{align*}
	\paragraph{b)}
	\begin{itemize}
		\item Die Funktion kann durch eine Veränderung von $\mu$ verschoben wird.
		\item Die Funktion kann durch eine Veränderung von $\sigma$ gestreckt bzw. gestaucht werden.
	\end{itemize}
	\paragraph{c)}
	\begin{align*}
		\varphi(x) &\geq 0 \\
		\int_a^b \varphi(x) dx &= 1 \\
	\end{align*}
	Wenn $\sigma > 0 \to \sigma \cdot x \geq 0\ \land\ (x-\mu)^2 \geq 0$ ist die Funktion eine Wahrscheinlichkeitsdichte.
\end{document}
\documentclass[12pt,a4paper]{report}
\usepackage[T1]{fontenc}
\usepackage[utf8]{inputenc}
\usepackage{charter}
\usepackage{ngerman}
\usepackage[left=2cm,right=2cm,top=2cm,bottom=2cm]{geometry}

\begin{document}
	\noindent
	\Large
	Testaufgaben zur Stochastik 
	\large
	\begin{itemize}
		\item[6.] richtig
		\item[7.] Die höchste Säule eines Säulendiagramms muss nicht zwingend der Durchschnitt sein.
		\item[9.] richtig
		\item[10.] Wahrscheinlichkeiten sagen nicht direkt den Ausgang eines Zufallsexperiments vorher, da durch die Wahrscheinlichkeiten nicht exakt im Vorfeld bestimmt werden kann, welches Ergebnis eintritt.
		\item[11.] richtig
		\item[12.] Wahrscheinlichkeiten für gegenüberliegende Seiten des Quaders sind gleich groß, für relative Häufigkeiten gilt dies nur annähernd.
		\item[13.] Es kann streng genommen keinen wirklichen Laplacewürfel geben, da die Seiten eines Würfels auf Grund von Material verschiedene Wahrscheinlichkeiten haben können.
		\item[14.] Nein, die Wahrscheinlichkeit sechsmal die 6 in sechs Würfen zu werfen liegt bei $\frac{1}{46656}$ entspricht $(\frac{1}{6})^6$
		\item[15.] 1,2,3,4,5
		\item[16.] 
		\item[17.] falsch, bei absolut sicheren Ergebnissen liegt die Wahrscheinlichkeit bei 100 \%
		\item[18.] falsch
		\item[19.] Nein, das richtige Ergebnis lautet $\frac{125}{324}$
		\item[20.] Nein, das richtige Ergebnis lautet $\frac{1}{64}$
		\item[21.] Nein, die Teilwahrscheinlichkeiten müssen miteinander multipliziert werden
		\item[22.] Nein, das richtige Ergebnis lautet $\frac{3}{8}$, da die Ziehung mit dem Zurücklegen der Kugel abläuft.
		\item[23.] richtig
		\item[24.] 
	\end{itemize}
\end{document}
\documentclass[12pt,a4paper]{report}
\usepackage[T1]{fontenc}
\usepackage[utf8]{inputenc}
\usepackage{charter}
\usepackage{ngerman}
\usepackage[left=2cm,right=2cm,top=2cm,bottom=2cm]{geometry}
\usepackage{amsmath}

\begin{document}
	\paragraph{S. 292 Nr. 1} \mbox{} \\
	\begin{align*}
		a)&\ \ B_{15;0.03}(0) \approx 63.3251\ \% \\
		b)&\ \ F_{25;0.03}(25) - F_{25;0.03}(1) \approx 17.19\ \% \\
		c)&\ \ F_{50;0.03}(2) \approx 81.07\ \% \\
		d)&\ \ F_{100;0.03}(4) - F_{100;0.03}(1) \approx 62.32\ \%
	\end{align*}
	\paragraph{S. 292 Nr. 3} \mbox{} \\
	\begin{align*}
		a)&\ \ F_{100;0.95}(90) \approx 2.81\ \% \\
		b)&\ \ F_{100;0.85}(100) - F_{100;0.85}(89) \approx 9.94\ \% \\
	\end{align*}
	\paragraph{c)} Die Wahrscheinlichkeit für eine Gratis-Lieferung wird geringer, da durch die erhöhte Anzahl der Pfade die Gegenwahrscheinlichkeit deutlich gestiegen ist.
	\paragraph{d)}
	Folgendes wurde bezüglich des Wachstums der Zwiebeln nicht beachtet, wodurch das Wachstum der Zwiebeln als unabhängig angesehen wird:
	\begin{itemize}
		\item Wasserzugabe
		\item Licht
		\item Erde
		\item Mutterzwiebel
	\end{itemize}
\end{document}
\documentclass[12pt,a4paper]{report}
\usepackage[T1]{fontenc}
\usepackage[utf8]{inputenc}
\usepackage{charter}
\usepackage{ngerman}
\usepackage[left=2cm,right=2cm,top=2cm,bottom=2cm]{geometry}
\usepackage{amsmath}

\renewcommand\thesection{\arabic{section}.}

\begin{document}
	\tableofcontents
	\newpage
	\section{S. 292 Nr. 5a)}
	\begin{align*}
		P(X = 0) &\leq 0.05 \\
		p &= 0.25 \\
		P(X = 0) &= (1-p)^n \cdot p^0 \\
		P(X = 0) &= 0.75^n \cdot 0.25^0 \\
		&= 0.75^n \leq 0.05 \\
		\xrightarrow{CAS} n&= 11
	\end{align*}
	\section{S. 292 Nr. 5b)}
	\begin{align*}
		P(X \leq 1) &\leq 0.1 \\
		p &= 0.25 \\
		P(X \leq 1) &= (1-p)^n \cdot p^1 \\
		P(X \leq 1) &= 0.75^n \cdot 0.25^1 \\
		&= 0.25^1 \cdot 0.75^n \leq 0.1 \\
		\xrightarrow{CAS} n&= 15
	\end{align*}
	\section{S. 292 Nr. 5c)}
	\begin{align*}
		P(X = n) &\leq 0.01 \\
		p &= 0.25 \\
		P(X = n) &= (1-p)^n \cdot p^1 \\
		P(X = n) &= 0.75^n \cdot 0.25^n \\
		&= 0.25^n \cdot 0.75^n \leq 0.01 \\
		&= (0.75 \cdot 0.25)^n \leq 0.01 \\
		\xrightarrow{CAS} n&= 6\ \lor\ 7
	\end{align*}
	\section{S. 292 Nr. 5d)}
	\begin{align*}
		P(X \leq 2) &\leq 0.025 \\
		p &= 0.25 \\
		P(X \leq 2) &= (1-p)^n \cdot p^2 \\
		P(X \leq 2) &= 0.75^n \cdot 0.25^2 \\
		\xrightarrow{CAS} n&= 27
	\end{align*}
	\section{S. 293 Nr. 7a)}
	\begin{align*}
		p &= \frac{1}{6} \\
		P(X \geq 1) &\geq 0.99 \\
		P(X = 0) &\leq 0.01 \\
		P(X = 0) &= (\frac{1}{6})^0 \cdot (\frac{5}{6})^n \\
		(\frac{5}{6})^n &\leq 0.01 \\
		\xrightarrow{CAS} n &= 26
	\end{align*}
	\section{S. 293 Nr. 7b)}
	\begin{align*}
		p &= \frac{3}{6} \\
		P(X \geq 1) &\geq 0.99 \\
		P(X = 0) &\leq 0.01 \\
		P(X = 0) &= (\frac{3}{6})^0 \cdot (\frac{3}{6})^n \\
		(\frac{3}{6})^n &\leq 0.01 \\
		\xrightarrow{CAS} n &= 7
	\end{align*}
	\section{S. 293 Nr. 7c)}
	\begin{align*}
		p &= \frac{3}{6} \\
		P(X \geq 2) &\geq 0.99 \\
		P(X \leq 1) &\leq 0.01 \\
		P(X \leq 1) &= (\frac{3}{6})^1 \cdot (\frac{3}{6})^n \\
		(\frac{3}{6})^n &\leq 0.01 \\
		\xrightarrow{CAS} n &= 11
	\end{align*}
	\section{S. 293 Nr. 7d)}
	\begin{align*}
		p &= \frac{5}{6} \\
		P(X \geq 3) &\geq 0.99 \\
		P(X \leq 2) &\leq 0.01 \\
		P(X \leq 2) &= (\frac{5}{6})^2 \cdot (\frac{1}{6})^n \\
		(\frac{1}{6})^n &\leq 0.01 \\
		\xrightarrow{CAS} n &= 6
	\end{align*}
	\section{S. 293 Nr. 9a)}
	\begin{align*}
		P(X \leq 2) &= 54.053\ \%
	\end{align*}
	\section{S. 293 Nr. 9b)}
	Wie groß ist die Wahrscheinlichkeit, dass von 50 befragten Fahrgästen genau 2 Fahrgäste unzufrieden sind?
	\section{S. 293 Nr. 9c)}
	\begin{align*}
		p &= 0.05 \\
		P(X \geq 1) &\geq 0.9 \\
		P(X = 0) &\leq 0.1 \\
		P(X = 0) &= 0.05^{n} \cdot 0.9^0 \\
		0.05^n &\leq 0.1 \\
		\xrightarrow{CAS} &\ n = 77
	\end{align*}
	\section{S. 293 Nr. 9d)}
	\begin{align*}
		p &= 0.05 \\
		P(X \geq 2) &\geq 0.9 \\
		P(X \leq 1) &\leq 0.1 \\
		\xrightarrow{CAS} &\ n = 105
	\end{align*}
	\section{S. 293 Nr. 9e)}
	\begin{align*}
		P(X \leq 1) &= \binom{100}{1}
	\end{align*}
	
\end{document}
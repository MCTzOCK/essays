\documentclass[12pt,a4paper]{report}
\usepackage[T1]{fontenc}
\usepackage[utf8]{inputenc}
\usepackage{charter}
\usepackage{ngerman}
\usepackage[left=2cm,right=2cm,top=2cm,bottom=2cm]{geometry}
\usepackage{amsmath}

\renewcommand\thesection{\arabic{section}.} 

\begin{document}
	\section{Teilaufgabe d)}
	\paragraph{(1)}
	\begin{align*}
		n &= 200 \\
		\alpha &= 5\ \% = 0.05 \\
		H_0 &: p=0.08 \\
		H_1 &: p<0.08 \\
		P(X \leq k) &\leq 0.05 \\
		\xrightarrow{CAS} & k=9 \\
		&\bar{A}[0;9]
	\end{align*}
	Wenn 9 oder weniger Schrauben fehlerhaft sind, hat \dq Wind24\dq\ nicht recht.
	\paragraph{(2)}
	Bei 10-200 fehlerhaften Schrauben hat \dq Wind24\dq\ rect.
	\begin{align*}
		n &= 200 \\
		p &= 0.045 \\
		P(X \geq 10) &= 0.4125 \\
		&= 41.25\ \%
	\end{align*}
	Mit einer Wahrscheinlichkeit von 41,25\ \% weist die Firma die Vorwürfe fehlerhafter Weise nicht zurück.
\end{document}
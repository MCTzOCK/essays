\documentclass[12pt,a4paper]{report}
\usepackage[T1]{fontenc}
\usepackage[utf8]{inputenc}
\usepackage{charter}
\usepackage{ngerman}
\usepackage[left=2cm,right=2cm,top=2cm,bottom=2cm]{geometry}
\usepackage{amsmath}
\usepackage{tabularx}

\begin{document}
	\noindent
	\Large
	Abituraufgabe Stochastik 2021
	\large
	\paragraph{a2)} \mbox{} \\
	\begin{tabularx}{0.74\textwidth}{|X|X|X|X|}
		\hline
		& $B^+$ & $B^-$ & \\
		\hline
		$K^+$ & $0.97 \cdot 0.985 \approx 0.95$& $0.97 \cdot 0.015 \approx 0.02$& 0,97 \\
		\hline
		$K^-$ & $0.03\cdot 0.95 \approx 0.029$ &$0.03\cdot 0.05 \approx 0.001$& 0,03 \\
		\hline
		&$0.979$&$0.021$& 1 \\
		\hline
	\end{tabularx}
	\begin{align*}
		P_{B^-} (K-) &= \frac{P(K^-\cap B^-)}{P(B^-)} \\
	\end{align*}
	\paragraph{b)}	
	\begin{align*}
		1)&\ B_{500;0.049} (15) = 0.024 \approx 2,4\% \\
		2)&\ 1-F_{n;0.955}(199) \geq 0.95 \\
		&\ \xrightarrow{CAS} n = 215
	\end{align*}
	\paragraph{c)}
	\begin{align*}
		(1)&\ \mu = n \cdot p \\
		&\ \Rightarrow 5000 = n \cdot 0.955 \\
		&\ \Leftrightarrow n = 5235,6 \approx 5236 \\
		(2)&\ P(X \geq 5000) \\
		&\ \Leftrightarrow 1-F_{5236;0.955}(5000) \\
		&\ \xrightarrow{CAS} P\approx 0.5274 = 52.74\ \%
	\end{align*}
\end{document}
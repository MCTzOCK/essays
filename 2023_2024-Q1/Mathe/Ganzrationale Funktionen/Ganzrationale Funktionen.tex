\documentclass[12pt,a4paper]{report}
\usepackage[T1]{fontenc}
\usepackage[utf8]{inputenc}
\usepackage{charter}
\usepackage{ngerman}
\usepackage[left=2cm,right=2cm,top=2cm,bottom=2cm]{geometry}
\usepackage{amsmath}
\usepackage{pgfplots}
\usepackage{tikz}
\usepackage{tcolorbox}
\tcbuselibrary{skins,breakable}
\usetikzlibrary{shadings,shadows}

\newenvironment{gblock}[1]{
    \tcolorbox[beamer,
        noparskip,
        colback=green!50!,
        colbacklower=green!75!green,
        title=#1]}
{\endtcolorbox}


\begin{document}
	\noindent
	\Large
	Ganzrationale Funktionen
	\large
	\\
	\begin{gblock}{Übersicht}
		Eine ganzrationale Funktion ist eine Summe aus mehreren Potenzfunktionen, die mit natürlichen Exponenten beschrieben werden können.
		\paragraph{Beispiel:} \mbox{}
		\begin{align*}
			f(x) = 8x^5 + 10x^4 - 5x^2 + 4x + 1
		\end{align*}
		\begin{center}
			\begin{tikzpicture}
				\begin{axis}[
					axis lines = middle,
					xlabel = $x$,
					ylabel = $f(x)$,
				]
				\addplot[domain=-3:3, samples=100, color=red]{8*x^5 + 10*x^4 - 5*x^2 + 4*x + 1};
				\end{axis}
			\end{tikzpicture}
		\end{center}
	\end{gblock}
	
	\begin{gblock}{Eigenschaften}
	Ganzrationale Funktionen haben verschiedene Eigenschaften, die von den Potenzen der Funktion abhängen.
	\paragraph{Grad:} Der Grad einer ganzrationalen Funktion ist der höchste Exponent, mit dem die Variable x vorkommt. Er bestimmt das allgemeine Verhalten der Funktion.
	\paragraph{Nullstellen:} Die Nullstellen einer Funktion sind die x-Werte, für die die Funktion den Wert Null annimmt.
	\paragraph{Graph:} Der Graph einer ganzrationalen Funktion ist immer eine glatte Kurve ohne Ecken oder Spitzen.
	\paragraph{Verhalten nahe Null:} Der Graph einer Funktion $f(x)$ verhält sich nahe Null wie der Graph der Funktion $g(x)$ mit $g(x)$ als der kleinsten Potenz von $x$ in der Ursprungsfunktion.
\end{gblock}


\end{document}
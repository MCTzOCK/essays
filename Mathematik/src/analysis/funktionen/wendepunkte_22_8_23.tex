%! Author = ben
%! Date = 18.02.2024

% Preamble
\documentclass[arbeitsmappe.tex]{subfiles}

% Document
\begin{document}

    \begin{gblock}{Wendepunkte}
        Der Punkt einer Funktion mit der größten Steigung wird \textbf{Wendepunkt} genannt.

        \paragraph{Vorgehen zur Bestimmung von Wendepunkten}
        \begin{enumerate}
            \item Notwendige Bedingung für Wendepunkte: $f''(t) = 0$
            \item Hinreichende Bedingung für Wendepunkte: $f''(t) = 0\ \land\ f'''(t) \ne 0$
        \end{enumerate}
    \end{gblock}

    \subsection{S. 25 Nr. 2}
    \begin{rblock}{Aufgabe}
        Bestimmen Sie den Wendepunkt des Graphen von f sowie die Gleichung der Wendetangente.

        \paragraph{a)} $f(x) = 0.5x^3 - 3x^2 + 5x$

        \paragraph{e)} $f(x) = -x^3 - 3x^2 + 4x + 4$
    \end{rblock}

    \paragraph{a)}
    $
    f(x) = 0.5x^3 - 3x^2 + 5x \\
    f'(x) = 1.5x^2 - 6x + 5 \\
    f''(x) = 3x - 6 \\
    $
    notwendige Bedingung für Wendepunkte: $f''(x) = 0$ \\
    $
    f''(x) = 0 \\
    \Leftrightarrow 3x - 6 = 0 \\
    \Leftrightarrow x_1 = 2 \\
    $
    hinreichende Bedingung für Wendepunkte: $f''(x) = 0\ \land\ VZW$ \\
    $
    \left.
    \begin{array}{l}
        f''(1) = -3 \\
        f''(3) = 3
    \end{array}
    \right\} \text{+ / - VZW} \Rightarrow \text{Wendepunkt}
    $
    \\
    Wendetangente:
    $
    x = 2 \\
    t(x) = m \times x + b \\
    m = f'(x) = f'(2) \\
    f'(2) = 1.5 \times 2^2 - 6 \times 2 + 5 = -1 \\
    t(x) = -x + b \\
    \Leftrightarrow 2 = -1 \times 2 + b \\
    \Leftrightarrow -b = -2 -2 \\
    \Leftrightarrow b = 4 \\
    \Rightarrow t(x) = -x + 4
    $
    \\

    \paragraph{e)}
    $
    f(x) = -x^3 - 3x^2 + 4x + 4 \\
    f'(x) = -3x^2 - 6x + 4 \\
    f''(x) = -6x - 6 \\
    $
    notwendige Bedingung für Wendepunkte: $f''(x) = 0$ \\
    $
    f''(x) = 0 \\
    \Leftrightarrow -6x - 6 = 0 \\
    \Leftrightarrow x_1 = -1 \\
    $
    hinreichende Bedingung für Wendepunkte: $f''(x) = 0\ \land\ VZW$ \\
    $
    \left.
    \begin{array}{l}
        f''(-2) = 6 \\
        f''(0) = -6
    \end{array}
    \right\} \text{+ / - VZW} \Rightarrow \text{Wendepunkt}
    $
    \\
    Wendetangente:
    $
    x = -1 \\
    t(x) = m \times x + b \\
    m = f'(x) = f'(-1) \\
    f'(-1) = -3 \times (-1)^2 - 6 \times (-1) + 4 = 7 \\
    t(x) = 7x + b \\
    \Leftrightarrow -2 = 7 \times (-1) + b \\
    \Leftrightarrow 0 = -7 + b + 2 \\
    \Leftrightarrow b = 5 \\
    \Rightarrow t(x) = 7x + 5
    $
    \\
    \newpage

    \subsection{S. 25 Nr. 3}
    \begin{rblock}{Aufgabe}
        Bestimmen Sie die Wendepunkte des Graphen der Funktion f.
        Geben Sie anschließend die Intervalle an, in denen der Graph links- bzw. rechtsgekrümmt ist.

        \paragraph{a)} $f(x) = x^4 + x^2$

        \paragraph{g)} $f(x) = \frac{1}{60}x^6 - \frac{1}{10}x^5 + \frac{1}{6}x^4$
    \end{rblock}

    \paragraph{a)}
    $
    f(x) = x^4 + x^2 \\
    f'(x) = 4x^3 + 2x \\
    f''(x) = 12x^2 + 2 \\
    $
    notwendige Bedingung für Wendepunkte: $f''(x) = 0$ \\
    $
    f''(x) = 0 \\
    \xrightarrow{CAS} \text{keine reellen L\"osungen}
    $
    \\
    Krümmungsverhalten: \\
    $
    I[-\infty; +\infty] \to \text{linksgekrümmt}
    $

    \paragraph{g)}
    $
    f(x) = \frac{1}{60}x^6 - \frac{1}{10}x^5 + \frac{1}{6}x^4 \\
    f'(x) = \frac{x^5}{10} - \frac{x^4}{2} + \frac{2x^3}{2} \\
    f''(x) = \frac{x^4}{2} - 2x^3 + 2x^2 \\
    $
    notwendige Bedingung für Wendepunkte: $f''(x) = 0$ \\
    $
    f''(x) = 0 \\
    \xrightarrow{CAS} x_1 = 0\ \land\ x_2 = 2 \\
    $
    hinreichende Bedingung für Wendepunkte: $f''(x) = 0\ \land\ VZW$ \\
    $
    \left.
    x_1:
    \begin{array}{ll}
        f''(-1) = \frac{9}{2} \\
        f''(2) = 0            \\
    \end{array}
    \right\} \text{kein VZW}
    $
    \\
    $
    \left.
    x_2:
    \begin{array}{ll}
        f''(1) = \frac{1}{2} \\
        f''(3) = \frac{9}{2} \\
    \end{array}
    \right\} \text{kein VZW}
    $
    \\
    Krümmungsverhalten: \\
    $
    I[-\infty; +\infty] \to \text{linksgekrümmt} \\
    $
    \newpage

    \subsection{S. 25 Nr. 5}
    \begin{rblock}{Aufgabe}
        Forscher haben das Wachstum einer bestimmten Bakterienkultur in einer Petrischale beobachtet.
        Die von Bakterien bedeckte Fläche (in $cm^2$) in Abhängigkeit der vergangenen Zeit (in h) seit
        dem Beobachtungsbeginn um 8 Uhr morgens kann im Zeitraum von 8 Uhr morgens bis 12 Uhr mittags des
        darauf folgenden Tages nährungesweise durch die Funktion A mit $A(t) = -0.005t^3+0.2t^2+0.9t+1$
        beschrieben werden.

        \paragraph{a)} Bestimmen Sie die von Bakterien bedeckte Fläche um 2 Uhr morgens.

        \paragraph{b)} Bestimmen Sie die maximale Zunahme der von Bakterien bedeckten Fläche.
    \end{rblock}

    \paragraph{a)}
    $
    t = 19 \\
    A(19) \xrightarrow{CAS} 56.005
    $ \\
    Um 3 Uhr morgens bedecken die Bakterien eine Fläche von etwa $56.005\ cm^2$.

    \paragraph{b)}
    $
    A'(t) = -0.015t^2 + 0.4t + 0.9 \\
    A''(t) = 0.4 - 0.03t \\
    $
    notwendige Bedingung für Wendepunkte: $A''(t) = 0$ \\
    $
    0.4 - 0.03t = 0 \\
    \xrightarrow{CAS} t_1 = 13.333 \\
    $
    hinreichende Bedingung für Wendepunkte: $A''(t) = 0\ \land\ VZW$ \\
    $
    \left.
        t_1:
        \begin{array}{l}
            A''(13) = 0.4 - 0.03 \times 13 = 0.01 \\
            A''(14) = 0.4 - 0.03 \times 14 = -0.02
        \end{array}
    \right\} \text{+ / - VZW} \Rightarrow \text{Hochpunkt}
    $
    \\
    maximale Zunahme der Fläche: \\
    $
    a'(t) = a'(13.333) \approx 3.56cm^2
    $
    \\
    Randverhalten:
    \\
    $
    A'(0) = 0.9 \\
    A'(28) = 0.34 \\
    $
    Die maximale Zunahme ist nach 13,33 Stunden mit $3.56 \frac{cm^2}{h}$ erreicht.


\end{document}

%! Author = ben
%! Date = 18.02.2024

% Preamble
\documentclass[arbeitsmappe.tex]{subfiles}

% Document
\begin{document}

    \begin{gblock}{Einleitung}
        Der Graph der Funktion f mit $f(t) = -\frac{1}{3000}t^3 + \frac{3}{20}t^2$ soll für $0 \leq t \leq 300$ verwendet werden,
        um näherungsweise die zurückgelegte Strecke in Metern einer S-Bahn zwischen zwei Haltestellen zu beschreiben (t beschreibt die Zeit in Sekunden).
    \end{gblock}

    \begin{rblock}{Aufgabe}
        Bestimme die maximale Geschwindigkeit der S-Bahn und beschreibe dein Vorgehen.
    \end{rblock}

    $
    f'(t) = \frac{3t}{10} - \frac{t^2}{1000} \\
    f''(t) = \frac{3}{10} - \frac{t}{500} \\
    f'''(t) = -\frac{1}{500}
    $
    \\
    notwendige Bedingung für EST: $f''(t) = 0$
    \\
    $
    f''(t) = 0 \\
    \frac{3}{10} - \frac{t}{500} = 0 \\
    \xrightarrow{CAS} t_1 = 150
    $
    \\
    hinreichende Bedingung für EST: $f''(t) = 0\ \land VZW$
    \\
    $
    \left.
    \begin{array}{l}
        f''(149) = \frac{1}{500} \\
        f''(151) = -\frac{1}{500}
    \end{array}
    \right\} \text{+ / - VZW} \Rightarrow \text{Maximum}
    $
    \\
    $
    f'(150) = \frac{45}{2} \approx 22.5 \frac{m}{s} \\
    \frac{km}{h} \to 22.5\frac{m}{s} \times 3.6 = 81\frac{km}{h}
    $
    \\
    Die Höchstgeschwindigkeit der S-Bahn beträgt $81 \frac{km}{h}$. Diese erreicht sie nach 150 Sekunden bzw. nach $2.5$ Minuten.


\end{document}

\documentclass[a4paper, 12pt]{report}

\usepackage[T1]{fontenc}
\usepackage[utf8]{inputenc}
\usepackage{ngerman}
\usepackage{charter}
\usepackage{titlesec}
\usepackage{tabularx}
\usepackage[left=2cm,right=2cm,top=2cm,bottom=2cm]{geometry}

\begin{document}
    % line spacing 1.5
    \renewcommand{\baselinestretch}{1.5}
    % remove chapter title
    \titleformat{\chapter}{\normalfont\LARGE}{\thechapter.}{15pt}{\Large}
    \titlespacing*{\chapter}{0pt}{*0}{10pt}
    \chapter*{Die String Klasse in Java}
    Die String Klasse in Java ist eine Klasse, die es ermöglicht, Strings zu erstellen und zu manipulieren.
    Sie ist eine Klasse, die in Java bereits vordefiniert ist und muss nicht importiert werden.\\
    \textbf{Beispiel:}
    \begin{verbatim}
    String s = "Hallo Welt";
    \end{verbatim}
    Man kann Strings auch mit dem \texttt{new} Operator erstellen, dies ist aber nicht zu empfehlen, da es zu Problemen mit dem \texttt{==} Operator kommen kann.\\
    \textbf{Beispiel:}
    \begin{verbatim}
    String s = new String("Hallo Welt");
    \end{verbatim}
    Um zwei Strings zusammen zufügen, kann der \texttt{+} Operator verwendet werden.\\
    \textbf{Beispiel:}
    \begin{verbatim}
    String s = "Hallo" + "Welt";
    \end{verbatim}


    \section*{Methoden}

    \begin{tabularx}{\textwidth}{|X|X|}
        \hline
        \textbf{Methode}                                     & \textbf{Beschreibung}                                                                                                                     \\
        \hline
        \texttt{length()}                                    & Gibt die Länge des Strings zurück.                                                                                                        \\
        \hline
        \texttt{charAt(int index)}                           & Gibt das Zeichen an der Stelle \texttt{index} zurück.                                                                                     \\
        \hline
        \texttt{substring(int beginIndex, int endIndex)}     & Gibt den Teilstring von \texttt{beginIndex} bis \texttt{endIndex} zurück.                                                                 \\
        \hline
        \texttt{substring(int beginIndex)}                   & Gibt den Teilstring von \texttt{beginIndex} bis zum Ende des Strings zurück.                                                              \\
        \hline
        \texttt{indexOf(String s)}                           & Gibt den Index des ersten Vorkommens von \texttt{s} zurück.                                                                               \\
        \hline
        \texttt{lastIndexOf(String s)}                       & Gibt den Index des letzten Vorkommens von \texttt{s} zurück.                                                                              \\
        \hline
        \texttt{startsWith(String s)}                        & Gibt \texttt{true} zurück, wenn der String mit \texttt{s} beginnt.                                                                        \\
        \hline
        \texttt{endsWith(String s)}                          & Gibt \texttt{true} zurück, wenn der String mit \texttt{s} endet.                                                                          \\
        \hline
        \texttt{contains(String s)}                          & Gibt \texttt{true} zurück, wenn der String \texttt{s} enthält.                                                                            \\
        \hline
        \texttt{equals(String s)}                            & Gibt \texttt{true} zurück, wenn der String gleich \texttt{s} ist.                                                                         \\
        \hline
        \texttt{equalsIgnoreCase(String s)}                  & Gibt \texttt{true} zurück, wenn der String gleich \texttt{s} ist, ohne auf Groß- und Kleinschreibung zu achten.                           \\
        \hline
        \texttt{compareTo(String s)}                         & Gibt einen Wert kleiner als 0 zurück, wenn der String lexikographisch vor \texttt{s} kommt.                                               \\
        \hline
        \texttt{compareToIgnoreCase(String s)}               & Gibt einen Wert kleiner als 0 zurück, wenn der String lexikographisch vor \texttt{s} kommt, ohne auf Groß- und Kleinschreibung zu achten. \\
        \hline
        \texttt{toUpperCase()}                               & Gibt den String in Großbuchstaben zurück.                                                                                                 \\
        \hline
        \texttt{toLowerCase()}                               & Gibt den String in Kleinbuchstaben zurück.                                                                                                \\
        \hline
        \texttt{trim()}                                      & Gibt den String ohne Leerzeichen am Anfang und Ende zurück.                                                                               \\
        \hline
        \texttt{replace(String oldString, String newString)} & Gibt den String zurück, in dem alle Vorkommen von \texttt{oldString} durch \texttt{newString} ersetzt wurden.                             \\
        \hline
        \texttt{split(String regex)}                         & Gibt ein String Array zurück, das den String an den Stellen, an denen \texttt{regex} vorkommt, aufteilt.                                  \\
        \hline
        \texttt{startsWith(String prefix)}                   & Gibt \texttt{true} zurück, wenn der String mit \texttt{prefix} beginnt.                                                                   \\
        \hline
        \texttt{endsWith(String suffix)}                     & Gibt \texttt{true} zurück, wenn der String mit \texttt{suffix} endet.                                                                     \\
        \hline
        \texttt{valueOf(int i)}                              & Gibt den String zurück, der die Zahl \texttt{i} repräsentiert.                                                                            \\
        \hline
    \end{tabularx}

\end{document}

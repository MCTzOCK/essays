%! Author = ben
%! Date = 06.01.2024

% Preamble
\documentclass[25pt, a3paper, portrait]{tikzposter}

\title{Funktionsuntersuchung}
\author{Ben Siebert}
\date{\today}
\institute{Gymnasium Holthausen Hattingen}

\usetheme{Rays}
\tikzposterlatexaffectionproofoff


% Packages
\usepackage{ngerman}
\usepackage{charter}
\usepackage[fleqn]{mathtools}

% Document
\begin{document}

\maketitle
\begin{columns}
    \column{0.2}
    \block{Einleitung}
    {Bei der Funktionsuntersuchung geht es darum, eine Funktion zu analysieren.
        Dazu gehören alle wichtigen Eigenschaften einer Funktion.
    }
    \column{0.4}
    \block{Nullstellen}
    {
        Um die Nullstellen einer Funktion zu bestimmen, musst du die Funktion gleich Null setzen: $f(x) = 0$.
        Anschließend musst du die Gleichung nach x auflösen. \\
        \textbf{Beispiel:}
        \begin{align*}
            f(x)           & = x^2 - 4x + 3    \\
            f(x)           & = 0               \\
            x^2 - 4x + 3   & = 0               \\
            (x - 3)(x - 1) & = 0               \\
            x_1            & = 3 \quad x_2 = 1
        \end{align*}
    }
    \column{0.4}
    \block{Symmetrie}
    {
        Es gibt zwei verschiedene Arten von Symmetrie: Achsensymmetrie und Punktsymmetrie. \\
        Achsensymmetrie bedeutet, dass die Funktion an der y-Achse gespiegelt werden kann.
        Punktsymmetrie bedeutet, dass die Funktion an einem Punkt (meist der Urpsung) gespiegelt werden kann. \\
        \textbf{Achsensymmetrie:} \\
        Eine Funktion ist achsensymmetrisch, wenn sie die Gleichung $f(x) = f(-x)$ erfüllt. \\
        \textbf{Punktsymmetrie:} \\
        Eine Funktion ist punktsymmetrisch, wenn sie die Gleichung $f(x) = -f(-x)$ erfüllt. \\
    }
\end{columns}
\begin{columns}
    \column{0.4}
    \block{Randverhalten}
    {
        Das Randverhalten einer Funktion beschreibt, wie sich die Funktion für sehr große und sehr kleine x-Werte verhält. \\
        \textbf{Beispiel:}
        \begin{align*}
            f(x)                      & = x^2 - 4x + 3 \\
            \lim_{x \to \infty} f(x)  & = \infty       \\
            \lim_{x \to -\infty} f(x) & = \infty
        \end{align*}
        Man kann also sagen, dass die Funktion für sehr große und sehr kleine x-Werte gegen unendlich geht.
    }
    \column{0.6}
    \block{Verhalten nahe Null}
    {
        Nahe Null verhält sich eine Funktion $f(x)$ wie der Graph der Funktion $f(x) = ax^n$ mit $n$ als der niedrigsten Potenz von x der Ursprungsfunktion. \\
        \textbf{Beispiel:}
        \begin{align*}
            f(x) & = \frac{1}{4}x^4 + 3x^2 - 2 \\
            g(x) & = 3x^2                      \\
        \end{align*}
        Hier ist $g(x)$ die Funktion, die das Verhalten von $f(x)$ nahe Null beschreibt.
    }
\end{columns}
\begin{columns}
    \column{0.5}
    \block{Extremstellen}
    {
        Um die Extremstellen einer Funktion zu bestimmen benötigst du zunächst die erste Ableitung der Funktion.
        Anschließend musst du die erste Ableitung gleich Null setzen und nach x auflösen.
        Die x-Werte, die du erhältst, sind die x-Werte der Extremstellen. \\
        Zuletzt musst du noch die zweite Ableitung der Funktion an den x-Werten der Extremstellen berechnen.
        Ist die zweite Ableitung an einer Extremstelle positiv, so handelt es sich um einen Tiefpunkt.
        Ist die zweite Ableitung an einer Extremstelle negativ, so handelt es sich um einen Hochpunkt. \\
        \textbf{Beispiel}: \\\\
        Funktion: $f(x) = 2x^2$\\
        Erste Ableitung: $f'(x) = 4x$\\
        Zweite Ableitung: $f''(x) = 4$\\\\
        \textbf{notwendige Bedingung für EST: $f'(x) = 0$}\\
        $
            f'(x) = 0 \\
            4x = 0 \\
            x_1 = 0
        $
        \\\\
        \textbf{hinreichende Bedingung für EST: $f'(x) = 0 \land f''(x) \ne 0$}\\
        $
            f''(x_1) = 4 \ne 0 > 0 \Rightarrow \text{Hochpunkt bei } x_1 = 0
        $
    }

    \column{0.5}
    \block{Wendestellen}
    {
        Um die Wendestellen einer Funktion zu bestimmen benötigst du zunächst die zweite Ableitung der Funktion.
        Anschließend musst du die zweite Ableitung gleich Null setzen und nach x auflösen.
        Die x-Werte, die du erhältst, sind die x-Werte der Wendestellen. \\
        Zuletzt musst du noch die dritte Ableitung der Funktion an den x-Werten der Extremstellen berechnen.
        Ist die dritte Ableitung an einer berechneten $x$-Koordinate ungleich Null, so handelt es sich um eine Wendestelle.
        \textbf{Beispiel}: \\
        Funktion: $f(x) = 4x^3$\\
        Zweite Ableitung: $f''(x) = 24x$\\
        Dritte Ableitung: $f'''(x) = 24$\\\\
        \textbf{notwendige Bedingung für WDS: $f''(x) = 0$}\\
        $
            f''(x) = 0 \\
            24x = 0 \\
            x_1 = 0
        $
        \\
        \textbf{hinreichende Bedingung für WDS: $f''(x) = 0 \land f'''(x) \ne 0$}\\
        $
            f'''(x_1) = 24 \ne 0 \Rightarrow \text{Wendestelle bei } x_1 = 0
        $
    }
\end{columns}

\end{document}

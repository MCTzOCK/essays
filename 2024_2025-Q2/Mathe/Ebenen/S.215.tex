\documentclass[12pt,a4paper]{report}
\usepackage[T1]{fontenc}
\usepackage[utf8]{inputenc}
\usepackage{charter}
\usepackage{ngerman}
\usepackage[left=2cm,right=2cm,top=2cm,bottom=2cm]{geometry}
\usepackage{amsmath}
\usepackage{tabularx}
\renewcommand\thesection{\arabic{section}.} 

\newcommand{\vek}[1]{\begin{pmatrix}{#1}\end{pmatrix}}


\begin{document}
	\section*{S. 215 Nr. 4}
	\begin{align*}
		E: \overrightarrow x &= \vek{3 \\ 0 \\ 2} + r \cdot \vek{2 \\ 1 \\ 7} + s \cdot \vek{3 \\ 2 \\ 5} \\\\
		A&(8|3|14) \\
		B&(1|1|0) \\
		C&(4|0|11) \\
		\\
		I. & 8 = 3 + 2r + 3s \\
		II. & 3 =r+2s \\
		III. & 14 = 2 + 7r + 5s \\
		&III. \cdot 3 - I. \cdot 2 \\\\
		I.& 8 = 3 + 2r + 3s \\
		II.& 3 =r+2s \\
		III.& 26 = 17r + 9s \\
		&III. -17\cdot II \\\\
		I.& 8 = 3 + 2r + 3s \\
		II.& 3 =r+2s \\
		III. &-25 = -25s \\
		\Leftrightarrow &s = 1 \\\\
		I.& 8 = 3 + 2r + 3 \\
		II.& 3 =r+2 \\
		III. &1 = s \\\\
		II.& r=1 \\
		I.&1=1
	\end{align*}
	Der Punkt $A$ liegt auf der Ebene. \newpage
	
	\begin{tabularx}{\textwidth}{|x|X|}
		\hline
		I. & $1 = 3 + 2r + 3s$ \\
		II. & $1 = r + 2s$ \\
		III. & $0 = 2 + 7r +5s$\\
		\hline
		& III. $\cdot$ 3 - I. $\cdot$ 2 \\
		\hline
		I. & $1 = 3 + 2r + 3s$ \\
		II. & $1 = r + 2s$ \\
		III. & $-2 = 17r +9s$\\
		\hline
		& III. - $17 \cdot$ II. \\
		\hline
		I. & $1 = 3 + 2r + 3s$ \\
		II. & $1 = r + 2s$ \\
		III. & $-19 = -25s$\\
		& $\Leftrightarrow$ s = $\frac{19}{25}$ \\
		\hline
		I. & $1 = 3 + 2r + 3 \cdot \frac{19}{25} $ \\
		& $\Leftrightarrow 1 = 3 + 2r + \frac{57}{25}$ \\
		II. & $1 = r + 2 \cdot \frac{19}{25} $ \\
		& $\Leftrightarrow 1 = r + \frac{38}{25}$ \\
		III. & $s = \frac{19}{25} $\\
		\hline
		& II. - $\frac{38}{57}$\\
		\hline
		I. & $1 = 3 + 2r + \frac{57}{25}$ \\
		II. & $\frac{19}{25} = r$ \\
		III. & $s = \frac{19}{25} $\\
		\hline
	\end{tabularx}
	\\[0.5cm]
	Der Punkt B liegt auf der Ebene $E$.
\end{document}
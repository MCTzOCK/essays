\documentclass[12pt,a4paper]{report}
\usepackage[T1]{fontenc}
\usepackage[utf8]{inputenc}
\usepackage{charter}
\usepackage{ngerman}
\usepackage[left=2cm,right=2cm,top=2cm,bottom=2cm]{geometry}
\usepackage{amsmath}
\renewcommand\thesection{\arabic{section}.} 


\begin{document}
	\section{S. 182 Nr. 1}
	\subsection{a)}
	\begin{align*}
		\overrightarrow x &= \begin{pmatrix}
			1 \\ 1 \\ 2
		\end{pmatrix}
		+ t \cdot \begin{pmatrix}
			0\\ -2 \\ 7
		\end{pmatrix} \\
		\overrightarrow x &= \begin{pmatrix}
			1 \\ 1 \\ 2
		\end{pmatrix}
		+ 1 \cdot \begin{pmatrix}
			0\\ -2 \\ 7
		\end{pmatrix} = \begin{pmatrix}
			1 \\ -1 \\ 9
		\end{pmatrix} \\
		\overrightarrow x &= \begin{pmatrix}
			1 \\ 1 \\ 2
		\end{pmatrix}
		- 1 \cdot \begin{pmatrix}
			0\\ -2 \\ 7
		\end{pmatrix} = \begin{pmatrix}
			1 \\ 3 \\ -5
		\end{pmatrix} \\\\
		&P_0(1|1|2) \\
		&P_1(1|-1|9) \\
		&P_2(1|3|-5)
	\end{align*}
	\subsection{b)}
	\begin{align*}
		\overrightarrow x &= \begin{pmatrix}
			1 \\ 3 \\ -5
		\end{pmatrix} + t \cdot \begin{pmatrix}
			0 \\ -2 \\ 7
		\end{pmatrix}
	\end{align*}
	\section{S. 182 Nr. 2}
	\subsection{a)}
	\begin{align*}
		\begin{pmatrix}
			1 \\ 1
		\end{pmatrix} &= \begin{pmatrix}
			7 \\ 3
		\end{pmatrix} + t \cdot \begin{pmatrix}
			-2 \\ 3
		\end{pmatrix} \\
		1 &= 7 - 2t \ \Leftrightarrow t = 3 \\
		1 &= 3 + 3t\ \Leftrightarrow t = -\frac{2}{3}
	\end{align*}
	\subsection{c)}
	\begin{align*}
		\begin{pmatrix}
			2 \\ 3 \\ -1
		\end{pmatrix} &= \begin{pmatrix}
			7 \\ 0 \\ 4
		\end{pmatrix} + t \cdot \begin{pmatrix}
			5 \\ -3 \\ 5
		\end{pmatrix} \\
		2 &= 7 + 5t\ \Leftrightarrow t = -1 \\
		3 &= 0-3t\ \Leftrightarrow t = -1 \\
		-1 &= 4 + 5t\ \Leftrightarrow t = -1
	\end{align*}
	\section{S. 182 Nr. 6}
	\begin{align*}
		\overrightarrow x_1 &= \begin{pmatrix}
			44 \\ 20
		\end{pmatrix}
		+ t \cdot \begin{pmatrix}
			4 \\ 10
		\end{pmatrix} \\
		\overrightarrow x_ 2 &= t \cdot \begin{pmatrix}
			8 \\ 5
		\end{pmatrix}
	\end{align*}
	\subsection{a)}
	\begin{align*}
		P_{Hafen}(0|0)
	\end{align*}
	\subsection{b)}
	\begin{align*}
		\begin{pmatrix}
			48 \\ 30
		\end{pmatrix} &= \overrightarrow x_1 \\
		\Rightarrow 48 &= 44 + t\cdot 4 \\
		\Rightarrow 30 &= 20 + t\cdot 10 \\
		\Leftrightarrow t &= 1 \\
		\begin{pmatrix}
			48 \\ 30
		\end{pmatrix} &= \overrightarrow x_2 \\
		\Rightarrow 48 &= t \cdot 8 \\
		\Rightarrow 30 &= t \cdot 5 \\
		\Leftrightarrow t &= 6
	\end{align*}
	Das erste Boot erreicht den Punkt nach einer Stunde und das zweite nach 6 Stunden.
\end{document}
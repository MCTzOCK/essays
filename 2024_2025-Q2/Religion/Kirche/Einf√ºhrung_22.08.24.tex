\documentclass[../Religion.tex]{subfiles}

\begin{document}
	\section{Eigene Definition Kirche}
	Die Kirche ist eine Institution die Christen dabei hilft, ihren Glauben auszuleben.
	Die Quantität der Kirchgänge sagt zwar nichts über den Glauben einer Person, aber über die Verbundenheit zu dieser Institution aus.
	\section{Was ist Kirche?}
	\begin{itemize}
		\item Begriff ist enorm vieldeutig.
		\item In Sätzen, wie \dq Die Kirche hat verkündet\dq\ wird von einer bestimmten institution geredet.
		\item Nicht-katholische Christen unterstehen nicht dem Papst
		\item Das Singular des Wortes ist irreführend.
		\item Es gibt mehrere Kirchen (pro Konfession)
		\item Ein Besuch der Kirche umfasst nicht nur das Betreten eines Gebäudes, sondern auch die Teilnahme an der Heiligen Messe
		\item \dq Kirche\dq\ kommt vom griechischen Wort Kyriakón (dem Herrn gehörig)
		\item lexikalisch: Organisationsform von Religion
		\item es gibt keine muslimische oder buddhistische Kirche
		\item nichtchristliche Gruppierungen nutzen auch den Begriff
		\item Die Kirche wurzelt in der nachösterlichen Jüngerschaft
		\item Formell ist die Kirche eine Gemeinschaft aller Getauften
		\item Wer aus der Kirche ausgetreten ist und wieder eintritt kann nicht erneut getauft werden.
		\item Ekklesiologie (die Lehre der Kirche) spricht von einem \dq Grundsakrament\dq
	\end{itemize}
	Der Begriff der Kirche ist enorm vieldeutig.
	In manchen Kontexten ist von einer bestimmten Institution die Rede.
	Christen, die nicht katholisch sind unterstehen auch nicht dem Papst.
	Wichtig ist, dass es nicht \dq die Kirche\dq\ gibt, sondern jede Konfession eine eigene Kirche besitzt. Deswegen ist der Singular des Wortes auch irreführend.
	Sagt ein Mensch, dass er gleich in die Kirche geht, meint er damit nicht nur den reinen Besuch eines Gebäudes, sondern auch die Teilnahme an der Heiligen Messe.
	Das Wort \dq Kirche\dq\ stammt vom griechischen Wort \dq Kyriakón\dq\ ab, welches in etwas \dq dem Herrn gehörig\dq\ bedeutet.
	Lexikalisch gesehen ist die Kirche nur ein Begriff für eine Organisationsform einer Religion.
	Da das Wort einen ausschließlich christlichen Kontext voraussetzt gibt es keine muslimischen oder buddhistischen Kirchen.
	Allerdings gibt es auch nichtchristliche Gruppierungen, die diesen Begriff für sich nutzen.
	Historisch wurzelt die Kirche in der nachösterlichen Jüngerschaft, ist allerdings formell nur eine Gemeinschaft aller Getauften.
	Sollte ein Mensch sich dazu entscheiden, aus der Kirche auszutreten und irgendwann wieder einzutreten, so wird er nicht erneut getauft, da das nur einmal möglich ist.
	Die Ekklesiologie spricht von einem Grundsakrament.
	\section{Definitionen}
	\begin{itemize}
		\item Amtskirche
		\item Kirche als Konfession
		\item Gotteshaus
		\item Eucharistiefeier (umgangssprachlich)
		\item Gemeinschaft aller Getauften
		\item Kirche als Grundsakramen
	\end{itemize}
\end{document}
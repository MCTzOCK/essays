\documentclass[12pt,a4paper]{report}
\usepackage[T1]{fontenc}
\usepackage[utf8]{inputenc}
\usepackage{charter}
\usepackage{ngerman}
\usepackage[left=2cm,right=2cm,top=2cm,bottom=2cm]{geometry}

\begin{document}
	\section{Funktionsweise Kaugummiautomat (23.08.24)}
	\begin{enumerate}
		\item Ein Käufer legt eine Münze mit einem bestimmten Geldwert in das Münzfach
		\item Der Käufer dreht am Rad
		\item Der Automat überprüft den Wert der Münze und schaut, ob dieser mit dem eingespeicherten Wert übereinstimmt. Dies kann entweder durch mechanische,  visuelle oder elektromechanische Vorgänge geschehen.
		\item Ist der Wert der Münze korrekt, wird ein Kaugummi ausgegeben.
		\item Ist der Wert der Münze falsch, so wird diese wieder zurückgegeben.
	\end{enumerate}
\end{document}